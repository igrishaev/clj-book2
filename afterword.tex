
\ifprint
\chapter{Послесловие}
\else
\chapter*{Послесловие}
\fi

С момента выхода первой книги прошло два года, и в мире Clojure многое
изменилось. Все реже язык воспринимают как экзотику. Больше фирм замечены в
поиске специалистов по Clojure — не только западе, но и в России. Clojure
используют в крупных банках (RBI, Сбер). Растет число пользователей в слаке
\footurl{Clojurians}{https://clojurians.slack.com/}[clojurians.\\*slack.com]
и Телеграм-канале \footurl{clojure\_ru}{https://t.me/clojure\_ru}[@clo\-ju\-re\_ru].

Набирает популярность интерпретатор Babashka и его экосистема. Babashka и
GraalVM открыли для Clojure новую область применения: скрипты и быстрый запуск в
облаке AWS Lambda. Автор мечтает развить эту тему в будущей книге.

Материал, что вы прошли в трех главах, по праву считается сложным. Если вы
разобрались в предмете, запустили код и выполнили хотя бы часть задач, ваш
уровень действительно высок. Закрепите новые знания практикой: опробуйте их в
проекте, найдите работу на Clojure или перепишите личный проект.

Желаю читателю успеха во всех начинаниях.

\vspace{1em}

\noindent

\hspace{\fill}\parbox{4cm}{\textit{Иван Гришаев,\\Россия,\\2020--2023}}
