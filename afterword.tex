
\ifprint
\chapter{Послесловие}
\else
\chapter*{Послесловие}
\fi

С момента выхода первой книги прошло два года, и в мире Clojure многое
изменилось. Язык все реже воспринимают как экзотику. Больше фирм замечены в
поиске специалистов по Clojure, причем не только на Западе, но и в России. Clojure
используют в крупных банках (NuBank, RBI). Растет число пользователей в группе
\footurl{Clojurians}{https://clojurians.slack.com/}[clojurians.\\*slack.com]
и Телеграм-канале \footurl{clojure\_ru}{https://t.me/clojure\_ru}[@clo\-ju\-re\_ru].

\index{компании!NuBank}
\index{NuBank}

\index{компании!RBI}
\index{RBI}

Набирает популярность утилита Babashka для запуска скриптов на Clojure. Вокруг
нее растет экосистема со своими проектами и библиотеками. Babashka и GraalVM
открыли для Clojure новую область применения: системное программирование и
запуск в AWS Lambda. Автор мечтает развить эту тему в будущей книге.

\index{Babashka}
\index{утилиты!Babashka}

Материал, что вы прошли в трех главах, по праву считается сложным. Если вы
разобрались в предмете, запустили код и выполнили хотя бы часть задач, ваш
уровень действительно высок. Закрепите знания практикой: внедрите Clojure на
текущей работе или перепишите личный проект.

Желаю читателю успеха во всех начинаниях.

\vspace{1em}

\noindent

\hspace{\fill}\parbox{4cm}{\textit{Иван Гришаев,\\Россия,\\2020--2023}}
