
\section*{Об этой книге}

Перед вами второй том <<Clojure на производстве>>. Это продолжение первой книги,
которая вышла два года назад. Мы продолжим изучать Clojure~--- замечательный язык
с акцентом на неизменяемость и асинхронность. Clojure называют современным
Лиспом, потому что код на нем пишут S-выражениями — то есть со скобками.

По структуре и изложению книга не отличается от первой. Мы подробно рассмотрим
несколько тем, чередуя теорию с практикой. Каждая мысль в тексте подтверждается
кодом и наоборот: сложный код разбит на части и подробно описан текстом.

Как и первый том, продолжение написано на русском языке. Автор много лет пишет
на Clojure и знаком с индустрией и~сообществом. Вас ждут привычные термины
вместо неуклюжих адаптаций. Все примеры и задачи автор взял из реальных
проектов; каждую строчку кода выполнил в REPL.

Коротко о том, что вас ждет. Первая глава расскажет о~зипперах в Clojure. Это
особый способ работы с коллекциями: непривычный, но крайне мощный. О зипперах
мало информации даже на английском языке, и книга закрывает этот недостаток.

Вторая глава посвящена реляционным базам данных, в основном PostgreSQL. Мы
рассмотрим основы SQL, подключение и работу с базой из Clojure. Автор учел все
наболевшие темы: построение сложных запросов, шаблонизацию SQL, работу с
выборкой и все то, о чем забывают другие руководства.

Третья глава охватывает сразу три смежные темы — REPL, Cider и Emacs. Читатель
узнает, что такое REPL и как подключиться к нему из редактора. Мы поговорим о
сетевом протоколе nREPL, о запуске проекта в Docker и на удаленной
машине. Рассмотрим REPL на платформе Javascript и проведем массу экспериментов.

В тексте мы не раз ссылаемся на первую книгу, особенно когда речь идет об
исключениях, системах или Clojure.spec. Это не помешает разобраться с темой,
даже если вы не читали первый том. Все же автор советует ознакомится с ним для
лучшего понимания.

Книга рассчитана на продвинутую аудиторию. Желательно, чтобы у вас был опыт если
не с Clojure, то хотя бы с одним из промышленных языков. Пожелаем читателю
терпения, чтобы пройти книгу до конца.

\section*{Код}

\def\urlcljbooksecond{https://github.com/igrishaev/clj-book2}

Исходный код книги в виде файлов \LaTeX находится в репозитории
\footurl{\code{igri\-sha\-ev/\-clj-book2}}{\urlcljbooksecond}[clj-book2][-1mm].
Если вы нашли опечатку, откройте pull request или issue с описанием проблемы.

\def\urlzipman{https://github.com/igrishaev/zipper-manual}

\def\urlbooksess{https://github.com/igrishaev/book-sessions}

Код первой главы о зипперах доступен в репозитории
\footurl{\code{igr\-isha\-ev/zip\-per-ma\-nu\-al}}{\urlzipman}[Zipper\\*manual]. Код
второй и третьей — в репозитории
\footurl{igrishaev/book-sessions}{\urlbooksess}[Book\\*sessions], пути
\code{src/book/db.clj} и \code{repl-chapter} соответственно.

Используйте код в любых целях, в том числе коммерческих.

\section*{Благодарности}

Автор благодарен стартапу \footurl{Clashapp}{https://huddlesapp.co}[Clashapp]
(ныне Huddles) и его коллективу за полученный опыт. Некоторые техники из этого
проекта нашли место в книге.

Спасибо читателям блога за присланные опечатки и уточнения. С ними текст удалось
улучшить до сдачи в печать.

Особая благодарность Андрею Листопадову за детальные отзывы к черновикам
глав. Посетите его сайт:
\footurl{andreyorst.gitlab.io}{https://andreyorst.gitlab.io}[Andrey\\*Orst].

\ifdmk
Благодарю коллектив издательства <<ДМК Пресс>> за то, что взяли рукопись в
работу. Их усилиями вы читаете эту книгу сейчас.
\fi

\ifridero
Благодарю коллектив издательства Rider\'{o} за то, что взяли рукопись в
работу. Их усилиями вы читаете эту книгу сейчас.
\fi

\section*{Обратная связь}

Присылайте ошибки и замечания на почту \EMAILLINK. Автор обновит
макет, и, возможно, следующий читатель получит исправленную версию книги.
