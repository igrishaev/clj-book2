
\section*{Об этой книге}

Перед вами второй том <<Clojure на производстве>>. Это продолжение первой книги,
которая вышла два года назад. Мы продолжим изучать Clojure~--- замечательный язык
с акцентом на неизменяемость и асинхронность. Clojure называют современным
Лиспом, потому что код на нем пишут S-выражениями — то есть со скобками.

По структуре и изложению книга не отличается от первой. Мы подробно рассмотрим
несколько тем, чередуя теорию с практикой. Каждая мысль в тексте подтверждается
кодом и наоборот: сложный код разбит на части и подробно описан текстом.

Как и первый том, продолжение написано на русском языке. Автор много лет пишет
на Clojure и знаком с индустрией и сообществом. Вас ждут привычные термины
вместо неуклюжих адаптаций. Все примеры и задачи автор взял из реальных
проектов; каждую строчку кода выполнил в REPL.

Коротко о том, что вас ждет. Первая глава расскажет о зипперах в Clojure. Это
особый способ работы с коллекциями: непривычный, но крайне мощный. О зипперах
мало информации даже на английском языке, и книга закрывает этот недостаток.

Вторая глава посвящена реляционным базам данных, в основном PostgreSQL. Мы
рассмотрим основы SQL, подключение и работу с базой из Clojure. Автор учел все
наболевшие темы: построение сложных запросов, шаблонизацию SQL, работу с
выборкой и все то, о чем забывают другие руководства.

Третья глава охватывает сразу три смежные темы — REPL, Cider и Emacs. Читатель
узнает, что такое REPL и как подключиться к нему из редактора. Мы поговорим о
сетевом протоколе nREPL, о запуске проекта в Docker и на удаленной
машине. Рассмотрим REPL на платформе Javascript и проведем массу экспериментов.

В тексте мы не раз ссылаемся на первую книгу, особенно когда речь идет об
исключениях, системах или Clojure.spec. Это не помешает разобраться с темой,
даже если вы не читали первый том. Все же автор советует ознакомится с ним для
лучшего понимания.

Книга рассчитана на продвинутую аудиторию. Желательно, чтобы у вас был опыт если
не с Clojure, то хотя бы с одним из промышленных языков. Пожелаем читателю
терпения, чтобы пройти книгу до конца.

\section*{Благодарности}

Спасибо стартапу Flyerbee, моей первой работе на Clojure. Именно там я закрепил
скромные знания языка.

Я счастлив работать в компании Exoscale в окружении талантливых
инженеров. Многие вещи, не только технические, я узнал в~этом коллективе.

Спасибо Петру Маслову и Евгению Климову за крупные партии найденных
опечаток. Досбол Жантолин внёс важные замечания к последней главе. Молодцы все,
кто указал на ошибки в~комментариях в~блоге.

Алексей Шипилов адаптировал книгу под мобильные устройства и выполнил много
рутинных задач по вёрстке.

Вместе с Евгением Бартовым мы перевели книгу на английский язык. Во время
перевода Евгений нашёл неточности в русской версии, которые мы тоже исправили.

Алексей Иванцов нашёл ошибки во втором издании книги и исправил их в репозитории
на GitHub.

\ifdmk
Благодарю коллектив издательства <<ДМК Пресс>> за то, что взяли рукопись в
работу. Их усилиями вы читаете эту книгу сейчас.
\fi

\ifridero
Благодарю коллектив издательства Rider\'{o} за то, что взяли рукопись в
работу. Их усилиями вы читаете эту книгу сейчас.
\fi

\section*{Обратная связь}

Автор будет признателен за указанные опечатки и неточности. Присылайте их
по~адресу \EMAILLINK. Возможно, в промежутках между тиражами получится обновить
макет, и следующий читатель не~увидит ошибки, о~которой вы сообщили. Ваши
замечания попадут и в английскую версию книги.

\section*{Код}

\iflarge
\setlength{\marginparoffset}{-10mm}
\fi

Исходный код книги в виде файлов \LaTeX{} находится на GitHub в репозитории
\footurl{\texttt{igrishaev/clj-book}}{https://github.com/igrishaev/clj-book}[Clojure\\*book][\marginparoffset]. Если
вы нашли опечатку, откройте pull request или issue с описанием проблемы.

Все фрагменты кода из этой книги записаны в репозитории
\footurl{\code{igrishaev/\-book-sessions}}{https://github.com/igrishaev/book-sessions}[Book\\*sessions]. Вы
можете использовать код в любых целях, в том числе коммерческих.

\section*{Ресурсы}

Следующие ресурсы помогут вам освоить язык и найти на нём работу.

\begin{itemize}

\item
  Официальный сайт \footurl{Clojure}{https://clojure.org}[Clojure]. Его
  разделы <<Getting Started>>, <<Reference>> и <<Guides>> подробно описывают
  язык и экосистему в целом. Прочтите их, даже если уверены в своих знаниях.

\item
  Сообщество в Slack под названием
  \footurl{Clojurians}{https://clojurians.slack.com}[Slack\\*Clo\-ju\-ri\-ans].
  Включает сотни каналов на разные темы, в том числе для отдельных библиотек и
  проектов. Каналы с кодами стран объединяют пользователей по языку. Есть канал
  \verb|#ru| для русскоговорящих пользователей.

\item
  \footurl{Чат в Телеграме}{https://t.me/clojure\_ru}[Telegram\\*cloju\-re\_ru]
  на русском языке. Основные темы: решение проблем, советы по
  оформлению кода, вакансии и поиск работы, анонсы мероприятий.

\item
  \footurl{Ask Clojure}{https://ask.clojure.org}[Ask Clojure]~---
  сервис вопросов и ответов по языку и его окружению, аналог
  StackOverflow.

\end{itemize}
