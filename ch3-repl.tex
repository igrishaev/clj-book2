\chapter{REPL, Cider, Emacs}

\begin{teaser}
Эта глава расскажет о REPL~--- главной особенности Clojure. Так называют интерактивную работу с языком, когда код наращивают постепенно. Мы рассмотрим, что такое REPL-driven development и почему, однажды узнав, от него тяжело отказаться.
\end{teaser}

\index{REPL}

Сочетание REPL происходит от четырех слов: Read, Eval, Print и Loop. Дословно они означают прочитать, выполнить, напечатать и повторить. REPL~--- устойчивый термин, под которым понимают интерактивный режим программы.

\index{Python}
\index{Node.js}
\index{языки!Python}

REPL доступен не только в Clojure, но и в других языках. Как правило, он запускается, если вызвать интерпретатор без параметров. Команды \code{python} или \code{node} запустят интерактивные сеансы Python и Node.js. В Ruby для этого служит утилита \code{irb} (где i означает interactive). REPL поддерживают не только интерпретаторы, но и языки, которые компилируются в байт-код (Java, Scala) или машинный код (Haskell, SBCL).

Несмотря на разнообразие, именно в Лиспе REPL имеет решающее значение. Если в Python или Node.js он считается дополнением, то в Лиспе он необходим. Разработка на любом Лиспе зависит от того, насколько хорошо вы взаимодействуете с REPL. На REPL опираются все инструменты и практики, документация, уроки и так далее.

В мире Лиспа принято понятие REPL-driven development. Это стиль разработки, когда код пишут малыми порциями и запускают в REPL. С таким подходом сразу видно поведение программы. Легче проверить неочевидные случаи, например вызвать функцию с \code{nil} или обратиться к ресурсу, которого не существует.

REPL полезен в работе с данными по сети, например извлечь что-то из источника и исследовать результат. Эта задача идеально ложится на интерактивный режим. Как правило, обращение к источнику предполагает несколько этапов: подготовку запроса, отправку, чтение ответа и поиск нужных полей. Эти шаги проходят интерактивно методом проб и ошибок. Позже мы рассмотрим пример HTTP-запроса в сеансе REPL.

\section{Исторический экскурс}

\def\urllispmachine{https://en.wikipedia.org/wiki/Lisp\_machine}

\def\urldart{https://en.wikipedia.org/wiki/Dynamic\_Analysis\_and\_Replanning\_Tool}

\index{Лисп-машина}

REPL отсчитывает историю от \footurl{первых Лисп-машин}{\urllispmachine}[Lisp machine][5mm]. Это были мейнфреймы с запущенным на них интерпретатором Лиспа. Подобные машины использовали в Xerox для печати, обработки изображений, управления оборудованием, решения задач на оптимизацию и машинного обучения. Разработку Лисп-машин \footurl{поддерживал отдел DARPA}{\urldart}[DART], в том числе потому, что их применяли в военной отрасли.

\index{DARPA}

Золотой век Лисп-машин пришелся на период с 1977 по 1985 год, после чего их популярность пошла на спад. Из-за особенностей архитектуры они не могли конкурировать с процессором x86 и компилируемыми языками~--- в плане как цены, так и быстродействия. В итоге Лисп-машины ушли с рынка, но подход REPL, придуманный полвека назад, навсегда остался в индустрии.

Подход и вправду был инновационным. До него программу набирали в редакторе, компилировали и только потом запускали (для краткости опустим перфокарты и прочую рутину). Процесс был долгим и дорогим. Наоборот, Лисп-машина принимала код и выполняла его мгновенно. Для своего времени Лисп был очень высокоуровневым языком. На нём было легко выразить сложную логику, не отвлекаясь на низкоуровневые проблемы. Именно в Лиспе появился автоматический контроль за памятью и сборщик мусора. Всё это делало Лисп-машину идеальной площадкой для экспериментов.

Интерпретатор Лисп-машины был не просто программой по запуску кода. Фактически он был её операционной системой, потому что имел доступ к регистрам процессора, оперативной памяти и устройствам ввода-вывода. В REPL можно было просмотреть все переменные и функции, переопределить и удалить их. Это свойство~--- полный контроль системой~--- тоже стало неотъемлемой частью REPL.

Clojure, как и другие диалекты Лиспа, активно поддерживает REPL и всё связанное с ним. Без знания REPL работа с Clojure неэффективна. Классический подход, когда сначала вы пишете программу, а потом запускаете, здесь не работает. Цель этой главы~--- показать практическую, REPL-ориентированную разработку, принятую в Clojure.

\section{Пробуем REPL}

Чтобы познакомиться с REPL, запустим его. Это можно сделать несколькими способами.

\def\urllein{https://leiningen.org/}

\index{утилиты!Leiningen}
\index{Leiningen}

\textbf{Первый}~--- установить утилиту \footurl{Leiningen}{\urllein}[Leiningen] для управления проектами на Clojure. Инструкции по установке вы найдете на официальном сайте. Когда утилита установлена, выполните в терминале:

\begin{english}
  \begin{bash}
> lein repl
  \end{bash}
\end{english}

\def\urlinstallclj{https://clojure.org/guides/install\_clojure}

\textbf{Второй} способ~--- установить набор утилит \footurl{Clojure CLI}{\urlinstallclj}[Clojure CLI]. На официальном сайте Clojure описана установка для Linux и MacOS. В системе появятся команды \code{clojure} и \code{clj}. Если вызвать любую из них, запустится REPL:

\index{утилиты!Clojure CLI}
\index{Clojure CLI}

\begin{english}
  \begin{bash}
> clj
> clojure
  \end{bash}
\end{english}

\textbf{Третий} и устаревший способ~--- запустить архив jar командой \code{java}. Старые версии Clojure (до 1.8 включительно) состояли из одного файла, путь к которому передается в параметре \code{-jar}:

\index{Java}

\begin{english}
  \begin{bash}
> java -jar clojure-1.8.0.jar
  \end{bash}
\end{english}

Вариант этой же команды, когда архив находится в classpath и явно указан класс \code{clojure.main}:

\begin{english}
  \begin{bash}
> java -cp clojure-1.8.0.jar clojure.main
  \end{bash}
\end{english}

\def\urlmaven{https://mvnrepository.com}

С версии 1.9 Clojure состоит из нескольких jar-файлов. Библиотека Clojure.spec, которую мы рассмотрели в первой книге, поставляется отдельно. Скачайте jar-файлы из \footurl{репозитория Maven}{\urlmaven}[Maven] в разделах \code{org.clojure/clojure} и \code{org.clojure/spec.alpha}. Далее выполните в терминале:

\begin{english}
  \begin{bash}
> java -cp clojure-1.11.1.jar:spec.alpha-0.3.218.jar \
       clojure.main
  \end{bash}
\end{english}

Запустив REPL любым из способов, вы увидите приглашение:

\begin{english}
  \begin{text}
user=>
  \end{text}
\end{english}

Слово \code{user} означает текущее пространство имен. Если не задано иное, REPL открывается в пространстве \code{user}. Позже мы узнаем, как задать другое пространство или переключить его.

Введите любое выражение на Clojure: число, строку в двойных кавычках или кейворд. Эти значения вычисляются сами в себя:

\begin{english}
  \begin{clojure}
user=> 1
1

user=> :test
:test

user=> "Hello REPL!"
"Hello REPL!"
  \end{clojure}
\end{english}

Задайте глобальную переменную:

\begin{english}
  \begin{clojure}
user=> (def amount 3)
#'user/amount
  \end{clojure}
\end{english}

\noindent
и сошлитесь на неё в выражении:

\begin{english}
  \begin{clojure}
user=> (+ amount 4)
7
  \end{clojure}
\end{english}

Более сложный пример. Определите функцию \code{add}, которая складывает два числа. Введите ее в одну строку:

\begin{english}
  \begin{clojure}
user=> (defn add [a b] (+ a b))
#'user/add
  \end{clojure}
\end{english}

\noindent
и проверьте вызов:

\begin{english}
  \begin{clojure}
user=> (add 2 3)
5
  \end{clojure}
\end{english}

REPL поддерживает ввод нескольких строк за раз. Предположим, мы хотим задать функцию с переносом после параметров, чтобы код выглядел аккуратнее:

\begin{english}
  \begin{clojure}
(defn add [a b]
  (+ a b))
  \end{clojure}
\end{english}

Если напечатать \code{(defn add [a b]} и нажать ввод, по незакрытой скобке REPL определит, что выражение неполное. Ошибки не произойдет, и следующая строка дополнит исходную. Как только скобки станут сбалансированы, REPL выполнит форму.

\begin{english}
  \begin{clojure}
user=> (defn add [a b]
  #_=> (+ a b))
#'user/add
  \end{clojure}
\end{english}

\index{модули!clojure.string}

Подключите любой из модулей Clojure, например встроенный \code{clojure.string} для работы со строками:

\begin{english}
  \begin{clojure}
user=> (require '[clojure.string :as str])
nil
  \end{clojure}
\end{english}

С его помощью разбейте строку или выполните автозамену:

\begin{english}
  \begin{clojure}
user=> (str/split "one two three" #"\s+")
["one" "two" "three"]
  \end{clojure}
\end{english}

\begin{english}
  \begin{clojure}
user=> (str/replace
          "Two minutes, Turkish!",
           #"Two" "Five")

;; "Five minutes, Turkish!"
  \end{clojure}
\end{english}

\index{модули!clojure.inspector}
\index{инспекция}

Модуль \code{clojure.inspector} предлагает примитивный графический отладчик. Его функция \code{inspect-tree} принимает данные и выводит окно Swing с деревом папок. Значок папки означает коллекцию; если его раскрыть, появятся дочерние элементы с иконками файлов. Чтобы изучить переменные среды, выполните:

\begin{english}
  \begin{clojure}
(require '[clojure.inspector :as insp])

(insp/inspect-tree (System/getenv))
  \end{clojure}
\end{english}

\index{классы!System}
\index{переменные среды}

Содержимое окна будет примерно таким:

\vspace{3mm}

\dirtree{%
 .1 \{\}.
 .2 JAVA\_MAIN\_CLASS\_68934=clojure.main.
 .2 LC\_TERMINAL=iTerm2.
 .2 COLORTERM=truecolor.
 .2 LOGNAME=ivan.
 .2 TERM\_PROGRAM\_VERSION=3.3.12.
 .2 PWD=/Users/ivan/work/book-sessions.
 .2 SHELL=/bin/zsh.
}

\vspace{3mm}

Опробуйте случай с ошибкой: поделите число на ноль или сложите число с \code{nil}. REPL не завершится, но выведет исключение на экран:

\begin{english}
  \begin{clojure}
user=> (/ 1 0)
Execution error (ArithmeticException) at ...
Divide by zero
  \end{clojure}
\end{english}

Это правильное поведение: в разработке ошибки случаются часто, и нам бы не хотелось завершать JVM. Однако это справедливо только для сеанса REPL. В боевом запуске программы на Clojure ведут себя как обычно: если исключение не поймано, программа завершится с нулевым кодом.

\index{классы!ArithmeticException}

По умолчанию REPL выводит краткое сообщение об ошибке. Последнее исключение остаётся в переменной \code{*e}. Исследуем её:

\begin{english}
  \begin{clojure}
user=> *e

#error {
 :cause "Divide by zero"
 :via
 [{:type java.lang.ArithmeticException
   :message "Divide by zero"
   :at [c.l.Numbers divide "Numbers.java" 188]}]
 :trace
 [[clojure.lang.Numbers divide "Numbers.java" 188]
  [clojure.lang.Numbers divide "Numbers.java" 3901]
  ...
  [clojure.lang.AFn run "AFn.java" 22]
  [java.lang.Thread run "Thread.java" 829]]}
  \end{clojure}
\end{english}

В первой книге мы рассмотрели, что можно сделать с исключением: напечатать в удобном виде, записать в лог, отправить в систему сборки ошибок.

Если исключения не было, результат остаётся в переменной~\code{*1}. С ней легко избежать повторных вычислений. Это особенно полезно, когда выражение даёт объёмный результат:

\begin{english}
  \begin{clojure}
(into {} (System/getenv))

{"HOME" "/Users/ivan"
 "LC_TERMINAL_VERSION" "3.3.12"
 "USER" "ivan"
 ...}
  \end{clojure}
\end{english}

Чтобы не вычислять его повторно, введите:

\begin{english}
  \begin{clojure}
(get *1 "USER")
;; "ivan"
  \end{clojure}
\end{english}

Переменная \code{*1} полезна для записи в файл. Предположим, мы хотим сохранить переменные среды, чтобы исследовать позже. Для этого введите:

\begin{english}
  \begin{clojure}
(into {} (System/getenv))

(spit "dump.edn" (pr-str *1))
  \end{clojure}
\end{english}

\index{EDN}
\index{функции!slurp}
\index{функции!read-string}

На диске появится файл \code{dump.edn} с данными. Позже мы прочтем его комбинацией \code{slurp} и \code{read-string}:

\begin{english}
  \begin{clojure}
user=> (read-string (slurp "dump.edn"))

{"HOME" "/Users/ivan"
 "LC_TERMINAL_VERSION" "3.3.12"
 "USER" "ivan"
 ...}
  \end{clojure}
\end{english}

В REPL доступны три переменные результата: \code{*1}, \code{*2} и \code{*3}. С~каждым вычислением результаты смещаются: последний будет в \code{*1}, предпоследний в \code{*2} и так далее. Покажем это на примере:

\begin{english}
  \begin{clojure}
user=> 1
1

user=> 2
2

user=> 3
3

user=> (println *1 *2 *3)
3 2 1
  \end{clojure}
\end{english}

\index{функции!load-file}

Чтобы загрузить несколько определений, используйте функцию \code{load-file}. Она принимает один аргумент~--- путь к файлу с кодом на Clojure:

\begin{english}
  \begin{clojure}
(load-file "my_functions.clj")
  \end{clojure}
\end{english}

Эффект аналогичен тому, как если бы вы скопировали содержимое и вставили в REPL. В боевом коде \code{load-file} не используют, потому что такая загрузка делает код неочевидным: не ясно, откуда взялось то или иное определение. Но для экспериментов \code{load-file} подходит идеально.

REPL предлагает макросы для интроинспеции. Выражение \code{(doc ...)} напечатает справку указанной функции, например:

\begin{english}
  \begin{text}
(doc assoc)
-------------------------
clojure.core/assoc
([map key val] [map key val & kvs])
  assoc[iate]. When applied to a map, returns a new map
    of the same (hashed/sorted) type, that contains the
    mapping of ...
  \end{text}
\end{english}

А форма \code{(source ...)}~--- ее исходный код (приведем в сокращении):

\begin{english}
  \begin{clojure}
(source assoc)

(def
 ^{:arglists '([map key val] [map key val & kvs])
   :doc "..."
   :added "1.0"
   :static true}
 assoc
 (fn ^:static assoc
   ([map key val] (clojure.lang.RT/assoc map key val))
   ([map key val & kvs]
    (...)))) ;; truncated
  \end{clojure}
\end{english}

REPL предлагает и другие возможности, о которых мы поговорим позже. Пока что завершите сеанс нажатием \code{Ctrl+D} или выполните \code{(quit)} либо \code{(exit)}.

\section{Более сложный сценарий}

\index{сервисы!Joke API}
\index{Joke API}
\index{Telegram}

REPL удобен не только для быстрых экспериментов; опытные разработчики проводят в нем часы и дни. Одна из причин в том, что REPL~--- лучший способ разведать ситуацию, когда вы не знаете точно, какие данные ожидать от внешних источников.

\def\urljokeapi{https://jokeapi.dev/}

Предположим, мы пишем бота для Telegram, чтобы публиковать шутки для программистов. Понадобится сервис, который бы выступил в роли источника шуток. Быстрый поиск дает нам сервис \footurl{Joke API}{\urljokeapi}[Joke API][-15mm] с удобным API по протоколу HTTP.

Прежде чем писать бота, убедимся в работе сервиса. Для этого понадобятся HTTP-клиент и парсер JSON. Если вы запускаете REPL при помощи \code{lein}, создайте файл \code{project.clj} с содержимым:

\begin{english}
  \begin{clojure}
(defproject repl-chapter "0.1.0-SNAPSHOT"
  :dependencies [[org.clojure/clojure "1.10.0"]
                 [clj-http "3.9.1"]
                 [cheshire "5.8.1"]])
  \end{clojure}
\end{english}

\index{библиотеки!Clj-http}
\index{библиотеки!Cheshire}
\index{Clj-http}
\index{Cheshire}
\index{HTTP}
\index{JSON}

Для утилит Clojure CLI файл \code{deps.edn} выглядит так:

\begin{english}
  \begin{clojure}
{:deps
 {clj-http/clj-http {:mvn/version "3.9.1"}
  cheshire/cheshire {:mvn/version "5.8.1"}}}
  \end{clojure}
\end{english}

Запустите REPL. Обе библиотеки, если ещё не были установлены локально, скачаются на ваш компьютер. Подключите их в сеансе:

\begin{english}
  \begin{clojure}
(require '[clj-http.client :as client])
(require 'cheshire.core)
  \end{clojure}
\end{english}

Подготовим словарь запроса. Мы передаём параметр \code{type=twopart}, чтобы шутка состояла из двух частей (подробнее об этом~--- ниже):

\begin{english}
  \begin{clojure}
(def request
  {:url "https://v2.jokeapi.dev/joke/Programming"
   :method :get
   :query-params {:type "twopart"}
   :as :json})
  \end{clojure}
\end{english}

Получим для него ответ:

\begin{english}
  \begin{clojure}
(def response
  (client/request request))
  \end{clojure}
\end{english}

Каждый шаг мы связываем с переменной при помощи \code{def}, чтобы позже сослаться на него. Такой стиль не подходит для промышленного кода, но приемлем в REPL. Поместим тело в отдельную переменную и напечатаем его:

\begin{english}
  \begin{clojure}
(def data
  (:body response))

{:category "Programming"
 :delivery "They only like chicken NuGet."
 :type "twopart"
 :setup ".NET developers are picky when it comes to food."
 :lang "en"
 :id 49
 :error false
 :safe true
 :flags
 {:nsfw false
  :religious false
  :political false
  :racist false
  :sexist false
  :explicit false}}
  \end{clojure}
\end{english}

\index{инспекция}

Чтобы лучше понять структуру ответа, исследуем его в инспекторе:

\begin{english}
  \begin{clojure}
(require '[clojure.inspector :as insp])

(insp/inspect-tree data)
  \end{clojure}
\end{english}

Видим, что шутка состоит из двух частей: \code{setup} и \code{delivery} (термины можно перевести как <<подводка>> и <<разрешение>>). Такая структура полезна, когда разрешение показывают не сразу, а после паузы или под тегом спойлера. Читатель сможет подумать над продолжением шутки, перед тем как посмотреть ответ. В~нашем случае мы просто объединим обе фразы:

\begin{english}
  \begin{clojure}
(def joke
  (let [{:keys [setup
                delivery]} data]
    (format "%s %s" setup delivery)))

;; .NET developers are picky when it comes to food.
;; They only like chicken NuGet.
  \end{clojure}
\end{english}

\index{Python}
\index{JavaScript}

Если передать необязательный параметр \code{contains}, получим шутку на заданную тему. Например, если это чат о языке Python, будем шутить про JavaScript:

\begin{english}
  \begin{clojure}
(def request
  {:url "https://v2.jokeapi.dev/joke/Programming"
   :method :get
   :query-params {:type "twopart" :contains "javascript"}
   :as :json})

...

;; Why was the JavaScript developer sad?
;; Because they didn't Node how to Express themself!
  \end{clojure}
\end{english}

Итак, мы написали фрагменты кода под каждый шаг, и пора объединить их. Составим функцию, которая принимает язык, про который нужно шутить, и возвращает текст шутки:

\begin{english}
  \begin{clojure}
(defn get-joke
  "Get a random joke about a given language."
  [lang]
  (let [request
        {:url "https://v2.jokeapi.dev/joke/Programming"
         :method :get
         :query-params {:type "twopart" :contains lang}
         :as :json}

        response
        (client/request request)

        {:keys [body]}
        response

        {:keys [setup delivery]}
        body]

    (format "%s %s" setup delivery)))
  \end{clojure}
\end{english}

\pagebreaklarge

Вот что получилось у автора:

\begin{english}
  \begin{clojure}
(get-joke "python")

;; Why did the Python programmer not respond
;; to the foreign mails he got? Because his interpreter
;; was busy collecting garbage."

(get-joke "javascript")

;; How did you make your friend rage?
;; I implemented a greek question mark
;; in his JavaScript code."
  \end{clojure}
\end{english}

Пока что рано завершать REPL: данные, что мы получили с сервера, пригодятся в тестах. Сохраним их в JSON-файл. Для этого выполним:

\begin{english}
  \begin{clojure/lines}
(spit "joke-ok.json"
      (cheshire.core/generate-string
       data {:pretty true}))
  \end{clojure/lines}
\end{english}

Обратите внимание на параметр \code{:pretty} \coderef{3}. С ним JSON будет записан с отступами и переносами строк, что удобно при чтении.

Важно узнать, как ведёт себя сторонний сервис в случае ошибки. Если пошутить о Clojure, получим неприятный результат:

\begin{english}
  \begin{clojure}
(get-joke "clojure")
"null null"
  \end{clojure}
\end{english}

Чтобы понять, почему так случилось, исследуем ответ сервера:

\begin{english}
  \begin{clojure}
(def data
  (-> {:url "https://v2.jokeapi.dev/joke/Programming"
       :method :get
       :query-params {:type "twopart" :contains "clojure"}
       :as :json}
      (client/request)
      (:body)))
  \end{clojure}
\end{english}

Данные:

\begin{english}
  \begin{clojure}
{:error true
 :internalError false
 :code 106
 :message "No matching joke found"
 :causedBy ["No jokes were found that match ..."]
 :additionalInfo "No jokes were found ..."
 :timestamp 1651054623055}
  \end{clojure}
\end{english}

Сервер не нашел подходящих шуток и вернул словарь с полем \code{{:error true}}. Перепишите функцию так, чтобы в случае ошибки мы получили \code{nil}, а не строку с \code{null}.

Негативный ответ тоже пригодится в тестах. Запишите его в файл с понятным именем:

\begin{english}
  \begin{clojure}
(cheshire.core/generate-stream
  data
  (io/writer "joke-err.json")
  {:pretty true})
  \end{clojure}
\end{english}

Итак, разведка выполнена. Мы убедились, что сервис отвечает на запросы. Мы получили данные для тестов. Стало ясно, как ведет себя сервис в случае ошибки. Появились наброски кода, из которых легко составить конечную версию.

Важно, что эти наброски опираются на реальные данные, а не документацию или систему классов. И то, и другое может устареть и вводить в заблуждение о том, что передается по сети. В случае с REPL подобной ошибки быть не может.

Только теперь, с багажом опыта и данных, можно садиться за промышленный код. Мы уже проделали основную работу, и остальная часть не потребует усилий. Все это благодаря REPL, который работает как черновик, отладчик и площадка для экспериментов.

\section{Свой REPL}

\index{REPL}

Чтобы лучше понять устройство REPL, напишем свою реализацию. Подготовьте минимальный проект с файлом \code{project.clj} или \code{deps.edn}. В файл \code{src/my\_repl.clj} поместите следующий код:

\begin{english}
  \begin{clojure/lines}
(ns my-repl
  (:gen-class))

(defn -main [& args]
  (repl))
  \end{clojure/lines}
\end{english}

Форма \code{(:gen-class)} в теле \code{ns} означает, что при компиляции пространства получится одноимённый класс \coderef{2}. Это же пространство должно быть указано главным в файле \code{project.clj}:

\begin{english}
  \begin{clojure}
:main my-repl
  \end{clojure}
\end{english}

Функция \code{-main} служит точкой входа в будущий класс. Она запускает функцию \code{repl}, которую предстоит написать. Вот её черновик:

\begin{english}
  \begin{clojure}
(defn repl []
  (loop []
    (let [input (read-line)
          expr (read-string input)
          result (eval expr)]
      (println result)
      (recur))))
  \end{clojure}
\end{english}

Это бесконечный цикл в четыре шага, каждый из которых мы и рассмотрим.

\subsubsection{Чтение}

\index{REPL!Read}

Функция \code{read-line} читает строку из стандартного канала ввода (stdin). Если канал пуст, система ожидает ввода с клавиатуры. Пользователь набирает текст в терминале и жмет \enter. В~переменной \code{input} окажется строка.

Функция \code{read-string} читает объект Clojure из строки. Числа, строки и другие примитивы выражаются сами в себя. Например, из строки <<1>> получим единицу. Символы остаются невычисленными: строка \code{"(foo bar)"} вернёт список с символами \code{foo} и \code{bar}. В~переменной \code{expr} окажется объект Clojure: символ, строка, число или их коллекция.

\begin{english}
  \begin{clojure}
=> (read-string "(foo [1 false {:foo BAZ}])")

;; (foo [1 false {:foo BAZ}])
  \end{clojure}
\end{english}

\subsubsection{Вычисление (Eval)}

\index{REPL!Eval}
\index{eval}

Функция \code{eval} вычисляет выражение. В отличие от других языков, в Clojure \code{eval} ожидает не строку, а форму. Ей может быть как примитив (число, символ, кейворд), так и коллекция. Примитивы вычисляются сами в себя:

\begin{english}
  \begin{clojure}
=> (eval 1)
1

=> (eval :foo)
:foo
  \end{clojure}
\end{english}
Символы вычисляются в переменные текущего пространства. Например, символ \code{+}
связан с функцией \code{clojure.core/+}, и его вычисление вернёт объект функции:

\begin{english}
  \begin{clojure}
=> (eval '+)
#function[clojure.core/+]
  \end{clojure}
\end{english}

В вычислении можно сослаться на любую переменную, в том числе созданную вами:

\begin{english}
  \begin{clojure}
(def some-text "Hello")

(eval '(str some-text " World!"))

;; "Hello World!"
  \end{clojure}
\end{english}

Вариант посложнее со словарем и \code{update-in}:

\begin{english}
  \begin{clojure}
(def data {:a {:b 0}})

(eval '(update-in data [:a :b] inc))

;; {:a {:b 1}}
  \end{clojure}
\end{english}

Если переменной с таким именем нет, получим исключение:

\begin{english}
  \begin{clojure}
(eval 'dunno)

Syntax error compiling at ...
Unable to resolve symbol: dunno in this context
  \end{clojure}
\end{english}

\subsubsection{Печать (Print)}

\index{REPL!Print}

Следующий шаг~--- печать. Результат \code{eval} выводится на экран обычной функцией \code{println}:

\begin{english}
  \begin{clojure}
(println result)
  \end{clojure}
\end{english}

\subsubsection{Повтор (Loop)}

\index{REPL!Loop}

Оператор \code{loop} переносит нас к первому шагу~--- чтению с клавиатуры,~--- и все повторяется.

\subsubsection{Эксперименты}

Хоть это и крайне сырая версия REPL, она работает. Скомпилируйте проект командой \code{lein uberjar} и запустите jar-файл:

\begin{english}
  \begin{clojure}
java -jar target/uberjar/repl-chapter-standalone.jar
  \end{clojure}
\end{english}

Введите несколько выражений на Clojure, отделяя клавишей \enter. После каждого появится результат вычисления. Приведем фрагмент сеанса:

\begin{english}
  \begin{clojure}
=> (+ 1 2)
3

=> (assoc {:one 1} :two 2)
{:one 1, :two 2}

=> (defn add [a b] (+ a b))
#'clojure.core/add

=> (add 4 5)
9

=> (require '[clojure.string :as str])
nil

=> (str/split "1 2 3" #"\s")
[1 2 3]
  \end{clojure}
\end{english}

Мы поработали со словарями, определили функцию и вызвали её, затем подключили модуль. Даже в таком примитивном REPL доступны все возможности языка.

\subsection{Улучшения}

Предлагаем читателю доработать наш самодельный REPL~--- это будет отличная практика.

\index{REPL!выход}

\subsubsection{Выход из цикла}

На текущий момент REPL завершается нажатием \code{Ctrl+C}, что неудобно. Сделайте так, чтобы какое-то выражение означало остановку. Например, если пользователь ввел кейворд \code{:repl/exit}, REPL завершается. Проверка может выглядеть так:

\begin{english}
  \begin{clojure}
(let [input
      (read-line)

      expr
      (read-string input)]

  (when-not (= expr :repl/exit)
    ...))
  \end{clojure}
\end{english}

При вводе \code{:repl/exit} оператор \code{recur} не сработает, и REPL выйдет из цикла.

\subsubsection{Перехват исключений}

\index{REPL!исключения}
\index{исключения!REPL}

Сейчас, если выражение содержит ошибку, программа завершится аварийно. Исключение~--- не повод завершать REPL: просто напечатаем его и продолжим. Оберните тело цикла в \code{try/catch} с классом \code{Throwable}, чтобы поймать любое исключение:

\begin{english}
  \begin{clojure/lines}
(defn repl []
  (loop []
    (let [[result e]
          (try
            [(-> (read-line)
                 (read-string)
                 (eval))
             nil]
            (catch Throwable e
              [nil e]))]
      (if e
        (binding [*out* *err*]
          (println (ex-message e)))
        (println result))
      (recur))))
  \end{clojure/lines}
\end{english}

Скомпилируйте проект заново и запустите. На этот раз программа не <<вывалится>>, а покажет сообщение и пригласит к дальнейшему вводу:

\begin{english}
  \begin{clojure}
=> 1
1

=> (/ 0 0)
Divide by zero
  \end{clojure}
\end{english}

\index{loop}

Обратите внимание: форма \code{recur} не может быть внутри \code{try/catch}, поэтому идем на уловку. Выражение \code{try} возвращает пару, где первый элемент~--- результат, если не было ошибок, а второй~--- \code{nil} или ошибка, если таковая случилась \coderefs{4--10}. Прием с парой мы рассмотрели в первой книге в главе об исключениях.

В зависимости от того, что мы получили~--- результат или ошибку,~--- выводим результат в \code{stdout} или \code{stderr} при помощи связывания \code{binding}. В данном примере мы только печатаем текст исключения. Больше деталей можно получить функцией \code{print-stack-trace} из модуля \code{clojure.stacktrace}.

\index{стектрейс}

Ещё одно замечание касается формы \code{(if e ...)} \coderef{11}. Мы проверяем на истину именно исключение, а не результат, потому что результат вполне может быть \code{nil}.

\subsubsection{Обработчик исключения}

Чтобы сделать REPL гибче, поместим обработку исключений в отдельную функцию:

\begin{english}
  \begin{clojure}
(defn exception-handler [e]
  (binding [*out* *err*]
    (println (ex-message e))))
  \end{clojure}
\end{english}

С ней логика \code{loop/recur} станет чище:

\begin{english}
  \begin{clojure}
(loop []
  (...
   (if e
     (exception-handler e))))
  \end{clojure}
\end{english}

Пусть функция \code{repl} принимает параметр, чтобы задать свой обработчик исключения. Если он не задан, сработает функция по умолчанию. Для ясности переименуем ее в \code{default-exception-handler}. Далее перепишем \code{repl}:

\begin{english}
  \begin{clojure/lines}
(defn repl
  [& [{:keys [exception-handler]}]]
  (let [ex-handler
        (or exception-handler
            default-exception-handler)]
    (loop []
      (...
        (if e
          (ex-handler e))))))
  \end{clojure/lines}
\end{english}

\pagebreaklarge

До того как мы вступим в цикл, переменной \code{ex-handler} назначается обработчик~--- переданный или заданный по умолчанию \coderef{2}. Эта оптимизация нужна, чтобы не вычислять его на каждом шаге.

Запустите \code{repl} с обработчиком, который печатает тип исключения:

\begin{english}
  \begin{clojure}
(defn -main
  [& args]
  (repl
    {:exception-handler
      (fn [e]
        (println (type e)))}))
  \end{clojure}
\end{english}

Результат:

\begin{english}
  \begin{clojure}
=> (/ 0 0)

java.lang.ArithmeticException
  \end{clojure}
\end{english}

\subsubsection{Красивая печать}

\index{печать с отступами}
\index{pretty print}
\index{модули!clojure.pprint}

Функция \code{println} выводит данные в одну строку, что не подходит для вложенных коллекций. Чтобы читать их с экрана, нужны переносы строк и отступы. Воспользуйтесь функцией \code{pprint} из модуля \code{clojure.pprint}:

\begin{english}
  \begin{clojure}
(ns my-repl
  (:require
   [clojure.pprint :as pprint]))

(defn repl
  [...]
  (loop []
    (let [...]
      (pprint/pprint result)
      (recur))))
  \end{clojure}
\end{english}

Опробуем красивую печать в действии. Понадобится большая коллекция, например словарь переменных среды:

\pagebreaklarge

\begin{english}
  \begin{clojure}
(pprint/pprint
  (into {} (System/getenv)))

{"LEIN_VERSION" "2.9.5",
 "HOME" "/Users/ivan",
 "LC_TERMINAL_VERSION" "3.3.12",
 "USER" "ivan"
 ...
 }
  \end{clojure}
\end{english}

Данные будут напечатаны построчно, что гораздо удобнее для чтения. Обратите внимание, что красивая печать работает только для типов Clojure, поэтому мы приводим вызов \code{(System/getenv)} к словарю функцией \code{into}. Иначе мы получим экземпляр класса \code{UnmodifiableMap}, который не работает с красивой печатью.

\index{классы!UnmodifiableMap}
\index{классы!System}

В первой книге мы упоминали, что на \code{pprint} влияют динамические переменные \code{*print-length*} и \code{*print-level*}~--- максимальные длина и глубина коллекции. Пусть наш REPL позволит изменить эти значения. Если они заданы, цикл запускается в форме \code{binding} с переопределением длины и глубины. Минимальные правки:

\begin{english}
  \begin{clojure}
(defn repl [& [{:keys [print-level
                       print-length]}]]
  (binding [*print-level*
            (or print-level *print-level*)
            *print-length*
            (or print-length *print-length*)]
    ...))

(defn -main [& args]
  (repl {:print-length 3}))
  \end{clojure}
\end{english}

Запустив REPL с настройками выше, введите коллекцию длиннее трёх элементов. При печати вы увидите её усечённую версию:

\begin{english}
  \begin{clojure}
=> [1 2 3 4 5]
[1 2 3 ...]
  \end{clojure}
\end{english}

\subsubsection{Приглашение}

\index{REPL!приглашение}

По умолчанию REPL показывает приглашение \code{user=>}. Это удобно по двум причинам. Во-первых, ясно, в каком пространстве мы находимся сейчас~--- в процессе работы его часто переключают. Во-вторых, стрелка подсказывает, в каком месте от нас ожидают ввода.

Добавьте приглашение перед вводом с клавиатуры (вызовом \code{read-line}). Для начала ограничимся строкой:

\begin{english}
  \begin{clojure}
(loop []
  (print "repl=> ")
  (flush)
  (let [...]
    ...))
  \end{clojure}
\end{english}

Вызов \code{(flush)} необходим, чтобы отправить текст в терминал, не дожидаясь наполнения буфера. С приглашением REPL выглядит живее:

\begin{english}
  \begin{clojure}
repl=> :hello/repl
:hello/repl

repl=> {:foo "test"}
{:foo test}
  \end{clojure}
\end{english}

Логично, чтобы за приглашение отвечала функция \code{get-prompt}, которая принимает текущее пространство:

\begin{english}
  \begin{clojure/lines}
(defn get-prompt
  [this-ns]
  (format "%s=> " (ns-name this-ns)))

(defn repl
  []
  (binding [*ns* *ns*]
    (loop []
      (print (get-prompt *ns*))
      ...)))
  \end{clojure/lines}
\end{english}

\pagebreaklarge

При смене пространства в REPL меняется и приглашение:

\begin{english}
  \begin{clojure}
clojure.core=> (in-ns 'repl-test)

repl-test=> (clojure.core/refer-clojure)
nil

repl-test=> (+ 1 2)
3

repl-test=> (in-ns 'clojure.string)
nil

(blank? "hello")
false
  \end{clojure}
\end{english}

\index{макросы!binding}
\index{binding}

Обратите внимание на форму \code{(binding [*ns* *ns*] ...)} перед \code{loop} \coderef{5}. Без неё не получится сменить пространство: функция \code{in-ns} меняет переменную \code{*ns*} формой \code{set!}, что невозможно вне макроса \code{binding}.

Доработайте REPL так, чтобы можно было задать свой обработчик приглашения. Напишите функцию, которая выводит текущее время или длительность сеанса:

\begin{english}
  \begin{clojure}
18:12=> ...
18:14=> ...

00:00:05=> ...
00:03:34=> ...
  \end{clojure}
\end{english}

\subsubsection{Переменная результата}

Ещё одна полезная доработка~--- хранить последний результат в переменной, чтобы ссылаться на него. Назовём переменную \code{-r} (result).

\index{макросы!with-local-vars}

Исправим REPL так, что цикл окажется в форме \code{with-local-vars}. Этот макрос задает локальные переменные, которые напоминают атомы. Чтобы изменить переменную, вызывают \code{var-set}. Значение получают функцией \code{var-get} или оператором \code{@} \code{(deref)}. Новая версия REPL:

\begin{english}
  \begin{clojure/lines}
(defn repl []
  (with-local-vars [-r nil]
    (loop []
      (let [input (read-line)
            expr (read-string input)
            result
            (case expr
              -r (var-get -r)
              (eval
               \code{(let [~'-r ~(var-get -r)]
                  ~expr)))]
        (var-set -r result)
        (println result)
        (recur)))))
  \end{clojure/lines}
\end{english}

Обратите внимание на то, как вычисляется переменная \code{result} \coderefs{7--11}. С~помощью оператора \code{case} мы проверяем: если поступил символ \code{-r}, вернём значение переменной \code{-r} при помощи \code{var-get}. Ввод \code{-r} считается особенным, потому что вычисляется без \code{eval}.

Чтобы сослаться на \code{-r} в выражении, например \code{(+ -r 3)}, идут на трюк. Форма \code{expr} погружается в макрос \code{let}, где символ \mbox{\code{-r}} связан со значением \code{-r} \coderef{10}. Причина этих махинаций в том, что \code{eval} не учитывает локальные переменные, и без \code{let} мы получим ошибку, что символ \code{-r} неизвестен. Эту проблему мы подробно изучим в разделе про отладку \page{section-debug}, а пока что ограничимся минимально рабочим вариантом.

Запустите REPL и проверьте, что в \code{-r} остается результат последнего вычисления, при этом на него можно сослаться:

\begin{english}
  \begin{clojure}
=> (+ 1 2 3)
6

=> -r
6

=> (* -r 3)
18

=> -r
18
  \end{clojure}
\end{english}

Доработайте REPL так, чтобы, кроме результата, он хранил последнее исключение в переменной \code{-e}. Например:

\begin{english}
  \begin{clojure}
(/ 1 0)
;; ... Stacktrace ...

-e
;; Execution error (ArithmeticException)...
;; Divide by zero
  \end{clojure}
\end{english}

\index{try}
\index{catch}

Для этого добавьте в макрос \code{with-local-vars} переменную \code{[-e nil]}. При помощи \code{try/catch} перехватывайте ошибку. Если что-то поймано, запишите исключение в \code{-e}:

\begin{english}
  \begin{clojure}
(with-local-vars [-r nil -e nil]
  ...
  (try
    ...
    (catch Throwable e
      (var-set -e e))))
  \end{clojure}
\end{english}

\subsubsection{Многострочный ввод}

\index{REPL!многострочный ввод}

До сих пор мы вводили код под одной строке. Теперь мы хотим задать функцию с переносом после сигнатуры:

\begin{english}
  \begin{clojure}
=> (defn add [a b]
    (+ a b))
  \end{clojure}
\end{english}

Если нажать ввод после \code{b]}, произойдёт следующее. Клавиша \enter завершит приём символов, и в переменной окажется строка \code{"(defn add [a b]"}. Функция \code{read-string} не сможет ее прочитать и бросит исключение:

\begin{english}
  \begin{clojure}
=> (read-string "(defn add [a b]")

Execution error at ...
EOF while reading
  \end{clojure}
\end{english}

Чтобы исправить эту неприятность, перед вызовом \code{read-line} следует убедиться, что форма завершена. Для этого проверим строку на баланс скобок: на каждую открывающую приходится закрывающая того же типа (круглая, квадратная, фигурная). Если в строке незакрытые скобки, мы запрашиваем еще одну строку и продолжаем учет скобок. Как только все скобки закрыты, накопленные строки вычисляются как одно выражение.

Чтобы выделить многострочный ввод визуально, каждая следующая строка предваряется отступом:

\begin{english}
  \begin{clojure}
(defn add
..[a b]
..(let [c (+ a b)]
....(* c 3)))
  \end{clojure}
\end{english}

Длина отступа (число точек) равна уровню формы~--- числу вложенных скобок, умноженному на два.

\def\urlstack{https://en.wikipedia.org/wiki/Stack\_(abstract\_data\_type)}

\index{стек}

Для учета скобок подойдет \footurl{стек}{\urlstack}[Stack]~--- структура данных, которая работает по принципу FILO (First In Last Out, первым пришёл~--- последним ушёл). В стек добавляют и извлекают из него элементы. Особенность в том, что извлечь их можно только в обратном порядке. Например, если добавить в стек числа 1, 2, 3, то при извлечении получим 3, 2, 1.

При анализе строки мы перебираем ее символы. Открывающие скобки попадают в стек. Как только мы встретили закрывающую скобку, происходит следующее:

\begin{itemize}

\item
  убираем с вершины стека последнюю скобку;

\item
  убеждаемся, что она парная к найденной. Если это не так, бросаем исключение.

\end{itemize}

Например, если в стеке содержатся элементы \code{(}, \code{[}, \code{\{} и мы встретили закрывающую фигурную скобку \code{\}}, то всё в порядке: она относится к элементу на вершине \code{\{}. Если же попалась квадратная закрывающая \code{]}, это говорит об ошибке в синтаксисе.

\begin{english}
  \begin{text}
"(..[..{..}.." ;; ok
"(..[..{..].." ;; error
  \end{text}
\end{english}

Для начала опишем стек. Это функция, которая порождает функцию, замкнутую на атоме. Внутренняя функция принимает различные команды. Оператор \code{case} определяет их логику. У~функции два тела: команды без аргументов и с одним аргументом.

Стек поддерживает команды:

\begin{itemize}

\item
  \code{:count}, узнать число элементов;

\item
  \code{:empty?}, проверка на пустоту;

\item
  \code{:pop}, извлечь элемент с вершины;

\item
  \code{:push}, добавить элемент.

\end{itemize}

От других команд \code{:push} отличается тем, что принимает аргумент~--- значение, которое добавляют. Поэтому \code{:push} описан во втором теле.

\begin{english}
  \begin{clojure/lines}
(defn make-stack
  []
  (let [-stack
        (atom nil)]
    (fn stack
      ([cmd]
       (case cmd
         :count (count @-stack)
         :empty? (zero? (count @-stack))
         :pop (let [item (first @-stack)]
                (swap! -stack rest)
                item)))
      ([cmd arg]
       (case cmd
         :push
         (swap! -stack conj arg))))))
  \end{clojure/lines}
\end{english}

\index{функции!conj}
\index{функции!first}
\index{функции!rest}

Изначально атом хранит \code{nil}, при добавлении элемента к которому получится список. Функция \code{conj} добавляет элемент в голову списка. Вот почему вершину стека получают функцией \code{first}, а усекают функцией \code{rest} \coderefs{8 и~9}.

\pagebreaklarge

Стек в действии:

\begin{english}
  \begin{clojure}
(def s (make-stack))

(s :count)  ;; 0
(s :empty?) ;; true
(s :push 1) ;; (1)
(s :push 2) ;; (2 1)
(s :push 3) ;; (3 2 1)
(s :count)  ;; 3
(s :pop)    ;; 3
(s :pop)    ;; 2
(s :pop)    ;; 1
(s :empty?) ;; true
  \end{clojure}
\end{english}

Наш стек является изменяемым объектом. Более <<кложурный>> способ был бы в том, чтобы сделать его неизменяемым: каждая операция возвращала бы его копию подобно функциям \code{assoc} или \code{update}. Но для разнообразия мы решили поработать с изменяемым стеком. Читатели, знакомые с Java, могут использовать класс \code{java.util.Stack} со схожими возможностями.

\index{классы!Stack}

Анализ строки сводится к тому, чтобы сперва наполнить стек открывающими скобками, а затем опустошить закрывающими. Если в итоге стек пуст, строка сбалансирована.

Напишем функцию \code{multi-input}, которая читает ввод с клавиатуры до тех пор, пока выражение не сбалансировано. Считанные строки объединяются пробелом. При вводе очередной строки покажем уровень вложенности отступами. Готовый алгоритм:

\begin{english}
  \begin{clojure}
(defn multi-input []
  (let [stack (make-stack)]
    (loop [result ""]
      (print (make-indent stack))
      (flush)
      (let [line (read-line)
            result (str result " " line)]
        (consume-line stack line)
        (if (stack :empty?)
          result
          (recur result))))))
  \end{clojure}
\end{english}

Код довольно короткий и опирается на две служебные функции. Первая \code{make-indent} строит отступ из точек; его длина равна двойному числу элементов в стеке:

\begin{english}
  \begin{clojure}
(defn make-indent [stack]
  (str/join (repeat (* (stack :count) 2) ".")))
  \end{clojure}
\end{english}

\index{балансировка скобок}

Вторая функция \code{consume-line} сложнее. Она принимает стек и строку, проходит по символам и корректирует стек. Для корректировки нужен словарь парных скобок. Его ключи и значения~--- символы, объекты \code{java.lang.Character}:

\begin{english}
  \begin{clojure}
(def brace-pairs
  {\( \)
   \[ \]
   \{ \}})
  \end{clojure}
\end{english}

Понадобится зеркальная копия этого словаря, чтобы по закрывающей скобке найти открывающую:

\begin{english}
  \begin{clojure}
(def brace-oppos
  (into {} (for [[k v] brace-pairs]
             [v k])))
  \end{clojure}
\end{english}

Код насыщения стека:

\begin{english}
  \begin{clojure/lines}
(defn consume-line [stack line]
  (doseq [char line]
    (cond
      (contains? brace-pairs char)
      (stack :push char)

      (contains? brace-oppos char)
      (let [char-oppos
            (get brace-oppos char)
            char-lead
            (stack :pop)]
        (when-not (= char-lead char-oppos)
          (throw
           (new Exception
                (format "Unbalanced expression: %s...%s"
                        char-lead char))))))))
  \end{clojure/lines}
\end{english}

Логика следующая: если символ~--- открывающая скобка (входит в \code{brace-pairs}), добавить её в стек \coderef{4}. Если закрывающая (входит в \code{brace-oppos}), найти по ней открывающую \coderef{8}. Далее получить элемент с вершины стека методом \code{:pop} (переменная \code{char-lead}). Если переменные не равны, бросить исключение \coderef{12}.

Вернитесь в функцию \code{repl} и замените \code{(read-line)} на \code{(multi-input)}. Скомпилируйте проект и опробуйте REPL в действии. Вот что получилось у автора:

\begin{english}
  \begin{clojure}
=> (+ 1 2 3
=> ..3 4 5)
18

=> (defn add [a b]
=> ..(+ a b))
#'clojure.core/add

=> (assoc-in {:foo 1
=> ....:bar 2
=> ....:baz 3
=> ....}
=> ..[:test :hello]
=> ..3
=> ..)
{:bar 2 :baz 3 :foo 1 :test {:hello 3}}
  \end{clojure}
\end{english}

С этим улучшением гораздо легче ввести длинное выражение. Проверьте код с неверными скобками. Исключение подскажет, какие именно скобки были причиной ошибки:

\begin{english}
  \begin{clojure}
=> (+ 1 2 3]
java.lang.Exception: Unbalanced expression: (...]
  \end{clojure}
\end{english}

Заметим, что попытка извлечь элемент из пустого стека тоже считается ошибкой. Это случится в выражении \code{(+ 1 2 3))}. Предпоследняя скобка очистит стек, но последняя обратится к пустому стеку, что говорит о нарушении. Без доработок мы получим сообщение с \code{null}, что понятно:

\begin{english}
  \begin{clojure}
Unbalanced expression: null...)
  \end{clojure}
\end{english}

Перепишем команду \code{:pop} так, чтобы учесть этот случай:

\begin{english}
  \begin{clojure}
:pop (if (empty? @-stack)
       (throw (new Exception "Stack is empty!"))
       (let [item (first @-stack)]
         (swap! -stack rest)
         item))
  \end{clojure}
\end{english}

Если ввести код с лишними скобками на конце, получим исключение:

\begin{english}
  \begin{text}
java.lang.Exception: Stack is empty!
  \end{text}
\end{english}

Подсчет скобок может дать осечку для внутренних строк. Предположим, мы объявили строку со смайликом внутри:

\begin{english}
  \begin{clojure}
(def text "Hello Clojure :-)")
  \end{clojure}
\end{english}

Наш анализатор не знает, что нужно игнорировать скобки внутри строки. В результате получим ошибку о пустом стеке. Доработайте \code{(multi-input)} так, чтобы при переходе через двойную кавычку включался режим <<в строке>>, когда скобки не добавляются в стек. При выходе из строки флаг отключается.

\subsection{REPL в REPL}

\index{REPL!вложенный}
\index{вложенный REPL}

Еще одна задача: что случится, если вызвать в сеансе функцию \code{(repl)}? Другими словами, запустить REPL в REPL?

Ответ: вы запустите новый сеанс, а прежний повиснет в его ожидании. В новом REPL доступны изменения, сделанные раньше. При завершении вы вернётесь в прежний REPL и продолжите работу. Покажем это на примере. Изначальный REPL:

\begin{english}
  \begin{clojure/lines}
=> (+ 1 2)
3

=> (def x (+ 1 2))
#'my-repl/x
  \end{clojure/lines}
\end{english}

\pagebreaklarge

Переход во внутренний. Переменная \code{x}, объявленная во внешнем сеансе, доступна и здесь:

\begin{english}
  \begin{clojure}
=> (repl)

=> (* x x)
9
  \end{clojure}
\end{english}

Ввод \code{:repl/exit} возвращает нас в прежний REPL, не завершая программу:

\begin{english}
  \begin{clojure}
=> :repl/exit
nil

=> (println "still in the REPL")
;; still in the REPL
nil
  \end{clojure}
\end{english}

Усложним сценарий, поместив \code{(repl)} в середину вычислений:

\begin{english}
  \begin{clojure/lines}
=> (let [a 1 b 2] (repl) (+ a b))

=> (println "debug mode")
nil

=> :repl/exit
3
  \end{clojure/lines}
\end{english}

В этом примере форма \code{let} повиснет до тех пор, пока не завершится внутренний REPL. Пребывая в нем, мы выполнили посторонее действие \coderef{3}. После выхода получим результат главной формы~--- тройку.

Вложенные сеансы встречаются редко, но они не должны вводить вас в ступор. С точки зрения программы это то же самое, что бесконечный цикл в бесконечном цикле: при завершении внутреннего вы вернетесь во внешний. Вложенность сеансов ограничена только ресурсами компьютера.

\subsection{Доступ к локальным переменным}

\index{REPL!локальные переменные}
\index{локальные переменные}

Должно быть, вы догадались, что REPL удобен в качестве отладчика. Он прерывает код и поэтому работает как точка останова. С его помощью мы бы остановились на сложном участке кода, чтобы понять текущее состояние программы.

Недостаток нашего REPL в том, что у него нет доступа к локальным переменным. Предположим, мы добавили REPL в функцию с аргументами и вызвали ее:

\begin{english}
  \begin{clojure}
(defn add [a b]
  (repl)
  (+ a b))

(add 1 2)
  \end{clojure}
\end{english}

Логично ожидать, что при вводе \code{a} или \code{b} мы получим единицу и двойку. Однако это не так~--- сославшись на них, увидим исключение о том, что символ неизвестен:

\begin{english}
  \begin{clojure}
=> a
;; java.lang.RuntimeException
;; Unable to resolve symbol: a in this context
  \end{clojure}
\end{english}

Как мы упоминали выше, функция \code{eval} учитывает только глобальные переменные, заданные при помощи \code{def}. Предупреждая ваше огорчение, скажем: всё-таки можно сделать так, чтобы локальные переменные были доступны. В середине главы мы напишем свой отладчик, где эта проблема решена \page{section-own-debugger}. Пока что предложим читателю подумать над этим вопросом.

На этом мы закончим работу над собственным REPL и двинемся дальше.

\section{Полезные функции REPL}

\index{модули!clojure.repl}

Модуль \code{clojure.repl} содержит функции и макросы для интерактивной работы. В основном они служат для информации о переменных и окружении. Подключите модуль командой \code{use}, чтобы объединить его с текущим пространством:

\begin{english}
  \begin{clojure}
(use 'clojure.repl)
  \end{clojure}
\end{english}

Функция \code{apropos} ищет определение по строке или регулярному выражению. Для слова \code{"update"} найдутся следующие кандидаты:

\begin{english}
  \begin{clojure}
(apropos "update")

(clojure.core/update
 clojure.core/update-in
 clojure.core/update-keys
 clojure.core/update-proxy
 clojure.core/update-vals)
  \end{clojure}
\end{english}

Если объявить переменную со словом <<update>> в названии и выполнить поиск еще раз, в выборке окажется символ \code{user/up\-da\-ted-result}.

\begin{english}
  \begin{clojure}
(def updated-result 42)

(apropos "update")

(clojure.core/update
 clojure.core/update-in
 my.project/updated-result
 ...)
  \end{clojure}
\end{english}

Макрос \code{dir} выводит все публичные переменные пространства:

\begin{english}
  \begin{clojure}
(dir clojure.string)

blank?
capitalize
ends-with?
...
  \end{clojure}
\end{english}

\index{макросы!doc}

Знакомый макрос \code{(doc ...)} печатает документацию переменной. Если это функция или макрос, вы увидите параметры вызова.

\begin{english}
  \begin{text}
(doc +)
-------------------------
clojure.core/+
([] [x] [x y] [x y & more])
  Returns the sum of nums. (+) returns 0.
  Does not auto-promote longs, will throw on overflow.
  See also: +'
  \end{text}
\end{english}

\index{стектрейс}

Если возникло исключение, REPL напечатает только его класс и сообщение, чтобы не тратить место на стек-трейс. Функция \code{pst} (сокращение от \textbf{p}rint \textbf{s}tack \textbf{t}race) выводит данные о последнем исключении.

\begin{english}
  \begin{clojure}
=> (/ 1 0)
Execution error (ArithmeticException) at user/eval175
Divide by zero

=> (pst)
ArithmeticException Divide by zero
  clojure.lang.Numbers.divide (Numbers.java:190)
  user/eval175 (NO_SOURCE_FILE:1)
  clojure.main/repl/fn--9215 (main.clj:458)
  ...
  \end{clojure}
\end{english}

\def\urlclojurerepl{https://clojuredocs.org/clojure.repl}

Эти и другие возможности пригодятся вам в долгих сеансах REPL. Ознакомьтесь с ними \footurl{в документации}{\urlclojurerepl}[clojure.\\repl] к модулю \code{clojure.repl}.

\section{REPL в редакторе}

\index{REPL!в редакторе}

До сих пор мы набирали код в терминале, что не совсем удобно. Терминал подходит для коротких команд, но плохо справляется с многострочным вводом. Будет правильно набрать код в редакторе, а затем скопировать в терминал. Код останется в файле, и не придётся набирать его в следующий раз.

Со временем вы заметите, что переключение между редактором и терминалом отнимает время. Было бы здорово связать редактор и REPL напрямую. Вы набираете код и с помощью команды выполняете его в REPL. В отдельной области редактор показывает результат. С таким подходом доступна мощь обеих сред: REPL и редактора.

\index{Emacs}
\index{редакторы!Emacs}
\index{Common Lisp}
\index{Scheme}

\def\urlemacs{https://www.gnu.org/software/emacs/}

Описанный способ предлагает \footurl{Emacs}{\urlemacs}[Emacs]~--- текстовый редактор с историей более сорока лет. Emacs запускает любой Лисп, будь то Common Lisp, Scheme или Clojure, и управляет им из редактора. В~терминах Emacs запущенный Лисп называется внешним (external) в противоположность встроенному диалекту Emacs Lisp. Режим, когда код вычисляется внешним Лиспом, называется inferior lisp mode (анг. inferior~--- низший). Название объясняется тем, что, поскольку режим нацелен на любой Лисп, он поддерживает только базовые возможности.

Проведем короткий сеанс REPL из Emacs. Запустите редактор и выполните команду

\index{Emacs!inferior-lisp}

\begin{english}
  \begin{text}
M-x inferior-lisp
  \end{text}
\end{english}

Emacs запросит путь к интерпретатору Лиспа. Введите \code{clojure} или \code{lein repl} в зависимости от того, какая утилита у вас установлена. Чтобы не указывать ее каждый раз, объявите в файле \code{\tilde{}/.emacs} переменную \code{inferior-lisp-program}. По умолчанию она вступит в силу после перезагрузки редактора, но этого можно избежать, выполнив в буфере \code{*scratch*} одно из выражений:

\begin{english}
  \begin{lisp}
(setq inferior-lisp-program "clojure") ;; C-j
(setq inferior-lisp-program "lein repl") ;; C-j
  \end{lisp}
\end{english}

Emacs запустит процесс и соединится с его каналами ввода и вывода. В буфере \code{*inferior-lisp*} появится сеанс REPL. Он работает как в терминале: ожидает выражение, вычисляет, печатает и снова ожидает.

\begin{english}
  \begin{clojure}
Clojure 1.10.1
user=> (+ 1 2 3)
6
  \end{clojure}
\end{english}

\index{Emacs!буфер}

Поскольку это буфер Emacs, нам доступно больше возможностей. Можно свободно перемещать по нему курсор, копировать и вставлять код, искать в прямом и обратном направлениях, сохранить буфер в файл и многое другое.

Ряд команд передают код из редактора в REPL без ручного копирования. Перейдите в буфер с кодом на Clojure и включите режим Лиспа командой

\begin{english}
  \begin{text}
M-x lisp-mode
  \end{text}
\end{english}

\index{Emacs!lisp-mode}

Установите курсор после закрывающей скобки любого выражения, например \code{(+ 1 2)}. Выполните команду \code{M-x lisp-eval-last-sexp}, которая означает вычислить последнее S-выражение. В буфере \code{*inferior-lisp*} появится результат:

\begin{english}
  \begin{text}
user=> 3
  \end{text}
\end{english}

Эффект аналогичен тому, как если бы вы скопировали код, вставили в REPL и нажали \enter. Заметим, что выражение не обязательно вычислять целиком. Можно выполнить форму, которая находится внутри другой формы. Предположим, в вашем файле следующий код:

\begin{english}
  \begin{clojure/lines}
(let [a 1 b 2]
  (println "inner form") |
  (+ a b))
  \end{clojure/lines}
\end{english}

\index{S-выражение}
\index{Emacs!lisp-eval}

Подведите курсор на место вертикальной черты \coderef{2}. Выполните \code{M-x lisp-eval-last-sexp}, и REPL вычислит \code{(println "inner form")}. Ошибки не будет, потому что \code{(println ...)} не зависит от переменных \code{a} и \code{b}. Если же вычислить \code{(+ a b)}, получим ошибку, что символы неизвестны.

Команды с приставкой \code{lisp-eval-...} отвечают за то, какую часть файла выполнить в REPL. Например, \code{lisp-eval-region} выполнит только выделенную область, а \code{lisp-eval-defun}~--- функцию, на которой сейчас установлен курсор. Команды с окончанием \code{...-and-go} делают то же самое, но дополнительно переключат вас в REPL.

Проделайте несколько упражнений в REPL из Emacs. Подключите встроенные модули, объявите несколько функций, спровоцируйте исключение.

Обратите внимание, что команды \code{lisp-eval-...} вычисляют код без учета текущего пространства. Контроль за тем, какое пространство активно в данный момент, ложится на вас. Если вы работаете с двумя и более модулями, это станет проблемой. Легко объявить функцию в одном пространстве:

\begin{english}
  \begin{clojure}
(ns test1)

(defn add [a b]
  (+ a b))
  \end{clojure}
\end{english}

\noindent
и вызвать в другом, что приведёт к ошибке:

\begin{english}
  \begin{text}
(ns test2)

(add 1 2)

test2=> Syntax error compiling at (REPL:1:1).
Unable to resolve symbol: add in this context
  \end{text}
\end{english}

Эта проблема решена в более продвинутых системах, о которых мы скоро поговорим.

REPL в Emacs кажется примитивным подходом, но на самом деле это не так. Вам доступны все возможности Clojure и Emacs одновременно. Для эффективной работы требуется немного команд: выполнить S-выражение, регион или определение.

\def\urlbatsov{https://batsov.com/articles/2014/12/04/introducing-inf-clojure-a-better-basic-clojure-repl-for-emacs/}

\footurl{По словам Рича Хикки}{\urlbatsov}[batsov.\\com][3mm], автора Clojure, он работал над языком, используя Emacs и режим \code{inferior-lisp}. Это подтверждает: можно достичь значимых результатов малыми средствами. И хотя для Clojure созданы более мощные инструменты, полезно знать этот спартанский метод.

\subsection{Недостатки}

Способ, когда Emacs запускает внешний Лисп, не лишён недостатков. Перечислим основные из них.

Обмен данными между средами происходит по стандартным каналам операционной системы (stdin, stdout и stderr). Скорость их передачи ниже, чем по сети, что заметно на больших файлах.

Интерпретатор Лиспа, библиотеки и окружение должны быть установлены локально. Без специальных ухищрений нельзя подключиться к Лиспу, запущенному на удалённой машине.

REPL передает плоский текст, из-за чего Emacs показывает результат без каких-либо улучшений. Это сделано намеренно, поскольку режим \code{inferior-lisp} рассчитан на любой REPL, будь то Common Lisp, Racket или Clojure.

Заметим, что Clojure поддерживает сетевой режим в REPL. Чтобы его включить, передайте команде \code{clojure} следующий аргумент:

\index{REPL!сетевой режим}

\begin{english}
  \begin{bash}
clojure -J-Dclojure.server.repl=\
  "{:port 5555 :accept clojure.core.server/repl}"
  \end{bash}
\end{english}

С ним REPL принимает ввод не только с клавиатуры, но и с порта 5555. Чтобы это проверить, подключимся к серверу через \code{telnet} и введём код на Clojure:

\begin{english}
  \begin{text}
> telnet 127.0.0.1 5555

Trying 127.0.0.1...
Connected to localhost.
Escape character is '^]'.
user=> (defn add [a b] (+ a b))
#'user/add
  \end{text}
\end{english}

В результате \code{telnet} работает как обычный REPL; разница в том, что данные передаются по сети. Подключитесь по \code{telnet} со второй машины, на которой нет Clojure. Введите \code{(add 1 2)}, чтобы убедиться: изменения, сделанные на первой машине, доступны второй.

\index{Telnet}

Хотя сетевой режим снимает одну из проблем, озвученных выше, особой популярности он не получил. Данные по-прежнему передаются плоским текстом, что мешает эффективному обмену. Так появился проект nREPL, который закрывает этот и другие недостатки.

\section{Знакомство с nREPL}

\def\urlnreplorg{https://nrepl.org}

\index{nREPL}

В названии \footurl{nREPL}{\urlnreplorg}[nREPL] буква n означает network, то есть сетевой REPL. Проект нацелен на то, чтобы обеспечить сеанс REPL по сети. В отличие от терминала, nREPL обладает более сложной архитектурой; перечислим её главные свойства.

Сервер nREPL принимает команды по протоколу TCP. С~одним проектом работают несколько клиентов. Сервер может быть запущен локально, на удалённой машине или в виртуальном окружении (Docker, VirtualBox).

\index{Docker}

Сообщения nREPL обладают структурой. Каждое из них содержит номер сеанса, тип операции, аргументы и пространство имен. Сообщение легко дополнить или изменить при помощи промежуточных слоев~--- middleware.

nREPL опирается на транспорт сообщений. Транспортом называют соглашение о том, в каком виде передавать сообщения. По умолчанию nREPL предлагает транспорты Bencode, EDN и TTY. Создать новый транспорт означает расширить протокол библиотеки.

Обычный REPL работает синхронно: получив команду с клавиатуры, он не принимает текст до тех пор, пока не вычислит и не напечатает результат. nREPL устроен асинхронно, когда команда выполняется в отдельном потоке. В ответ на одну операцию сервер может прислать несколько сообщений. По специальному полю клиент определяет, последнее это сообщение или нет.

Технически nREPL~--- библиотека, доступная в Clojars. У нее нет зависимостей, что упрощает развитие и поддержку. Несмотря на свою роль, nREPL остаётся отдельным, а не встроенным модулем. С таким подходом он не зависит от цикла обновлений Clojure.

\subsection{Запуск nREPL}

Чтобы запустить nREPL вместо обычного REPL, добавьте библиотеку в проект. Если вы пользуетесь lein, откройте файл \code{project.clj} и расширьте зависимости:

\begin{english}
  \begin{clojure}
{:dependencies
 [... [nrepl/nrepl "0.9.0"]]}
  \end{clojure}
\end{english}

\index{lein}
\index{утилиты!lein}

Сохраните файл и выполните \code{lein repl}. Утилита \code{lein} устроена так, что если nREPL найден в зависимостях, ему отдаётся предпочтение. Убедиться, что вы запустили именно nREPL, можно по фразе <<nREPL server started>>, которая появится в терминале:

\begin{english}
  \begin{text}
> lein repl

nREPL server started on port 52002 on host 127.0.0.1
Clojure 1.10.1
OpenJDK 64-Bit Server VM 11.0.12+6-jvmci-21.2-b08
    Docs: (doc function-name-here)
          (find-doc "part-of-name-here")
  Source: (source function-name-here)
 Javadoc: (javadoc java-object-or-class-here)
    Exit: Control+D or (exit) or (quit)
 Results: Stored in vars *1, *2, *3, an exception in *e

user=>
  \end{text}
\end{english}

В случае с \code{deps.edn} объявите профиль \code{:nrepl}:

\begin{english}
  \begin{clojure/lines}
{:aliases
 {:nrepl
  {:extra-deps
   {nrepl/nrepl {:mvn/version "0.9.0"}}
   :main-opts ["-m" "nrepl.cmdline" "-i"]}}}
  \end{clojure/lines}
\end{english}

Ключ \code{-i} в \code{:main-opts} \coderef{5} означает интерактивный режим, то есть с вводом с клавиатуры. Без него nREPL работает в <<безголовом>> (headless) режиме, слушая только сетевой порт. Запустите утилиту \code{clj} с профилем \code{:nrepl}:

\begin{english}
  \begin{text}
> clj -M:nrepl

nREPL server started on port 55113 on host localhost
nREPL 0.9.0
Clojure 1.11.1
OpenJDK 64-Bit Server VM 11.0.12+6-jvmci-21.2-b08
Interrupt: Control+C
Exit:      Control+D or (exit) or (quit)
user=>
  \end{text}
\end{english}

На первый взгляд nREPL не отличается от обычного REPL. Он по-прежнему принимает ввод с клавиатуры, вычисляет и печатает результат. Истинная мощь nREPL проявляется в работе из редактора, и уже скоро мы дойдём до этого раздела.

После запуска nREPL вы обнаружите файл \code{.nrepl-port} в папке проекта. Если не указать порт, nREPL случайно выберет свободный для подключения. Дополнительно он запишет порт в файл, чтобы редактор прочитал его, не запрашивая в диалоге у пользователя.

Мы указали nREPL в основных зависимостях проекта~--- векторе \code{:dependencies} формы \code{defproject}. Поскольку nREPL относится к разработке, поместим его в профиль \code{:dev}:

\begin{english}
  \begin{clojure}
:profiles
{:dev
  {:dependencies
    [[nrepl/nrepl "0.9.0"]]}
  \end{clojure}
\end{english}

\index{lein!uberjar}
\index{uberjar}

При запуске \code{lein repl} библиотека будет доступна, потому что профиль \code{:dev} активен по умолчанию. При сборке проекта nREPL не окажется в зависимостях. Это легко проверить, вызвав команду \code{deps :tree} с профилем \code{uberjar}:

\begin{english}
  \begin{bash}
> lein with-profile uberjar deps :tree | grep nrepl
;; nothing
  \end{bash}
\end{english}

Быстро окажется, что nREPL нужен во всех проектах.Чтобы не добавлять его в каждый файл \code{project.clj}, прибегают к пользовательскому профилю. Создайте файл \code{\tilde{}/.lein/profiles.clj} со словарём внутри. В поле \code{:user} укажите зависимости, нужные только вам. Утилита \code{lein} объединит его с полем \code{:profiles} при запуске.

\begin{english}
  \begin{clojure}
{:user
  {:dependencies
    [[nrepl/nrepl "0.9.0"]]}}
  \end{clojure}
\end{english}

Теперь по команде \code{lein repl} запусится nREPL, даже если его нет в \code{project.clj}. Это полезно, когда в проекте несколько человек и их редакторы требуют разных версий nREPL (например, Cider и Calva). Каждый укажет свою версию в файле \code{\tilde{}/.lein/profiles.clj}, избежав конфликта.

Если вы используете Clojure CLI, похожий файл называется \code{\tilde{}/.clojure/deps.edn}. При запуске \code{clj} или \code{clojure} он дополняет текущий файл \code{deps.edn}. Поместите в него профиль \code{:nrepl}, созданный выше. Чтобы подчеркнуть, что это локальный профиль, добавьте ему пространство \code{local}:

\begin{english}
  \begin{clojure}
{:aliases
 {:local/nrepl
  {:extra-deps {nrepl/nrepl {:mvn/version "0.9.0"}}
   :main-opts ["-m" "nrepl.cmdline" "-i"]}}}
  \end{clojure}
\end{english}

Включите проект командой

\begin{english}
  \begin{bash}
> clj -M:local/nrepl
  \end{bash}
\end{english}

Поведение nREPL меняют с помощью параметров. В \code{lein} для этого служит ключ \code{:repl-options}. Перечислим опции, которые понадобятся чаще других.

\begin{itemize}

\item
  \code{:port}~--- сетевой порт, по которому nREPL принимает сообщения от клиентов. Если не задан, будет выбран случайно. В редких случаях порт указывают явно, например когда nREPL запущен в Docker или на удалённой машине. Эти случаи мы рассмотрим в конце главы \page{section-repl-docker}.

\item
  \code{:prompt}~--- функция приглашения. Принимает пространство имён и по умолчанию выводит его имя.

\item
  \code{:init-ns}~--- пространство, которое nREPL загрузит при запуске. В разработке используют \code{dev}, \code{user} или \code{sandbox}~--- своего рода песочницу с запуском системы, прогоном миграций другими служебными функциями.

\end{itemize}

Пример с нестандартными параметрами:

\begin{english}
  \begin{clojure}
{:repl-options
  {:port 9911
   :init-ns dev
   :prompt (fn [current-ns]
             (format "[%s] >> " current-ns))}}
  \end{clojure}
\end{english}

Вот что получим при запуске \code{lein repl}:

\begin{english}
  \begin{bash}
[dev] >> (+ 1 2 3)
6
  \end{bash}
\end{english}


\def\urlleinsample{https://github.com/technomancy/leiningen/blob/master/sample.project.clj\#L368}
\def\urlnreplopt{https://nrepl.org/nrepl/usage/server.html}

Остальные настройки вы найдёте в документации \footurl{Leiningen}{\urlleinsample}[Sample project.clj][-30mm] и \footurl{nREPL}{\urlnreplopt}[nREPL server][-10mm].

\subsection{Внутреннее устройство}

\def\urlring{https://github.com/ring-clojure/ring}

\index{Ring}
\index{nREPL!обработчик}

nREPL делится на три важные части: обработчик запроса, middleware и транспорт. Коротко опишем каждую из них.

Обработчик (\code{handler})~--- это функция, которая принимает словарь сообщения. В nREPL сообщения структурированы, то есть разбиты на поля. По полю \code{:op} (operation) функция понимает, что от нее требуется, и выполняет действие. Вот как выглядит команда вычислить код:

\begin{english}
  \begin{clojure}
{:op "eval" :code "(+ 1 2 3)"}
  \end{clojure}
\end{english}

\noindent
и ответ на нее:

\begin{english}
  \begin{clojure}
{:id "..." :session "..." :value 6}
  \end{clojure}
\end{english}

Мы привели сообщения в виде EDN для читаемости, однако в транспорте они выглядят иначе. Многоточия означают длинные идентификаторы, которые мы сократили за ненадобностью.

Логично ожидать, что обработчик возвращает словарь ответа подобно библиотеке \footurl{Ring}{\urlring}[Ring]. Это приводит к ограничению <<один запрос~--- один ответ>>, что нарушает сказанное выше: на одно входящее сообщение может быть несколько исходящих.

Сообщение, переданное обработчику, содержит поле \code{:trans\-port} с текущим объектом транспорта. Чтобы отправить ответ, обработчик вызывает метод \code{send} транспорта со словарем ответа:

\begin{english}
  \begin{clojure}
(defn handler [{:as message
                :keys [transport op code]}]
  (let [value (eval ...)]
    (t/send transport {:value value})))
  \end{clojure}
\end{english}

Ответов может быть несколько, когда вычисляют более одной формы за раз. Выделим в редакторе строки ниже и выполним \code{M-x cider-eval-region}:

\begin{english}
  \begin{clojure}
(+ 1 2 3)
(+ 1 2 3 4)
  \end{clojure}
\end{english}

На сервер уйдёт сообщение:

\begin{english}
  \begin{clojure}
{:id "..." :op "eval" :code "(+ 1 2 3)(+ 1 2 3 4)"}
  \end{clojure}
\end{english}

Получим три ответа: по одному на каждую форму плюс завершающий, который означает конец вычислений. У всех трёх ответов одинаковый ID, чтобы понять, к какому запросу они относятся.

\begin{english}
  \begin{clojure}
{:id "..." :session "..." :value "6"}
{:id "..." :session "..." :value "10"}
{:id "..." :session "..." :status ["done"]}
  \end{clojure}
\end{english}

Если бы обработчик был чистой функцией, он должен был бы вернуть список ответов. Но список нельзя отправить клиенту, пока он не вычислен полностью. В этом случае мы вынудим клиента ждать до тех пор, пока не будут готовы все ответы. Это неоптимально: гораздо лучше отправлять сообщения по мере готовности.

\subsection{Транспорт}

\index{nREPL!транспорт}

В терминах nREPL транспорт~--- это соглашение о том, какой канал связи использовать и как кодировать и декодировать сообщения. Чтобы клиент и сервер понимали друг друга, они должны использовать одинаковый транспорт. На уровне кода это объект, реализующий протокол \code{nrepl.transport.Transport}. В него входят методы \code{recv} и \code{send}, которые отвечают за прием и отправку сообщений.

nREPL предлагает три транспорта: Bencode, EDN и TTY. Мы перечислили их по убыванию важности. Большинство клиентов используют Bencode, поэтому он задан по умолчанию. Bencode опирается на одноименный формат данных, который мы рассмотрим чуть позже.

Транспорт EDN передает данные в формате, принятом в Clojure. Его используют в ClojureScript, поскольку там возможностей Bencode не хватает. Транспорт TTY предназначен для подключения через telnet. Это наиболее скудный транспорт, которым пользуются в крайних случаях.

\subsection{Middleware}

По аналогии с Ring, middleware~--- это прослойка между запросом и обработчиком. Задача middleware в том, чтобы расширить логику сервера, не меняя обработчик.

\index{nREPL!middleware}

\def\urlclasspath{https://docs.oracle.com/javase/tutorial/essential/environment/paths.html}

Предположим, мы бы хотели, чтобы команда <<class\-path>> вер\-ну\-ла список \footurl{путей JVM}{\urlclasspath}[Class path]. Для этого напишем middleware с логикой: если поле \code{:op} сообщения равно <<classpath>>, отправить ответ со списком строк. В противном случае вызвать обработчик по умолчанию:

\begin{english}
  \begin{clojure/lines}
(defn wrap-classpath
  [handler]
  (fn [{:as msg :keys [op transport]}]
    (if (= "classpath" op)
      (let [paths
            (get-classpath ...)]
        (t/send transport {... :classpath paths}))
      (handler msg))))
  \end{clojure/lines}
\end{english}

Как и в Ring, цепочка middleware образует стек. В примере выше переменная \code{handler}~--- не обязательно конечный обработчик nREPL \coderef{1}. Скорее всего, он многократно обёрнут другими middleware выше по стеку.

Проект Cider, который мы скоро рассмотрим, предлагает множество подобных middleware. Вместе они радикально расширяют возможности nREPL.

\section{Подключение из Clojure}

\index{библиотеки!nrepl}

Опробуем nREPL на практике: подключимся к серверу и выполним несколько выражений. Пока что мы не знаем, как подключиться из редактора, поэтому воспользуемся клиентом на Clojure. У нас будет два сеанса nREPL: первый в роли сервера, второй в качестве клиента.

Запустите оба сеанса командой \code{lein repl} или \code{clj -M:nrepl}. Библиотека \code{nrepl/nrepl} должна быть указана в профиле по умолчанию (файлы \code{\tilde{}/.lein/profiles.clj} и \code{\tilde{}/.clojure/deps.edn}). Запомните порт первого сеанса (в случае автора это 50411). Его можно увидеть в терминале или в файле \code{.nrepl-port} той директории, где запущен сеанс.

Во втором сеансе выполните:

\begin{english}
  \begin{clojure}
(require '[nrepl.core :as nrepl])

(def conn
  (nrepl/connect :port 50411))

(def client
  (nrepl/client conn 1000))
  \end{clojure}
\end{english}

Функция \code{connect} открывает соединение с сервером nREPL. Из соединения получают клиента (их может быть несколько в рамках одного соединения). Клиент отвечает за отправку и получение сообщений.

Для начала сложим несколько чисел. Отправьте сообщение с операцией \code{eval}:

\begin{english}
  \begin{clojure}
(nrepl/message client
               {:op "eval" :code "(+ 1 2 3)"})
  \end{clojure}
\end{english}

Получим два ответа: первый с результатом, второй с признаком окончания:

\begin{english}
  \begin{clojure}
({:id "..."
  :ns "my-repl"
  :session "..."
  :value "6"}
 {:id "..."
  :session "..."
  :status ["done"]})
  \end{clojure}
\end{english}

Проверим, что случится, если возникнет исключение. Поделим число на ноль:

\begin{english}
  \begin{clojure}
=> (nrepl/message client
                  {:op "eval" :code "(/ 1 0)"})
  \end{clojure}
\end{english}

Ответ:

\begin{english}
  \begin{clojure}
({:err "Execution error (ArithmeticException) at ..."
  :id "..."
  :session "..."}
 {:ex "class java.lang.ArithmeticException"
  :id "..."
  :root-ex "class java.lang.ArithmeticException"
  :session "..."
  :status ["eval-error"]}
 {:id "..."
  :session "..."
  :status ["done"]})
  \end{clojure}
\end{english}

Получили краткие сведения об исключении: класс, текст и последний элемент стектрейса. Сбором этих данных занимается функция, которую можно задать параметром \code{:nrepl.middle\-wa\-re.ca\-ught/caught}. Функция должна быть объявлена на сервере; в сообщении передают путь к ней:

\begin{english}
  \begin{clojure}
{:nrepl.middleware.caught 'project.util/caught-func}
  \end{clojure}
\end{english}

Убедимся, что изменения, переданные клиентом, вступают в силу. Объявите функцию \code{add}:

\begin{english}
  \begin{clojure}
=> (nrepl/message client {:op "eval" :code "
(defn add [a b]
  (+ a b))
"})
  \end{clojure}
\end{english}

Перейдите во вкладку терминала, где работает сервер. Введите \code{(add 1 2)}~--- функция сработает без ошибок.

\index{nREPL!lookup}

Другая полезная команда называется \code{lookup}. Она принимает символ и возвращает данные о переменной, связанной с ним. Данные содержат путь к исходному файлу, позицию в нем, документацию и параметры вызова, если это функция или макрос. На \code{lookup} опираются редакторы, чтобы переходить к определению и показывать документацию. Запросим информацию о символе \code{+}:

\begin{english}
  \begin{clojure}
=> (nrepl/message client {:op "lookup" :sym "+"})
  \end{clojure}
\end{english}

Символ может быть как с пространством, так и без него. Во втором случае поиск происходит в текущем пространстве имен. В ответ получим всё необходимое для перехода к определению (файл и строка), вызова (список аргументов) и документации (поле \code{:doc}):

\index{макросы!doc}

\begin{english}
  \begin{clojure}
({:id "..."
  :info {:protocol ""
         :added "1.2"
         :ns "clojure.core"
         :name "+"
         :file "jar:file:/Users/ivan/.m2/.../core.clj"
         :arglists-str "([] [x] [x y] [x y & more])"
         :column 1
         :line 984
         :arglists "([] [x] [x y] [x y & more])"
         :doc "Returns the sum of nums..."}
  :session "..."
  :status ["done"]})
  \end{clojure}
\end{english}

Команда \code{completions} возвращает список определений по префиксу. Это полезно, когда пользователь ввёл часть текста и ожидает выпадающее окно с вариантами. Например, функции семейства \code{ex-...} служат для работы с исключениями. Проверим, что найдет сервер по префиксу \code{ex-}:

\begin{english}
  \begin{clojure}
(nrepl/message client
               {:op "completions" :prefix "ex-"})
  \end{clojure}
\end{english}

Получили четыре варианта для дополнения:

\begin{english}
  \begin{clojure}
({:completions
  [{:candidate "ex-cause" :type "function"}
   {:candidate "ex-data" :type "function"}
   {:candidate "ex-info" :type "function"}
   {:candidate "ex-message" :type "function"}]
  :id "..."
  :session "..."
  :status ["done"]})
  \end{clojure}
\end{english}

nREPL поддерживает другие полезные команды, например \code{:load-file} для загрузки модуля. Сообщение передает его код и метаданные. Создайте файл \code{src/sample.clj} с кодом:

\begin{english}
  \begin{clojure}
(ns sample)

(defn multiply [a b]
  (* a b))
  \end{clojure}
\end{english}

Затем отправьте его на сервер (функция \code{slurp} читает файл в строку):

\begin{english}
  \begin{clojure}
(nrepl/message client {:op "load-file"
                       :file (slurp "src/sample.clj")})
  \end{clojure}
\end{english}

Сервер скомпилирует код из файла, и в результате появится пространство \code{sample} с функцией \code{add}. Перейдите в терминал с сервером и опробуйте её:

\begin{english}
  \begin{clojure}
=> (sample/multiply 3 4)
12
  \end{clojure}
\end{english}

\def\urlnreplops{https://nrepl.org/nrepl/ops.html}

Мы не будем перечислять все команды nREPL. Любопытный читатель найдет их в \footurl{документации}{\urlnreplops}[nREPL options] проекта. Завершим раздел командой \code{close}. Она принимает строго один параметр~--- номер сессии~--- и освобождает ресурсы, связанные с ней:

\begin{english}
  \begin{clojure}
(nrepl/message
  client
  {:op "close"
   :session "97ec6c4b-28ee-4402-87e3-43f3275a7430"})
  \end{clojure}
\end{english}

\subsection{Коротко о Bencode}

\index{nREPL!Bencode}
\index{Bencode}

Возможно, читателю интересно, что именно передают друг другу клиент и сервер nREPL. Пока открыты оба сеанса, запустите команду \code{tcpdump} для записи трафика в файл. В системах Linux и MacOS утилита доступна по умолчанию. \code{Tcpdump} требует прав суперпользователя, поэтому запускается с \code{sudo}:

\begin{english}
  \begin{bash}
> sudo tcpdump port 50411 -i lo0 -w nrepl.log
  \end{bash}
\end{english}

\def\urlwshark{https://www.wireshark.org}

Число 50411~--- это порт сервера nREPL, а \code{nrepl.log}~--- выходной файл с TCP-пакетами. После запуска \code{tcpdump} перейдите в REPL и выполните несколько действий. Завершите \code{tcpdump} нажатием \code{Ctrl+C} и откройте файл в программе \footurl{Wireshark}{\urlwshark}[Wireshark]. Это графическое приложение для анализа сетевого трафика. Содержимое файла выглядит в нем примерно так (переносы строк отделяют запрос и ответ):

\begin{english}
  \begin{text}
d4:code9:(+ 1 2 3)2:id36:...:op4:evale
d2:id36:...:ns7:my-repl7:session36:...:value1:6e
  \end{text}
\end{english}

\def\urlbencode{https://en.wikipedia.org/wiki/Bencode}

Каждая строка~--- структура данных в формате \footurl{Bencode}{\urlbencode}[Bencode] (многоточия заменяют идентификаторы). Формат создан в рамках проекта BitTorrent. Позже его переняли другие системы, в том числе nREPL. Bencode передает числа, строки, словари и списки. Несмотря на меньший по сравнению с JSON набор типов, он обладает решительным преимуществом~--- простотой.

Описание формата занимает меньше страницы. Числа записываются как \code{i<число>e}, например \code{i2020e} означает \code{2020}. Цепочка байтов~--- в виде \code{<длина>:<содержимое>}; строка \code{"hello"} становится \code{"5:hello"}. Выражение \code{l<...>e} означает список. Значения между \code{l} и \code{e} станут его содержимым. В строке \code{"l5:helloi42ee"} закодирован список с элементами \code{"hello"} и \code{42}. Форма \code{d<...>e} служит для словаря. От списка он отличается тем, что перед каждым значением идет строка с именем ключа. Сообщение

\begin{english}
  \begin{text}
"d5:title4:19844:yeari1948ee"
  \end{text}
\end{english}

\noindent
означает словарь \code{{:title "1984" :year 1948}}.

Формат допускает вложенность одних коллекций в другие, например список словарей или словарь, значения которого~--- списки. Упакуем в Bencode следующие данные:

\begin{english}
  \begin{clojure}
{:title "1984"
 :year 1948
 :tags ["novel" "fiction" "dystopia"]
 :author {:fname "George" :lname "Orwell"}}
  \end{clojure}
\end{english}

Результат с переносами строк для читаемости:

\begin{english}
  \begin{text}
d6:authord5:fname6:George5:lname6:Orwell \
e4:tagsl5:novel7:fiction8:dystopiae \
5:title4:19844:yeari1948ee
  \end{text}
\end{english}

Всё, вы знаете Bencode!

Этих правил достаточно для передачи сообщений в nREPL. Конечно, из-за скудных типов теряется семантика: объект \code{Keyword} становится строкой, но обратной операции не предусмотрено~--- получив строку, нельзя определить, была ли она раньше кейвордом. Вектор, список и множество становятся списком, и узнать его исходный тип невозможно. Однако чаще всего этого и не требуется.

\index{JSON}
\index{ClojureScript}

Почему бы не использовать более продвинутый формат, например JSON? Причина в том, что, в отличие от него, Bencode экстремально прост: код записи и чтения данных занимает около ста строк. Это важно для встроенных в редактор языков, которые не всегда поддерживают JSON. Разработка на них JSON-парсера требует усилилий.

Транспорт Bencode назначен в nREPL по умолчанию. Когда nREPL работает с ClojureScript, используется EDN. Сообщения в ClojureScript требуют больше типов данных, и возможностей Bencode не хватает.

Ради эксперимента укажите серверу и клиенту транспорт EDN. Для сервера это делается ключом \code{:repl-options}:

\begin{english}
  \begin{clojure}
:repl-options {...
               :transport nrepl.transport/edn}
  \end{clojure}
\end{english}

Если выполнить \code{lein repl}, вы увидите в консоли надпись \code{nrepl+edn://127.0.0.1:<port>}. Подключитесь к серверу с новыми параметрами клиента:

\begin{english}
  \begin{clojure}
(require '[nrepl.transport :as transport])

(def conn (nrepl/connect
           :port 61093
           :transport-fn transport/edn))
  \end{clojure}
\end{english}

\index{утилиты!Wireshark}
\index{Wireshark}

Запишите трафик в файл и исследуйте в Wireshark. Вы увидите привычные данные в формате Clojure:

\begin{english}
  \begin{clojure}
{:op "eval"
 :code "(+ 1 2 3)"
 :id "..."}

{:id "..."
 :session "..."
 :ns "my-repl"
 :value "6"}
  \end{clojure}
\end{english}

\def\urlnreplbencode{https://github.com/nrepl/bencode}

Bencode для Clojure доступен в двух вариантах: как часть nREPL и в виде отдельной библиотеки \footurl{nrepl/bencode}{\urlnreplbencode}[nREPL Bencode]. Последняя является копией модуля nREPL и обновляется параллельно с ним.

\section{Клиенты nREPL для редакторов}

Мы провели достаточно опытов, чтобы понять, как устроен клиент nREPL. Это код, который отправляет сообщения по особому протоколу. Клиенты пишут на разных языках, в том числе встроенных в редактор. Получаются плагины~--- программные модули, которые служат прослойкой между пользователем и nREPL. По нажатии клавиш плагин посылает сообщение и выводит результат рядом с кодом.

Клиенты nREPL пишут на Clojure, Emacs Lisp, Java, Python, Lua, JavaScript, TypeScript, VimScript и других языках. Многообразие объясняется тем, что каждый редактор использует свой язык для внутренних нужд. Например, плагины Emacs пишут на диалекте Elisp; редактор VS Code поддерживает JavaScript и TypeScript; модули к продуктам JetBrains создают на Java и так далее.

В этом разделе мы не будем рассматривать все клиенты. Наоборот, остановим выбор на модуле Cider для Emacs. В его пользу говорят следующие факты.

\index{Cider}

\def\urlciderinit{https://github.com/clojure-emacs/cider/tree/v0.1.0}

\textbf{Долгая история.} Первый коммит в репозиторий Cider \footurl{сделан в 2012 году}{\urlciderinit}[Cider v0.1.0][-20mm]. На момент написания книги проекту полных десять лет. Cider давно вышел из стадии любительского решения: у него обширное сообщество и документация.

\def\urlsyrvey{https://clojure.org/news/2022/06/02/state-of-clojure-2022}

\textbf{Популярность.} Согласно ежегодному опросу \footurl{Clojure Survey}{\urlsyrvey}[Clojure Survey 2022][5mm], связка Cider/Emacs держит первое место по популярности у~разработчиков. Доля голосов в пользу Cider превышает 40~\%, хоть и плавно снижается из-за развития других проектов.

\textbf{Компетенция в сообществе.} Исторически сложилось, что Emacs в б\'{о}льшей степени подходит для разработки на Лиспе, чем другие редакторы. За долгие годы его адаптировали под разные диалекты~--- Common Lisp, Racket, Scheme и другие. Cider опирается на этот опыт: основные его функции~--- это повтор удачных решений для других Лиспов.

Если вы пользуетесь другим редактором, не спешите пропускать раздел. Возможно, вы откроете подходы, о которых не знали раньше. Также вы заочно познакомитесь с Emacs: это сложный редактор, но он стоит потраченных сил.

\section{Emacs и Cider}

\index{Emacs!Cider}

\def\urlemacscider{https://github.com/clojure-emacs/cider}
\def\urlcidernrepl{https://github.com/clojure-emacs/cider-nrepl}

Проект Cider состоит из двух частей. Первая~--- одноименный \footurl{модуль для Emacs}{\urlemacscider}[Cider], чтобы подключаться к nREPL из редактора. Вторая часть~--- библиотека на Clojure под названием \footurl{cider-nrepl}{\urlcidernrepl}[Cider nREPL]. Это набор middleware, которые расширяют nREPL: добавляют запуск тестов, отладку, переходы по коду, профилирование и многое другое.

Взаимодействие Emacs и сервера выглядит так:

\begin{figure}[H]
  \centering
  \includesvg{charts/repl01.svg}
  \label{fig:chart-repl-01}
\end{figure}

Cider-nrepl работает на сервере и не зависит от языка, на котором написан клиент. На него опирается не только Emacs, но и плагины для Vim и других редакторов.

Есть несколько способов начать работу над проектом в Emacs. \textbf{Первый}~--- поручить все шаги Cider. Откройте любой .clj-файл и выполните:

\begin{english}
  \begin{text}
M-x cider-jack-in
  \end{text}
\end{english}

Emacs начнет искать файл \code{project.clj} в текущей папке, а затем всё выше и выше. Если он найден, Emacs запустит процесс \code{lein repl}. В параметрах окажутся библиотека \code{nrepl} и служебные плагины. Приведем итоговую команду в сокращении:

\begin{english}
  \begin{bash*}{fontsize=\small}
> /usr/local/bin/lein \
  update-in :dependencies conj [nrepl/nrepl "0.9.0"] \
  update-in :plugins conj [cider/cider-nrepl "0.28.3"] \
  update-in :plugins conj [mx.cider/enrich-classpath "1.9.0"] \
  update-in :middleware conj cider.enrich-classpath/middleware \
  repl :headless :host localhost
  \end{bash*}
\end{english}

Cider различает системы управления проектом: lein, Clojure CLI и Boot. Для каждой из них он выполнит разные команды. Если найдены файлы нескольких утилит, Emacs спросит, что именно запустить.

После запуска nREPL редактор подключится к нему. Откроется буфер \code{*cider-repl <project>*} для ввода выражений. Ещё один буфер \code{*nrepl-server <project>*} служит для вывода процесса \code{lein repl}.

\index{cider-jack-in}

Возможно, первый запуск \code{cider-jack-in} займёт время. Его бóльшая часть уйдёт на загрузку зависимостей.

При \textbf{втором способе} подключения шаги проделывают вручную: запускают nREPL в терминале и подключаются из редактора. Откройте файл \code{\tilde{}/.lein/profiles.clj} с локальным профилем. В вектор \code{:user} \arr \code{:plugins} добавьте плагин \code{cider/cider-nrepl}. Плагин зависит от модуля \code{nrepl/nrepl}, поэтому последний указывать не нужно~--- он загрузится как транзитивная зависимость.

\begin{english}
  \begin{clojure}
{:user
 {:plugins
  [[cider/cider-nrepl "0.28.3"]]}}
  \end{clojure}
\end{english}

\index{Emacs!мини-буфер}
\index{cider-connect}

Запустите в терминале процесс \code{lein repl}. Перейдите в Emacs и выполните \code{M-x cider-connect}. Редактор запросит у вас хост и порт сервера. В нашем случае хост будет \code{localhost} или \code{127.0.0.1}. Вводить порт вручную необязательно: Cider найдет его в файле \code{.nrep-port}. Для этого нажмите в мини-буфере TAB~--- появится список вариантов. Emacs покажет не только номера портов, но и названия проектов, с которыми они связаны.

\begin{english}
  \begin{text}
Possible completions are:
- etaoin:54446
- pact:64187
  \end{text}
\end{english}

После соединения вы окажетесь в буфере \code{*cider-repl*} с приглашением. Буфера \code{*nrepl-server*} не будет, поскольку сервер запущен вне Emacs и между ними нет связи по каналам stdin и stdout.

Для Clojure CLI проект выглядит как в примере ниже. Запустите его командой \code{clojure -M:cider}. По аналогии с \code{lein}, поместите профиль \code{:cider} в файл \code{\tilde{}~/.clojure/deps.edn}, чтобы пользоваться им в любом проекте.

\begin{english}
  \begin{clojure}
{:aliases
 {:cider
  {:extra-deps
   {cider/cider-nrepl {:mvn/version "0.25.9"}}
   :main-opts
   ["-m" "nrepl.cmdline"
    "--bind" "localhost"
    "--middleware" "[cider.nrepl/cider-middleware]"]}}}
  \end{clojure}
\end{english}

Возможно, у читателя возникнет вопрос: зачем нужно ручное подключение, если доступно автоматическое? Пока мы не ушли дальше, объясним разницу между подходами.

В автоматическом режиме (\code{cider-jack-in}) процесс \code{lein repl} запускается силами Emacs. Если редактор <<упадёт>>, завершатся открытые им процессы. Те, кто работает с Clojure на регулярной основе, держат несколько запущенных проектов одновременно. Восстанавливать их после перезапуска редактора утомительно.

Когда сеансы запущены вне Emacs, сбой в редакторе не скажется на них. Более того~--- они сохранят изменения, которые вы внесли до этого. Достаточно включить редактор и подключиться к запущенным проектам.

В редких случаях к проекту можно подключиться только в удаленном режиме. Например, если nREPL запущен на другой машине или в виртуальном окружении (Docker, VirtualBox). Эти сценарии мы рассмотрим ближе к концу главы \page{section-repl-docker}.

\subsection{Первые шаги}

Итак, если подключение состоялось, откроется буфер \code{*cider-repl*} приглашением. Введите \code{(+ 1 2)}, чтобы убедиться в его работе. В верхней части буфера находится краткая справка. Если вы только знакомитесь с Cider, прочитайте ее. Опытные разработчики отключают справку, назначив \code{nil} переменной Emacs:

\begin{english}
  \begin{lisp}
(setq cider-repl-display-help-banner nil)
  \end{lisp}
\end{english}

\index{Emacs!*cider-repl*}

Дальнейшие шаги зависят от конфигурации проекта. Если не задано пространство по умолчанию (параметры \code{:main} или \code{:repl-options} \arr \code{:init-ns}), ничего не будет загружено, и вы окажетесь в пространстве \code{user}. Загрузить код в nREPL можно двумя способами: вручную и автоматически.

В первом случае откройте файл с главным модулем. Как правило, это пространства \code{<project>.core} или \code{<project>.main}. Выполните команду \code{M-x cider-load-buffer}. Ей пользуются часто, поэтому команде назначено сочетание клавиш \code{C-c C-k}. Пространства имен, включая его зависимости, будут загружены на сервере.

При ручной загрузке может случиться так, что какие-то модули пропущены. Особенно досадно, когда не загружен модуль, расширяющий протокол или мультиметод. Код скомпилируется, но при запуске получим исключение, что нет нужной реализации.

Чтобы этого избежать, Cider предлагает автоматическую загрузку командой \code{M-x cider-ns-refresh}. Она перебирает пути classpath и принудительно загружает все файлы Clojure. Как только вы подключились к nREPL, выполните эту команду, и проект готов к работе.

\subsection{Выполнение кода}

После загрузки кода его можно выполнить. Перейдите в буфер \code{*cider-repl*} и введите вызов любой функции:

\begin{english}
  \begin{clojure}
(my.project.util/some-func {:message "hello"})
  \end{clojure}
\end{english}

Нажмите \enter. На сервер уйдёт сообщение с этим кодом. Там он выполнится, и ниже появится результат.

Чтобы быстро перейти в буфер REPL, связанный с проектом, введите команду \code{M-x cider-switch-to-repl-buffer} (сочетание клавиш \code{C-c C-z}).

Печатать длинный код в буфере \code{*cider-repl*} неудобно. Гораздо лучше выполнить его из файла. Команда \code{M-x cider-eval-last-sexp}, назначенная на \code{C-x C-e}, выполнит последнее перед курсором S-выражение. Предположим, вы написали следующий код:

\begin{english}
  \begin{clojure}
(let [fname "John"
      lname "Smith"]
  (format "%s %s" fname lname))
  \end{clojure}
\end{english}

Поставьте курсор за последнюю скобку и выполните \code{C-x C-e}. Справа от формы появится результат:

\begin{english}
  \begin{clojure}
(let [fname "John"
      lname "Smith"]
  (format "%s %s" fname lname)) => "John Smith"
  \end{clojure}
\end{english}

\index{S-выражение}

По аналогии с \code{inferior-mode}, S-выражение может быть где угодно: не только на верхнем уровне модуля, но и внутри другой формы. В выражении

\begin{english}
  \begin{clojure}
(let [id (java.util.UUID/randomUUID) |
      name "John"]
  {:id id
   :name name})
  \end{clojure}
\end{english}

\noindent
поставьте курсор на место вертикальной черты и выполните форму \code{(java.util...)}. Вы получите уникальный идентификатор, экземпляр класса \code{UUID}.

\index{классы!UUID}
\index{UUID}

Команды \code{cider-eval-region}, \code{cider-eval-buffer} и другие выполняют код из разных областей. Как следует из названий, \code{-re\-gi\-on} выполняет выделенный код, в котором может быть несколько форм. Команда \code{-buffer} охватывает буфер целиком.

Команда \code{cider-eval-defun-at-point} выполняет определения: \code{def}, \code{defn}, \code{defmacro} и другие. Особенность в том, что курсор может быть в любом месте формы, а не обязательно в конце. В примере ниже установите курсор на место черты и выполните команду. Результатом станет переменная \code{\#'user-description} (объект \code{Var}).

\begin{english}
  \begin{clojure}
(def user-description
  (let [name "John" |
        email "test@test.com"]
    (format "%s <%s>" name email)))

=> #'user-description
  \end{clojure}
\end{english}

Этот прием крайне полезен в работе. Бóльшую часть времени мы проводим, редактируя функции, и перемещать курсор в конец формы неудобно. С помощью \code{cider-eval-defun-at-point} функцию обновляют, находясь в любом месте её кода.

Если результат вычислений велик (например, выборка базы данных), Cider покажет усечённую версию. Чтобы исследовать данные, выполните \code{cider-inspect-last-result}. Откроется буфер \code{*cider-inspect*}, где данные напечатаны постранично с учётом вложенности. Для примера исследуем большой словарь, который получим функцией \code{ns-map} для пространства \code{clojure.core}:

\begin{english}
  \begin{text}
(ns-map 'clojure.core)
  \end{text}
\end{english}

Вот как выглядит буфер инспекции:

\begin{english}
  \begin{text}
Class: clojure.lang.PersistentHashMap
Contents:
  sort-by = #'clojure.core/sort-by
  contains? = #'clojure.core/contains?
  every? = #'clojure.core/every?
  proxy-mappings = #'clojure.core/proxy-mappings
  keep-indexed = #'clojure.core/keep-indexed
  ...
  Page size: 32, showing page: 1 of 29
  \end{text}
\end{english}

\index{инспекция}

За навигацию отвечают особые клавиши; приведём некоторые из них:

\begin{itemize}

\item
  \code{SPC} (пробел)~--- перейти на следующую страницу результата;

\item
  \code{M-SPC}~--- вернуться на предыдущую;

\item
  \code{RET} (Enter)~--- открыть вложенную структуру данных;

\item
  \code{l}~--- подняться на уровень ниже.

\end{itemize}

Команда \code{cider-inspect-last-sexp} (или \code{C-x TAB}) совмещает два шага: выполнить форму и открыть результат в инспекторе. С~ней не понадобится вызывать их по отдельности.

\def\urlinspector{https://docs.cider.mx/cider/debugging/inspector.html}

Полное описание инспектора, его команд и клавиш вы найдёте на сайте Cider в \footurl{одноимённом разделе}{\urlinspector}[Inspector][-10mm].

\subsection{Dev-секции}

\index{Dev-секции}

Опытные программисты оставляют в файлах так называемые dev sections~--- области разработки. Это код, который выполняют в REPL, чтобы проверить некоторые вычисления. Dev-секция помещается в макрос \code{comment}, чтобы не участвовать в компиляции.

Предположим, вы написали функцию \code{->fahr} для перевода температуры между шкалами Цельсия и Фаренгейта:

\begin{english}
  \begin{clojure}
(defn ->fahr [cel]
  (+ (* cel 1.8) 32))
  \end{clojure}
\end{english}

Чтобы её проверить, добавьте в конец файла отладочный код:

\begin{english}
  \begin{clojure/lines}
(comment
  (->fahr 36.6)
  (->fahr 0)
  (->fahr nil)
  )
  \end{clojure/lines}
\end{english}

Поставьте курсор после закрывающей скобки в первой форме и выполните её~--- получится 97.88, что совпадает с ожиданиями \coderef{2}. Выполните и другие выражения, в том числе с \code{nil}, чтобы спровоцировать исключение. Если функция изменилась, повторите шаги и оцените результат.

Обратите внимание на следующие моменты. Отладочный код находится в макросе \code{comment}, который игнорирует содержимое: при компиляции он вырождается в пустоту \coderef{1}. Не путайте макрос \code{comment} и комментирование точкой с запятой. В примере ниже форму нельзя выполнить: команда \code{cider-eval-last-sexp} не сработает, потому что строка игнорируется целиком.

\begin{english}
  \begin{clojure}
;; cannot be evaluated
;; (->fahr 36.6) |
  \end{clojure}
\end{english}

\index{макросы!comment}
\index{comment}

Закрывающая скобка \code{comment} стоит отдельно, чтобы по ошибке не выполнить комментарий вместо последней формы \coderef{5}. Покажем это на примере. Предположим, скобка стоит по правилам Lisp-синтаксиса на той же строке:

\begin{english}
  \begin{clojure}
(comment
  ...
  (->fahr nil))
  \end{clojure}
\end{english}

Чтобы выполнить \code{(->fahr nil)}, курсор ставят между двумя последними скобками:

\begin{english}
  \begin{clojure}
(comment
  ...
  (->fahr nil)|)
  \end{clojure}
\end{english}

На практике легко промахнуться и поставить курсор в конец:

\begin{english}
  \begin{clojure}
(comment
  ...
  (->fahr nil)) |
  \end{clojure}
\end{english}

В этом случае команда \code{cider-eval} выполнит форму \code{comment}, которая вернёт \code{nil} вне зависимости от содержимого. Сложится ощущение, что \code{(->fahr nil)} возвращает \code{nil}, что на самом деле не так. Чтобы этого не случилось, скобку \code{comment} переносят на новую строку.

Некоторые редакторы выделяют форму \code{comment} цветом, чтобы подчеркнуть: это не боевой код, а пример или справка. Если подсветка не работает, попробуйте список и тег игнорирования \code{\#\_}:

\begin{english}
  \begin{clojure}
#_
((->fahr 36.6)
  (->fahr 0)
  (->fahr nil)
)
  \end{clojure}
\end{english}

В результате dev-секция будет окрашена в особый цвет, и вы отличите её от кода.

Код dev-секции может быть с побочным эффектом, например когда вы тестируете HTTP-запросы или базу данных. Проследите, чтобы в коде не было паролей или ключей доступа. Если они необходимы, считайте их из файла или переменной среды:

\begin{english}
  \begin{clojure}
(comment
  (def -api-key
    (slurp "API_KEY"))
  (def -response
    (make-http-request ... -api-key)))
  \end{clojure}
\end{english}

\def\urlzipclj{https://github.com/clojure/clojure/blob/master/src/clj/clojure/zip.clj\#L281}

Dev-секции встречаются во многих библиотеках, в том числе Clojure. Например, модуль \code{clojure.zip} содержит блок \footurl{\code{comment}}{\urlzipclj}[zip.clj] с набором шагов, чтобы проверить зипперы.

\subsection{Сниппеты}

\index{сниппеты}

Чем дольше вы работаете с проектом, тем больше у вас будет сниппетов. Так называют фрагменты кода, которые делают что-то полезное. У сниппетов особое положение: с одной стороны, им не место в промышленном коде. С другой стороны, они нужны в разработке, поэтому их нужно где-то хранить.

Снипеттом может быть запрос к базе, например очистка таблиц или сложная выборка. В разработке сброс таблиц нужен часто, поэтому логично держать сниппет под рукой.

\begin{english}
  \begin{clojure}
(clojure.java.jdbc/execute!
 {:dbtype "postgresql"
  :dbname "test"
  :host "127.0.0.1"
  ...}
 ["truncate users, orders, ... cascade"])
  \end{clojure}
\end{english}

\index{библиотеки!Clj-http}
\index{Clj-http}

Другой пример~--- обращение к внешнему сервису по HTTP API. Это вызов функции \code{post} из библиотеки Clj-http с параметрами и заголовками:

\begin{english}
  \begin{clojure}
(clj-http.client/post
 "https://internal.api.com/api/v1"
 {:as :json
  :content-type :json
  :headers {"Authorization" "Bearer ..."}
  :form-params {:event "user_created"
                :user_id 10099}})
  \end{clojure}
\end{english}

Обратите внимание, что в сниппетах используют полные пространства имен. Это нужно для того, чтобы код сработал в любом пространстве, в том числе там, где нет импортов \code{clj-http.client} или \code{cheshire.core}. Сниппеты удобно хранить в отдельном .clj-файле, чтобы выполнять оттуда, не копируя в REPL. Так получается персональная среда разработки.

Cider предлагает особый буфер, чтобы выполнить код на Clojure. Когда вы подключены к nREPL, наберите команду \code{M-x cider-scratch}. Откроется буфер \code{*cider-scratch*}, связанный с текущим проектом. Скопируйте в него любой код. Поместите курсор за нужной формой и нажмите \code{C-j}~--- на следующей строке появится результат:

\index{Emacs!*cider-scratch*}

\begin{english}
  \begin{clojure}
(+ 1 2) | ;; press C-j

(+ 1 2)
3
  \end{clojure}
\end{english}

Особенность \code{*cider-scratch*} в том, что результат не пропадает, а остается в файле для дальнейшей работы. Выполните более сложный пример, нажимая \code{C-j} после каждой формы:

\begin{english}
  \begin{clojure}
(require '[clojure.walk :as walk])
nil

(walk/stringify-keys {:hello {:test 33}})
{"hello" {"test" 33}}
  \end{clojure}
\end{english}

\index{модули!clojure.pprint}
\index{печать с отступами}

Комбинация \code{C-u C-j} печатает результат при помощи \code{clojure.pprint}, то есть с отступами и переносами строк. Опробуйте ее на больших данных, например словаре переменных среды:

\begin{english}
  \begin{clojure}
(into {} (System/getenv)) | ;; C-j
  \end{clojure}
\end{english}

Пользователи Emacs догадались, что буфер \code{*cider-scratch*} повторяет поведение \code{*scratch*}. Так называется буфер Emacs, который выполняет код на ELisp. Разница в том, что \code{*cider-scratch*} ожидает код на Clojure и связан с проектом, в папке которого находится.

Если сохранить \code{*cider-scratch*}, Emacs запросит путь на диске, потому что по умолчанию буфер не связан с файлом. Введите любое имя, например \code{scratch.clj}. Если открыть его в следующий раз, он поведёт себя как обычный текст: нажатие \code{C-j} перенесёт каретку без выполнения кода. Так происходит потому, что при открытии буфер получит базовый режим (fundamental mode). Смените его командой \code{M-x cider-clojure-interaction-mode}, и выполнение кода заработает.

Чтобы не вводить команду каждый раз, воспользуйтесь одним из двух способов. Первый~--- поместите в начале файла строку

\begin{english}
  \begin{text}
; -*- mode: cider-clojure-interaction -*-
  \end{text}
\end{english}

\def\urlemacsfilevars{https://www.gnu.org/software/emacs/manual/html\_node/efaq/Associating-modes-with-files.html}

При открытии файлов Emacs \footurl{учитывает выражения}{\urlemacsfilevars}[Emacs file variables], заключённые в символы \code{-*-}, и выполняет их. Метка \code{mode} означает сменить режим буфера на указанный за двоеточием.

\index{Emacs!переменные}

Второй, более универсальный способ~--- связать регулярное выражение файла с режимом. Выражение охватывает только имя файла и расширение (путь отбрасывается). В нашем случае правило выглядит так:

\begin{english}
  \begin{lisp}
(add-to-list 'auto-mode-alist
  '("scratch\.clj" . cider-clojure-interaction-mode))
  \end{lisp}
\end{english}

Выполните этот код в буфере \code{*scratch*}, чтобы изменения вступили в силу (и добавьте его в конфигурацию Emacs). Закройте и откройте файл \code{scratch.clj}~--- он перейдет в интерактивный режим.

Чтобы сниппеты не попали в историю git, добавьте их в \code{.gitignore} проекта или глобальные настройки git (файл \code{\tilde{}/.gitignore} в домашней директории).

Пользу сниппетов трудно переоценить: гораздо легче найти код по ключевым словам, чем набирать по памяти. Сохраняйте полезные выражения в файл~--- в будущем они пригодятся вам и коллегам.

\subsection{Пространства имен}

Команды REPL всегда выполняются в каком-то пространстве. По умолчанию оно указано в приглашении:

\begin{english}
  \begin{clojure}
user=> (+ 1 2)
  \end{clojure}
\end{english}

Если перейти в другое пространство, изменится и приглашение:

\begin{english}
  \begin{clojure}
user=> (in-ns 'foobar)
foobar=>
  \end{clojure}
\end{english}

Код, что мы вводим в REPL, вычисляется в этом пространстве. Если объявить в модуле \code{user} переменную \code{number} и сослаться на неё в пространстве \code{foobar}, получим ошибку, что символ не найден в текущем контексте:

\begin{english}
  \begin{clojure}
(in-ns 'user)

(def number 1)

(in-ns 'foobar)

(+ 1 number)

;; Syntax error compiling at ...
;; Unable to resolve symbol: number in this context
  \end{clojure}
\end{english}

Другой пример: объявим в модулях \code{user} и \code{foobar} переменные \code{number} со значениями 1 и 2. Теперь одна и та же форма \code{(inc number)} даст разный результат в зависимости от того, какое пространство текущее. Поэтому перед вычислением мы должны убедиться, что находимся в нужном пространстве имен.

\begin{english}
  \begin{clojure}
(in-ns 'user)

(inc number) ;; 2

(in-ns 'foobar)

(inc number) ;; 3
  \end{clojure}
\end{english}

Чтобы уберечь нас от подобных ошибок, nREPL учитывает параметр \code{ns} в сообщениях. Когда мы выполняем код командой \code{cider-eval-...}, в сообщении, помимо полей \code{op} и \code{code}, передаётся \code{ns}. Его значение Cider берёт из формы \code{(ns...)} в начале файла. Вычисляя форму, сервер временно меняет пространство, и результат совпадает с тем, что мы ожидали.

Несмотря на это удобство, иногда пространство меняют вручную. Например, для того, чтобы вызывать приватную функцию, объявленную с помощью \code{(defn- ...)} или \code{(def \^{}:private ...)}. Обратиться к ней извне можно только формой \code{resolve} или оператором \code{\#'}, что неудобно:

\begin{english}
  \begin{clojure}
((resolve 'some-ns/private-func) 1 2)
;; or
(#'some-ns/private-func 1 2)
  \end{clojure}
\end{english}

Проще выполнить код в пространстве \code{some-ns}~--- внутри него доступ к приватным переменным не ограничен.

\begin{english}
  \begin{clojure}
(in-ns 'some-ns)

(private-func 1 2)
  \end{clojure}
\end{english}

\pagebreaklarge

Смена пространства важна при отладке кода. Мы подробно рассмотрим отладку чуть позже.

Перечислим возможности Cider для контроля управления пространствами. Команда \code{cider-find-ns} покажет загруженные модули. Они включают не только ваш код, но и сторонние библиотеки и модули Clojure. Имена следуют в алфавитном порядке; в диалоге работает автодополнение.

\begin{english}
  \begin{text}
M-x cider-find-ns

In this buffer, type RET to select the completion
near point.

Possible completions are:
- aleph.http
- bogus.core
- buddy.core.bytes
...
  \end{text}
\end{english}

\index{jar}

При выборе элемента откроется исходный код модуля. Cider поддерживает в том числе модули из jar-архивов. В случае с \code{clojure.core} на ноутбуке автора открывается файл:

\begin{english}
  \begin{text*}{breaklines, breakafter=/}
/Users/ivan/.m2/repository/org/clojure/clojure/1.10.1/clojure-1.10.1-sources.jar
  \end{text*}
\end{english}

Буферы из архивов доступны в режиме чтения. Без особых ухищрений нельзя изменить файл в архиве и сохранить его. В особых случаях это необходимо; в секции про отладку мы рассмотрим, как это сделать.

\index{модули!clojure.core.async}

Команда \code{cider-browse-ns} покажет переменные модуля. Приведём фрагмент для модуля \code{clojure.core.async}:

\begin{english}
  \begin{clojure}
clojure.core.async
- <! takes a val from port.
- <!! takes a val from port.
- >! puts a val into port.
- >!! puts a val into port.
  \end{clojure}
\end{english}

\def\urlciderdocs{https://docs.cider.mx/cider/index.html}

Выбор элемента открывает буфер с подробностями: документацией, спекой, ссылкой на файл. В секции <<Also see>> указаны ссылки на другие определения. \code{Cider-browse-ns} заменяет веб-документацию, которую производят из исходного кода.

Cider предлагает многие другие команды для работы с пространствами. Ознакомьтесь с ними на \footurl{странице документации}{\urlciderdocs}[Cider documentation].

\subsection{Переход к определению}

Программный код~--- это не просто текст; он обладает структурой. Классы, функции и другие элементы образуют его скелет. Продвинутый редактор понимает структуру кода и всячески использует ее. Например, по нажатии клавиши открывает файл с определением функции. По нажатии другой клавиши переходит к предыдущему файлу.

Emacs и Cider предлагают разные способы навигации по коду. Ниже мы рассмотрим некоторые из них.

Команда \code{M-x cider-find-var} запрашивает данные о символе под курсором. Предположим, вы написали код:

\begin{english}
  \begin{clojure}
(let [name "John"
      email "test@test.com"]
  (format "%s <%s>" name email))
  \end{clojure}
\end{english}

Поместите курсор на слово \code{format} и выполните команду. Откроется буфер \code{core.clj} из jar-архива на строке 5738, где объявлена функция \code{format}.

Информацию о символе находит сервер. Пространство, в котором мы ищем символ, должно быть предварительно загружено. В зависимости от технических деталей клиент посылает команду \code{info} или \code{lookup}. От сервера приходят данные об имени файла и позиции в нем.

Чтобы вернуться в прежний буфер, выполните \code{M-x cider-pop-back}. По умолчанию у команды нет сочетания клавиш, а нужна она столь же часто, что и \code{cider-find-var}. Опытным путем автор пришел к комбинации \code{C-x .} (точка). Добавьте в настройки Emacs выражение

\begin{english}
  \begin{lisp}
(global-set-key (kbd "C-x .") 'cider-pop-back)
  \end{lisp}
\end{english}

Переход к определению работает не только с функциями, но и с переменными, объектами \code{defmulti}, \code{defprotocol} и другими. В~случае с \code{defmulti} вы перейдете к объявлению мультиметода, но не его методов. Проверьте это на примере \code{print-method} из \code{clojure.core}.

Команда \code{cider-find-var} учитывает пространства имен и их псевдонимы. Предположим, в файле следующий заголовок \code{ns}:

\begin{english}
  \begin{clojure}
(ns some-ns
  (:require
   [clojure.walk :as walk]))
  \end{clojure}
\end{english}

\index{модули!clojure.walk}
\index{walk}

Чтобы открыть пространство \code{clojure.walk}, поместите курсор на \code{walk} и выполните \code{cider-find-var}.

\def\urlemacsxref{https://www.gnu.org/software/emacs/manual/html\_node/emacs/Xref.html}

\index{Xref}
\index{Emacs!Xref}

В последних версиях Cider используется более абстрактная команда \code{xref-find-definitions}. Она принадлежит встроенному в Emacs \footurl{пакету Xref}{\urlemacsxref}[Emacs Xref] для поиска определений и перекрёстных ссылок. Особенность Xref в том, что его легко расширить под нужный язык или платформу. Об этом мы расскажем чуть ниже \page{section-xref}.

Команда \code{cider-javadoc} открывает документацию к классу Java. Предположим, мы работаем с сертификатами, и в заголовке \code{ns} находятся импорты:

\begin{english}
  \begin{clojure}
(ns ...
  ...
  (:import
   java.security.cert.CertificateFactory
   java.security.cert.X509Certificate
   java.security.PublicKey
   ...))
  \end{clojure}
\end{english}

Наведите курсор на любой класс и выполните \code{M-x cider-javadoc}~--- откроется браузер с документацией для вашей версии JVM. В случае автора страница для класса \code{X509Certificate} оказалась следующей:

\begin{english}
  \begin{text*}{breaklines, breakafter=/}
https://docs.oracle.com/en/java/javase/11/docs/api/java.base/java/security/cert/X509Certificate.html
  \end{text*}
\end{english}

Команда \code{cider-find-keyword} служит для поиска кейвордов. Если навести курсор на ключ \code{:some.ns/name} и выполнить ее, Cider попытается:

\begin{itemize}

\item
  перейти в пространство \code{some.ns};

\item
  сместиться до первого упоминания \code{::name}.

\end{itemize}

Мы написали <<попытается>>, потому что способ работает только для ключей, пространство которых совпадает с модулем. Если у кейворда произвольное пространство, например \code{:book/name}, поиск не сработает: пространства \code{book} не существует, а перебор всех модулей будет долгим.

Переход к кейворду работает в том числе с псевдонимами. Предположим, пространству \code{company.api.user} задан псевдоним \code{user}, и в коде встречается кейворд \code{::user/email}:

\begin{english}
  \begin{clojure}
(ns some-ns
  (:require
   ...
   [company.api.user :as user]))

(get user ::user/email) ;; M-x cider-find-keyword
                ^       ;; cursor
  \end{clojure}
\end{english}

Наведите на него курсор и выполните \code{cider-find-keyword}. До двойному двоеточию nREPL определит, что пространство~--- псевдоним, и раскроет его.

Переход к кейвордам полезен в работе с \code{clojure.spec}. Спеки объявляют макросом \code{s/def}, который принимает кейворд. Макрос не создает переменную в модуле, а помещает спеку в глобальный реестр с указанными ключом. Найти её командой \code{cider-find-var} будет невозможно. Здесь и пригодится команда \code{cider-find-keyword}, которая работает как навигатор по спекам.

Представим, вы пишете конфигурацию приложения. Поле \code{:db} базы данных ссылается на спеку из модуля \code{cloju\-re.ja\-va.jdbc.spec}:

\begin{english}
  \begin{clojure}
(ns some-ns
  (:require
   [clojure.spec.alpha :as s]
   [clojure.java.jdbc.spec :as jdbc]))

(s/def ::db ::jdbc/db-spec)

(s/def ::config
  (s/keys :req-in [::db]))
  \end{clojure}
\end{english}

\index{JDBC}
\index{модули!clojure.java.jdbc}

Чтобы перейти к определению \code{::jdbc/db-spec}, наведите на него курсор и выполните \code{cider-find-keyword}. Вы окажетесь в файле \code{spec.clj} на строке 78 с макросом \code{s/def}:

\begin{english}
  \begin{clojure}
;; clojure/clojure/java/jdbc/spec.clj
(s/def ::db-spec ...)
  \end{clojure}
\end{english}

Примеры выше подсказывают: используйте ключи, связанные с текущим пространством имен (с двойным двоеточием), например \code{::name} или \code{::email}. В этом случае между ключом и пространством возникает связь, и по одному легко найти другое. Наоборот, ключ с произвольным пространством словно оторван от кода, и это снижает его возможности:

\begin{english}
  \begin{clojure}
:user/id
:error/not-found
  \end{clojure}
\end{english}

Представьте, что встретили в коде спеку \code{:user/id} или событие re-frame \code{:api/get-user}. Совершенно не ясно, где искать их определение. К сожалению, Cider тоже будет не в силах вам помочь.

\subsection{Xref}

\label{section-xref}

\def\urlfastautocomp{https://pypi.org/project/fast-autocomplete/}

\def\urlctags{https://en.wikipedia.org/wiki/Ctags}

\index{Ctags}

С версии 26 в Emacs появился новый способ навигации по коду. Он называется Xref~--- от английского cross-reference, перекрестная ссылка. Особенность Xref~--- в дизайне: модуль поддерживает разные источники (бэкенды), откуда приходят данные об определениях. Источником может быть файл тегов, созданный \footurl{командой ctags}{\urlctags}[Ctags][-5mm]. Также источником может быть функция, если известен иной алгоритм поиска. Например, если это проект на Python, плагин перехватывает вызов Xref и возвращает данные, полученные библиотекой \footurl{fast-autocomplete}{\urlfastautocomp}[Fast autocomplete][5mm] или похожей.

Тип бэкенда не влияет на работу пользователя. Поиск и переход по коду сводятся к командам семейства \code{xref-find-...}.

Чтобы Cider перехватывал вызовы Xref, установите переменную \code{cider-use-xref} в истину. По умолчанию это так, но на всякий случай выполните в \code{*scratch*} выражение

\begin{english}
  \begin{lisp}
(setq cider-use-xref t)
  \end{lisp}
\end{english}

Откройте любой модуль, загруженный в nREPL. Поместите курсор на символ функции и выполните \code{M-x xref-find-de\-fi\-ni\-ti\-ons}. По аналогии с \code{cider-find-var} откроется файл на той строке, где объявлена функция. Способ работает с макросами, протоколами, пространствами имен.

Команде \code{xref-find-definitions} назначена комбинация \code{M-.} (мета с точкой). Она работает и в других режимах Emacs, например \code{lisp-mode} или \code{python-mode} (с модулем Anaconda и аналогами).

\def\urlnsmap{https://clojuredocs.org/clojure.core/ns-map}

Чтобы участвовать в поиске, пространство должно быть загружено в REPL. Cider не ищет функцию локально, а посылает сообщение серверу. Код на сервере обходит загруженные пространства. Их определения получаются функцией \footurl{\code{ns-map}}{\urlnsmap}[ns-map], которая возвращает словарь вида символ \arr определение. Поиск сводится к проверке ключа в словаре, что довольно быстро.

Команда \code{xref-find-references} находит места, где встречается указанный символ. С её помощью проверяют, нуждается ли проект в функции. Если ссылок не найдено, удалите функцию без опасений. Другое применение команды~--- рефакторинг, когда вы изменили сигнатуру функции и теперь исправляете её вызовы.

\def\urlemacsxref{https://www.gnu.org/software/emacs/manual/html\_node/emacs/Xref.html}

Иные возможности Xref вы найдёте на сайте проекта GNU \footurl{в разделе Emacs}{\urlemacsxref}[Emacs Xref][-12mm].

\subsection{Imenu}

\def\urlclojuremode{https://github.com/clojure-emacs/clojure-mode}

Плагин \footurl{Clojure mode}{\urlclojuremode}[Clojure mode] расширяет Imenu в Emacs. Imenu (сокращение от Interactive menu)~--- это встроенный модуль для показа определений в файле. По команде \code{M-x imenu} откроется буфер с оглавлением~--- именами функций, макросов, переменных~--- и приглашением ввода. Приведем краткую версию буфера для модуля \code{clojure.core}:

\begin{english}
  \begin{text}
- Function / any?
- Function / str
- Function / symbol?
  \end{text}
\end{english}

Для каждого языка Imenu хранит набор правил, по которым строится оглавление. В случае с Clojure это шаблоны \code{def}, \code{defn} и другие. Правила можно расширить, чтобы учесть кейворды или особые формы. В этом редко бывает нужда, потому что по умолчанию в Clojure mode заданы обширные правила, в том числе для пакетов \code{clojure.spec} и \code{clojure.test} (формы \code{s/def}, \code{deftest} и другие).

Чтобы Imenu работал в Clojure, добавьте в настройки выражение:

\index{Emacs!Imenu}
\index{Imenu}

\begin{english}
  \begin{lisp}
(add-hook 'clojure-mode-hook #'imenu-add-menubar-index)
  \end{lisp}
\end{english}

Задайте команде \code{imenu} комбинацию клавиш. Автор предпочитает \code{C-i}:

\begin{english}
  \begin{lisp}
(global-set-key (kbd "<C-i>") 'imenu)
  \end{lisp}
\end{english}

Если вы пользуетесь графической версией Emacs, меню удивит вас. Появится всплывающее окно со списком определений. Предполагается, что пользователь перенесёт руку на мышь и выберет нужный пункт.

Странность решения в том, что окно полностью оторвано от Emacs. В нем не работают клавиши навигации, на больших файлах меню не влазит в экран. Перенос руки с клавиатуры на мышь и обратно нарушает идеи редактора. Поэтому назначьте следующей переменной \code{nil}:

\begin{english}
  \begin{lisp}
(setq imenu-use-popup-menu nil)
  \end{lisp}
\end{english}

С ней вместо окна появится буфер Emacs, где работает привычная навигация по элементам.

\def\urlhelm{https://github.com/emacs-helm/helm}

\index{Emacs!Helm}
\index{Helm}

Интерактивное меню станет ещё удобнее с пакетом \footurl{Helm}{\urlhelm}[Helm]. Установите его командой

\index{Emacs!пакеты}

\begin{english}
  \begin{text}
M-x package-install <RET> helm <RET>
  \end{text}
\end{english}

Задайте клавишам \code{C-i} команду \code{helm-imenu}:

\begin{english}
  \begin{lisp}
(global-set-key (kbd "<C-i>") 'helm-imenu)
  \end{lisp}
\end{english}

Helm предлагает более удобные диалоги. При вводе текста он покажет элементы, которые включают его. Например, для ввода \code{user} получим \code{get-user}, \code{delete-user} и другие имена. Обычный \code{imenu} ищет элементы, которые начинаются с текста, что неудобно, если вы не помните точное имя функции.

\section{Тесты в Cider}

\index{Cider!тесты}
\index{тесты}

Работая над программой, мы постоянно запускаем код в REPL. У людей, незнакомых с Lisp и Clojure, складывается ошибочное мнение, что тесты не нужны: зачем их писать, если все проверено в REPL?

Это не так: запуск кода в REPL не отменяет тестов. Когда программа закончена, проверки фиксируют в тестах. Далее код улучшают, постоянно сверяясь с тестами.

Перед тем как начать работу над кодом, убедитесь, что он покрыт тестом. В идеале вы запускаете тест до изменений, чтобы знать предпосылки. Если тест проходит, то вы отталкиваетесь от штатной ситуации. Если после изменений тест <<падает>>, ищите причину в ваших действиях.

Случается, что тесты <<сломаны>> уже до работы над кодом. Например, кто-то измененил код в обход регламента (CI, review). Предварительный прогон тестов покажет, что дело не в ваших изменениях.

В прошлой книге мы подробно разобрали тесты в Clojure. Если коротко, макрос \code{deftest} объявляет функцию, чьё тело находится в поле метаданных \code{:test}. Тесты запускают в особом режиме, когда включены фикстуры и сборщик данных (reporter).

Cider предлагает команды для запуска тестов. Чтобы опробовать их, загрузите модуль с тестами в nREPL. Для этого выполните либо \code{cider-load-buffer} (\code{C-c C-k}), либо \code{cider-ns-refresh} (\code{C-C M-n r}). Во втором случае путь к тестам должен быть в classpath. В lein это легко задать полем \code{source-paths} в профиле \code{dev}:

\begin{english}
  \begin{clojure}
{:profiles
 {:dev
  {:source-paths ["test"]}}}
  \end{clojure}
\end{english}

Приведём минимальный модуль с тестами:

\begin{english}
  \begin{clojure}
(ns sample-test
  (:require
   [clojure.test :refer
     [deftest is]]))

(deftest test-orwell
  (is (= 5 (* 2 2))))
  \end{clojure}
\end{english}

Поместите курсор в любое место \code{deftest} и выполните \code{M-x cider-test-run-test}. Команда запустит тест, при этом на сервер уйдут два сообщения. В первом клиент запросит данные о символе \code{sample/test-orwell}. Это необходимо, чтобы убедиться, что \code{sample/test-orwell}~--- действительно тест:

\begin{english}
  \begin{text}
  op   "info"
  sym  "sample/test-orwell"
  \end{text}
\end{english}

Во втором сообщении клиент отправит действие \code{test} со списком из одного теста:

\begin{english}
  \begin{text}
  op     "test"
  tests  ("test-orwell")
  \end{text}
\end{english}

\pagebreaklarge

Специальное middleware из пакета \code{cider-nrepl}, которое отвечает за тесты, выполнит следующий код:

\begin{english}
  \begin{clojure}
(clojure.test/test-var #'test-orwell)
  \end{clojure}
\end{english}

Также middleware перехватит вывод теста и вернет его в структурированном виде. Вот что получит клиент в положительном случае:

\begin{english}
  \begin{text}
  id      "317"
  column  1
  file    "file:/Users/ivan/work/.../src/sample.clj"
  line    6
  name    "test-orwell"
  ns      "sample"
  status  ("done")
  \end{text}
\end{english}

\noindent
и при ошибке:

\begin{english}
  \begin{text}
  id         "320"
  time-stamp "2022-05-21 20:02:51.069229000"
  gen-input  nil
  results    (dict sample
               (dict test-orwell
                 ((dict "actual" "4"
                        "context" nil
                        "diffs" (("4" ("5" "4")))
                        "expected" "5"
                        "file" "sample.clj"
                        "ns" "sample"
                        "type" "fail"
                        "var" "test-orwell"))))
  summary    (dict error 0 fail 1
                   ns 1 pass 0 test 1 var 1)
  testing-ns "sample"
  \end{text}
\end{english}

Во втором случае откроется буфер \code{*cider-test-report*} с отчетом. Красным цветом показаны места, где оператор \code{(is ...)} вернул ложь. Желтым отмечены формы, где возникло исключение. Ниже~--- отчет о том, что вычисление \code{(* 2 2)} не сошлось с ожидаемым результатом (5):

\begin{english}
  \begin{text}
Test Summary
1 non-passing test(s):

Fail in test-orwell

expected: 5
  actual: 4
    diff: - 5
          + 4
  \end{text}
\end{english}

Исправьте тест, заменив 5 на 4. Чтобы изменения вступили в силу, выполните \code{deftest} при помощи \code{cider-eval-defun-at-point}. По аналогии с функциями и переменными, тест нужно выполнить после изменений в редакторе. Если запустить тест без этого шага, сработает прошлая версия с ошибкой.

Cider предлагает многие другие удобства для тестов. Команда \code{cider-test-rerun-test} повторно выполнит последний запущенный тест. С ней не нужно переключаться между кодом, который вы редактируете, и его тестом. Достаточно выполнить тест и работать над кодом, время от времени вызывая \code{cider-test-rerun-test}.

Команда \code{cider-test-run-ns-tests} выполняет тесты определённого пространства. Если вызвать ее в модуле \code{project.sample}, Cider запустит тесты пространства \code{project.sample-test} (при условии что оно найдено). Следовать этому правилу не обязательно: можно именовать тесты иначе, например с частичкой \code{test} в начале, как в примере ниже. Однако в первом случае их легче выполнить в Cider.

\begin{english}
  \begin{clojure}
(ns test.project.sample)

(deftest test-...)
  \end{clojure}
\end{english}

Команда \code{cider-test-rerun-failed-tests} выполнит только те тесты из прошлого прогона, что окончились неудачей.

\def\urlcidertests{https://docs.cider.mx/cider/testing/running\_tests.html}

Этих команд достаточно для работы с тестами в Clojure. Полный список вы найдете в документации Cider в разделе \footurl{Running Tests}{\urlcidertests}[Running Tests][-10mm].

\section{Отладка сообщений nREPL}

В редких случаях понадобится перехват сообщений между сервером nREPL и Emacs. Например, вы пишете клиент для другого редактора и хотели бы знать, какие сообщения шлет и принимает Emacs, чтобы сделать так же у себя.

\index{nREPL!отладка}

Утилиты \code{tcpdump} и Wireshark в данном случае избыточны. Воспользуйтесь командой \code{nrepl-toggle-message-logging}. Она откроет буфер \code{*nrepl-messages*} с сообщениями текущей сессии. Приведем пару из них в сокращении:

\begin{english}
  \begin{text}
(-->
  id        "27"
  op        "eval"
  session   "..."
  code      "(+ 1 2)"
  column    6
  line      28
  ns        "foo"
)
(<--
  id         "27"
  session    "..."
  time-stamp "2022-06-18 17:15:30"
  value      "3"
)
  \end{text}
\end{english}

Направление стрелки означает характер сообщения: вправо~--- запрос клиента к серверу, влево~--- ответ сервера клиенту. Данные показаны независимо от транспорта (Bencode, EDN), что упрощает их анализ. Перехват сообщений замедляет работу клиента и поэтому выключен по умолчанию. Чтобы отключить перехват, введите команду снова.

\section{Отладка}

\label{section-debug}

\index{REPL!отладка}
\index{отладка}

Перейдём к наиболее важной части главы: рассмотрим, как отлаживать код на Clojure.

Cider предлагает полноценный отладчик, но по некоторым причинам им пользуются редко. Так происходит потому, что принципы Clojure~--- неизменяемость, чистые функции, REPL~--- уже отсекают многие ошибки, свойственные другим языкам. Однако в сложных проектах вам не избежать отладки.

\subsection{Наивные способы}

Прежде чем перейти к отладчику Cider, рассмотрим простые, <<народные>> способы отладить код~--- иногда их бывает достаточно. Предположим, функция принимает словарь опций с точками в названии. В ответ она возвращает вложенный словарь кейвордов. Входные данные и результат:

\begin{english}
  \begin{clojure}
(remap-props
  {"db.host" "127.0.0.1"
   "db.port" 5432
   "db.settings.ssl" false})

{:db
 {:host "127.0.0.1"
  :port 5432
  :settings {:ssl false}}}
  \end{clojure}
\end{english}

Тело функции:

\begin{english}
  \begin{clojure}
(require '[clojure.string :as str])

(defn remap-props [props]
  (reduce-kv
   (fn [result k v]
     (let [path
           (mapv keyword (str/split k #"\."))]
       (assoc-in result path v)))
   {}
   props))
  \end{clojure}
\end{english}

Если в словаре окажется поле, отличное от строки, получим ошибку приведения типа:

\iflarge

\begin{english}
  \begin{text}
(remap-props {"db.host"
              "127.0.0.1"
              :db/port 5432})
  \end{text}
\end{english}

\begin{english}
  \begin{text}
1. Class clojure.lang.Keyword cannot be cast
   to class java.lang.CharSequence

                string.clj:  219  clojure.string/split
                      REPL:   28  sample/remap-props/fn
   PersistentArrayMap.java:  377  clojure.lang...
  \end{text}
\end{english}

\else

\begin{english}
  \begin{text}
(remap-props {"db.host"
              "127.0.0.1"
              :db/port 5432})

1. Class clojure.lang.Keyword cannot be cast
   to class java.lang.CharSequence

                string.clj:  219  clojure.string/split
                      REPL:   28  sample/remap-props/fn
   PersistentArrayMap.java:  377  clojure.lang...
  \end{text}
\end{english}

\fi

В отчёте нет ни слова о том, какой именно ключ привел к ошибке. Самый простой способ узнать его~--- вывести на экран на каждом шаге. Для этого добавим \code{println} во внутреннюю функцию \code{reduce} \coderef{4}:

\begin{english}
  \begin{clojure/lines}
(defn remap-props [props]
  (reduce-kv
   (fn [result k v]
     (println ">>> " k v) ;; debugging
     (let [path
           (mapv keyword (str/split k #"\."))]
       (assoc-in result path v)))
   {}
   props))
  \end{clojure/lines}
\end{english}

Перезагрузите функцию командой \code{cider-eval-defun-at-po\-int}. Вызовите \code{remap-props}, и кроме результата в консоли появятся промежуточные шаги \code{reduce}:

\begin{english}
  \begin{text}
>>>  db.host 127.0.0.1
>>>  db.port 5432
>>>  db.settings.ssl false
  \end{text}
\end{english}

В случае с ошибочным словарем увидим, что дело в ключе \code{:db/port}, который не работает с функцией \code{split}:

\begin{english}
  \begin{text}
>>>  :db/port 5432
  \end{text}
\end{english}

\index{pretty print}
\index{печать с отступами}
\index{модули!clojure.pprint}

Исправьте функцию так, чтобы она проверяла ключ функцией \code{string?}. Если это не так, бросьте исключение с именем ключа. С ним отладка не понадобится, потому что ошибка станет явной. Удалите \code{println} и перезагрузите функцию.

Печать на экран, при всей примитивности, позволяет быстро найти ошибку в коде. Ниже мы рассмотрим её вариации.

Функция \code{println} выводит данные в одну строку, что неудобно для коллекций. Воспользуйтесь печатью с отступами из пакета \code{clojure.pprint}:

\begin{english}
  \begin{clojure}
(require 'clojure.pprint)

(fn [result k v]
  (clojure.pprint/pprint
    {:key k :value v})
  ...)
  \end{clojure}
\end{english}

С ней удобно исследовать запросы и ответы HTTP, потому что они описаны большими словарями.

Вернёмся к функции \code{get-joke} для поиска шуток о программировании. Освежим в памяти её код:

\begin{english}
  \begin{clojure}
(defn get-joke [lang]

  (let [request
        {:url "https://v2.jokeapi.dev/joke/Programming"
         :method :get
         :query-params {:type "twopart" :contains lang}
         :as :json}

        response
        (client/request request)

        {:keys [body]}
        response

        {:keys [setup delivery]}
        body]

    (format "%s %s" setup delivery)))
  \end{clojure}
\end{english}

Чтобы исследовать ответ сервера, добавьте в \code{let} переменную \code{\_} (подчеркивание) и печать \code{response} \coderefs{6 и~7}. Это спорный приём, потому что переменная \code{\_} не используется: она только уравновешивает форму печати. Преимущество в том, что не нужно разрывать цепочку \code{let}-вычислений:

\begin{english}
  \begin{clojure}
(defn get-joke [lang]

  (let [...
        response
        (client/request request)

        _
        (clojure.pprint/pprint response)

        {:keys [body]}
        response

        ...]))
  \end{clojure}
\end{english}

При поиске шутки вы увидите ответ сервера со статусом, заголовками и телом. Как только печать станет не нужна, удалите её.

Вызов \code{pprint} влечёт несколько неудобств. Во-первых, набирать выражение \code{(clojure.pprint/pprint ...)} долго. Во-вторых, нужно импортировать модуль \code{clojure.pprint} в REPL, иначе получим ошибку, что он не загружен. Пойдем на хитрость: сделаем так, чтобы модуль загружался автоматически. Откройте локальные настройки lein (файл \code{\tilde{}/.lein/profiles.clj}). В профиль \code{:user} добавьте ключ \code{:injections} с вектором:

\begin{english}
  \begin{clojure}
{:user
  :injections
   [(require 'clojure.pprint)]}
  \end{clojure}
\end{english}

Выражения \code{injections} выполняются при запуске nREPL. В~них помещают код с побочными эффектами, в том числе загрузку модулей. Приём служит только для разработки.

Перезагрузите nREPL и выполните \code{(clojure.pprint/pprint ...)} в любом месте проекта. Печать сработает без ошибок, и не понадобится импорт \code{clojure.pprint} в заголовке \code{(ns ...)}.

\def\urlwrapreg{https://github.com/rejeep/wrap-region.el}

Чтобы быстро вставить \code{pprint} в код, обратимся к плагину Emacs \footurl{wrap-region}{\urlwrapreg}[Wrap region][-5mm]. С его помощью выделенный текст оборачивают указанными строками. Установите плагин командой

\index{wrap region}
\index{Emacs!wrap-region}

\pagebreaklarge

\begin{english}
  \begin{text}
M-x package-install <RET> wrap-region <RET>
  \end{text}
\end{english}

Затем добавьте в настройки код:

\begin{english}
  \begin{clojure}
(require 'wrap-region)

(wrap-region-mode t)

(wrap-region-add-wrapper
  "(clojure.pprint/pprint "
  ")"
  "p"
  'clojure-mode)
  \end{clojure}
\end{english}

Если выделить \code{response} и нажать \code{p}, появится выражение \code{(clojure.pprint/pprint response)}. Вместо \code{response} может быть любой текст, в том числе коллекция, макрос, вызов функции.

Иногда \code{pprint} выводит слишком много информации, и данные уходят за пределы терминала. Модуль \code{clojure.inspector} решает эту проблему. Он выводит графическое окно Swing с виджетом дерева. Коллекции обозначены папкой, а их элементы~--- файлом.

Окно не блокирует поток, который его вызвал. Код сработает без задержек, и вы не спеша изучите данные в инспекторе. Это особенно важно для задач, которые зависят от времени, например отправка HTTP-запросов, работа с асинхронными каналами.

По аналогии с \code{clojure.pprint}, добавьте в секцию \code{injections} форму \code{(require 'clojure.inspector)}. Задайте клавишу для обертки кода в функцию \code{inspect-tree}:

\begin{english}
  \begin{clojure}
(wrap-region-add-wrapper
  "(clojure.inspector/inspect-tree "
  ")"
  "i"
  'clojure-mode)
  \end{clojure}
\end{english}

\subsection{Внедрение в чужой код}

До сих пор мы отлаживали код в директории \code{src}. Этот код под вашим контролем: в него легко добавить печать и инспекцию, а затем откатить изменения.

Все меняется, когда нужно отладить стороннюю библиотеку. Её код упакован в jar-файл и находится в недрах директории \code{\tilde{}/.m2}. Технически возможно распаковать архив jar, исправить код, упаковать обратно, а затем перезагрузить REPL. Однако это займет массу времени. Способ ниже описывает, как исправить код чужой библиотеки на лету.

Вернемся к функции \code{get-joke} для получения шуток. Функция обращается к сервису JokeAPI.dev при помощи библиотеки \code{clj-http}. Давайте шагнем вглубь \code{clj-http}, чтобы отследить, какие данные уходят в сеть.

Наведите курсор на символ \code{client/request} и выполните \code{M-.} (команда \code{M-x cider-find-var}). Откроется модуль \code{clj-http.cli\-ent} из jar-файла в директории \code{\tilde{}/.m2/repository/clj-http/clj-http/\-3.12.0}. Вы окажетесь на строке 1134, где объявлена переменная \code{request}:

\begin{english}
  \begin{clojure}
(def ^:dynamic request
  "..."
  (wrap-request #'core/request))
  \end{clojure}
\end{english}

У функции длинная документация, которую мы заменили многоточием. Видно, что \code{request} на самом деле ссылается на функцию \code{core/request}, обернутую многими middleware. Установите курсор на \code{core/request} и снова выполните \code{M-.}~--- вы окажетесь в модуле \code{clj-http.core} из того же jar-файла на строке 546:

\begin{english}
  \begin{clojure}
(defn request

  ([req]
   (request req nil nil))

  ([{:keys [...] :as req} respond raise]
   (let [...]
     ...)))
  \end{clojure}
\end{english}

Буфер \code{clj-http.core} открыт в режиме чтения, потому что связан с архивом. Чтобы редактировать код, выполните \code{M-x toggle-read-only}. Теперь, когда буфер доступен для изменений, добавьте инспекцию перед формой \code{let} \coderef{5}:

\index{Emacs!toggle-read-only}

\pagebreaklarge

\begin{english}
  \begin{clojure/lines}
(defn request
  ([req] (request req nil nil))
  ([{:keys [...]
     :as req} respond raise]
   (clojure.inspector/inspect-tree req)
   (let [...]
     ...)))
  \end{clojure/lines}
\end{english}

Обновите функцию на сервере командой \code{M-x cider-eval-defun-at-point}. Теперь вызов \code{client/request} покажет окно инспектора с полями запроса. Это касается не только функции \code{get-joke}, но и любого обращения к Clj-http.

Как только инспекция станет не нужна, вернитесь в буфер \code{clj-http.core}. Откатите изменения командой \code{C-/} (undo) и обновите функцию на сервере. В последующих запросах инспектор не появится.

\index{Clj-http}
\index{библиотеки!Clj-http}

Буфер \code{clj-http.core} будет отмечен как изменённый, и при закрытии Emacs предложит его сохранить. Откажитесь, потому что отладка не должна менять исходный код библиотек. Если ответить утвердительно, Emacs обновит архив, и изменения коснутся каждого проекта с этой библиотекой. В этом случае удалите jar с диска и перезагрузите REPL~--- библиотека скачается из репозитория.

Описанная техника работает со всеми модулями, в том числе встроенными в Clojure. Ради интереса перейдите в модули \code{clojure.walk}, \code{clojure.string} и другие. Добавьте в код побочные эффекты, проверьте изменения в REPL и откатите их.

\subsection{Подготовка к отладке}

\index{отладка}

Кроме печати и инспекции, Cider предлагает полноценный отладчик. С ним код выполняют по шагам, следят за локальными переменными и стеком вызовов, словом, делают всё, что доступно в современных IDE. Чтобы читатель лучше понял отладку, поговорим об её устройстве.

Вспомним, как работает отладка в IDE. Напротив строки ставят красную метку (точку останова) и запускают код. Когда исполнение достигает метки, программа останавливается и ждёт команды пользователя. При выходе из отладки программа продолжит ход.

Можно сказать, отладка работает как REPL, запущенный в середине кода. Это бесконечный цикл, который ожидает команду, выполняет её, выводит результат и вновь ожидает команду. Разница в том, что отладчик не только выполняет код. Он собирает множество данных о нем: переменные, стек вызовов, метрики и так далее.

Напишем простой отладчик для Clojure. Предположим, мы ничего не знаем о Cider и nREPL, поэтому используем только встроенные средства. Подготовим функцию \code{format-user}, на которой будем тренироваться:

\begin{english}
  \begin{clojure}
(ns debug
  (:require [clojure.main :as main]))

(defn format-user
  [{:keys [username email]}]
  (format "%s <%s>" username email))
  \end{clojure}
\end{english}

Проверим ее, подав на вход словарь:

\begin{english}
  \begin{clojure}
(format-user {:username "John"
              :email "john@test.com"})
;; "John <john@test.com>"
  \end{clojure}
\end{english}

Теперь сделаем так, чтобы на середине функции включился REPL. Для этого служит одноимённая функция из модуля \code{clojure.main}. Если вставить её в тело \code{format-user} и вызвать последнюю, при запуске она прервется, и откроется приглашение:

\begin{english}
  \begin{clojure/lines}
(defn format-user
  [{:keys [username email]}]
  (main/repl :prompt #(print "DEBUG>> "))
  (format "%s <%s>" username email))
  \end{clojure/lines}
\end{english}

\begin{english}
  \begin{text}
user=> (format-user {:username "John"
  #_=>               :email "john@test.com"})

DEBUG>> (+ 1 2)
3
  \end{text}
\end{english}

Обратите внимание на разницу в приглашении. Мы задали внутреннему REPL параметр \code{:prompt}, чтобы лучше понимать, в каком сеансе пребываем сейчас \coderef{3}. Для выхода из отладки нажмите \code{Ctrl/Command+D}. Сочетание подаст на вход символ \code{EOF}, что означает завершение. Управление выйдет из внутреннего REPL, и вы получите результат \code{format-user}:

\begin{english}
  \begin{text}
DEBUG>> ;; Ctrl/Command+D
"John <john@test.com>"
user=>
  \end{text}
\end{english}

Во время отладки вам захочется узнать локальные переменные, например выяснить, чему равны \code{username} и \code{email}. Тут вас ждет неприятность: если ввести \code{username}, REPL выдаст исключение о том, что символ неизвестен. То же самое относится к переменным \code{let}: в примере ниже символы \code{a} и \code{b} окажутся недоступны.

\begin{english}
  \begin{text}
user=> (let [a 1
  #_=>       b 2]
  #_=>   (main/repl :prompt #(print "DEBUG>> ")))
DEBUG>> a
Syntax error compiling at (REPL:1:1).
Unable to resolve symbol: a in this context
DEBUG>>
  \end{text}
\end{english}

Причина в том, что функция \code{eval}, с помощью которой REPL выполняет код, не учитывает локальные переменные. Мы упоминали эту проблему в начале главы, и настало время решить её.

\subsection{Продвинутый eval}

\index{eval}

В этом разделе мы напишем функцию \code{eval+}~--- улучшенную версию обычной \code{eval}. Она принимает пространство имён, словарь локальных переменных и форму, которую нужно вычислить. Вот как выглядит её сигнатура:

\begin{english}
  \begin{clojure}
(defn eval+ [ns locals form]
  ...)
  \end{clojure}
\end{english}

Представим вызов функции: вычислим форму \code{'(+ a b)} при $a = 1$ и $b = 2$ в пространстве \code{clojure.core}:

\begin{english}
  \begin{clojure}
(eval+ (the-ns 'clojure.core)
       {'a 1 'b 2}
       '(+ a b))
;; 3
  \end{clojure}
\end{english}

Прежде чем браться за код, выясним, где взять входные параметры. С аргументом \code{ns} нет сложностей, потому что текущее пространство имен доступно в переменной \code{*ns*}:

\begin{english}
  \begin{clojure}
> *ns*
;; #namespace[user]
  \end{clojure}
\end{english}

Если мы знаем имя пространства, его объект легко получить функцией \code{the-ns}:

\begin{english}
  \begin{clojure}
> (the-ns 'clojure.core)
;; #namespace[clojure.core]
  \end{clojure}
\end{english}

Локальные переменные (второй аргумент \code{locals})~--- это словарь, ключи которого~--- символы. Он выполняет роль контекста при вычислении формы: разные переменные дают разный результат.

\begin{english}
  \begin{clojure}
(eval+ ... {'a 1 'b 2} '(+ a b)) ;; 3
(eval+ ... {'a 3 'b 4} '(+ a b)) ;; 7
  \end{clojure}
\end{english}

Как получить локальные переменные для текущего участка кода? Их сбор должен быть автоматическим, а не ручным. Для этого напишем макрос \code{get-locals}:

\begin{english}
  \begin{clojure/lines}
(defmacro get-locals []
  (into {} (for [sym (keys &env)]
             [(list 'quote sym) sym])))
  \end{clojure/lines}
\end{english}

\index{классы!LocalBinding}
\index{LocalBinding}

Он опирается на скрытую переменную \code{\&env}, доступную только макросам \coderef{2}. Это словарь, где ключи~--- символы, а значения~--- экземпляры класса \code{LocalBinding}. Объект \code{LocalBinding} содержит метаданные о локальной переменной. Среди них нет значения переменной, но оно и не понадобится. Форма \code{(get-locals)} возвращает словарь, где ключи~--- <<замороженные>> символы переменных, например \code{(quote a)}, а значения~--- обычные символы \code{a} или \code{b}.

Покажем макрос в действии: при компиляции кода форма в левой колонке становится как во второй. Вычислив ее, получим словарь переменных в третьей колонке.

\begin{english}
\noindent
\begin{tabular}{ @{}p{4cm} @{}p{4cm} @{}p{3cm} }

  \begin{clojure}
;; source code
(let [a 1
      b 2]
  (get-locals))
  \end{clojure}

&

  \begin{clojure}
;; compilation
(let [a 1
      b 2]
  {(quote a) a
   (quote b) b})
  \end{clojure}

&

  \begin{clojure}
;; result
{'a 1 'b 2}
  \end{clojure}

\end{tabular}
\end{english}

Более сложный пример вызовом функции. Видно, что \code{get-locals} захватил аргументы \code{a} и \code{b} и переменную \code{c} из формы \code{let}:

\begin{english}
  \begin{clojure}
(defn add
  [a b]
  (let [c (+ a b)]
    (println "Locals:" (get-locals))
    (+ a b c)))

> (add 1 2)
;; Locals: {a 1, b 2, c 3}

6
  \end{clojure}
\end{english}

Теперь, когда у нас есть переменные, подумаем, как выполнить форму. Задача сводится к тому, чтобы как-то передать их в \code{eval}. Для этого есть несколько способов. Первый~--- временно сделать локальные переменные глобальными. Назовём этот трюк глобализацией. Чтобы <<глобализировать>> переменные, нужно:

\begin{itemize}

\item
  обойти словарь \code{locals} в цикле \code{doseq};

\item
  внедрить переменные в пространство имен функцией \code{intern};

\item
  вычислить форму при помощи \code{eval} и запомнить результат;

\item
  удалить внедренные переменные функцией \code{ns-unmap};

\item
  вернуть результат.

\end{itemize}

\pagebreaklarge

Вот как выглядит черновик \code{eval+} с этим алгоритмом:

\begin{english}
  \begin{clojure/lines}
(defn eval+ [ns locals form]
  (doseq [[sym value] locals]
    (intern ns sym value))
  (let [result
        (binding [*ns* ns]
          (eval form))]
    (doseq [[sym value] locals]
      (ns-unmap ns sym))
    result))
  \end{clojure/lines}
\end{english}

Проверка показывает, что всё верно:

\begin{english}
  \begin{clojure}
(eval+ *ns* {'a 1 'b 2} '(+ a b))
;; 3
  \end{clojure}
\end{english}

Убедимся, что мы не оставили за собой глобальных переменных: вне формы \code{eval+} символ a неизвестен:

\begin{english}
  \begin{text}
=> a
;; Unable to resolve symbol: a in this context
  \end{text}
\end{english}

Обратите внимание, что \code{(eval form)} \coderef{6} находится в макросе \code{binding} с привязкой пространства, которое передали в параметрах. Без него вычисление сработает в пространстве \code{clojure.core}, и локальные переменные не подхватятся.

Наша <<глобализация>> не учитывает важный момент. Если в словаре указана переменная \code{a} и такая же переменная задана в пространстве, после очистки мы потеряем её:

\begin{english}
  \begin{clojure}
=> (def a 3)

=> (eval+ *ns* {'a 1 'b 2} '(+ a b))
3

=> a
Unable to resolve symbol: a in this context
  \end{clojure}
\end{english}

\index{Var}
\index{классы!Var}

Доработайте код, чтобы перед функциями \code{intern} и \code{ns-unmap} была проверка, существует ли переменная с таким именем. Если да, переименуйте её в \code{\_\_old\_<var>\_\_}. На обратном пути, если \code{\_\_old\_<var>\_\_} существует, восстановите её в \code{<var>}. Чтобы проверить, существует переменная или нет, используйте \code{resolve}. Результат будет либо \code{nil}, либо \code{Var}. Значение переменной легко получить, <<дерефнув>> \code{Var} оператором \code{@} или функцией \code{deref}, предварительно проверив на \code{nil}.

Второй и более правильный способ выполнить форму с локальными переменными~--- сдвинуть их внутрь \code{eval}. Для этого погрузим форму в оператор \code{let} по следующему принципу:

\begin{english}
\noindent
\begin{tabular}{ @{}p{3.5cm} @{}p{3cm} @{}p{3cm} }

  \begin{clojure}
;; locals
{'a 1 'b 2}
  \end{clojure}

&

  \begin{clojure}
;; form
'(+ a b)
  \end{clojure}

&

  \begin{clojure}
;; final form
'(let [a 1 b 2]
   (+ a b))
  \end{clojure}

\end{tabular}
\end{english}

Если выполнить третью форму, получим ожидаемый результат 3. В этом подходе нет махинаций с глобальными переменными, что делает его безопаснее.

Рассмотрим, как составить подобную форму \code{let}. На первый взгляд задача кажется легкой: это список, где первый элемент~--- символ \code{let}, второй~--- вектор связывания, а третий~--- форма, которую вычисляют. Составим функцию \code{make-eval-form}:

\begin{english}
  \begin{clojure}
(defn make-eval-form
  [locals form]
  (list 'let (vec (mapcat identity locals)) form))
  \end{clojure}
\end{english}

\noindent
и убедимся в её работе:

\begin{english}
  \begin{clojure}
=> (make-eval-form {'a 1 'b 2} '(+ a b))

;; (let [a 1 b 2] (+ a b))
  \end{clojure}
\end{english}

Если выполнить результат в \code{eval}, получим 3. Однако более сложные примеры не сработают. Предположим, одна из переменных содержит список~--- не вектор, а именно список чисел:

\begin{english}
  \begin{clojure}
(make-eval-form {'numbers (list 1 2 3)}
                '(count numbers))
  \end{clojure}
\end{english}

\pagebreaklarge

В результате получится форма:

\begin{english}
  \begin{clojure}
(let [numbers (1 2 3)]
  (count numbers))
  \end{clojure}
\end{english}

Компилятор не сможет вычислить \code{(1 2 3)}, потому что 1 не является функцией. Чтобы список остался списком, он должен предстать в виде \code{(list 1 2 3)}, что требует лишних усилий.

Ещё одна ловушка кроется в представлении значений: не все из них могут быть прочитаны парсером Clojure. Например, если напечатать вектор \code{[1 2 3]}, получим строку, которая вычисляется в такой же вектор. В широком смысле \emph{представление} вектора совпадает с его \emph{синтаксисом}. То же самое относится к словарю и простым типам: числам, строкам, кейвордам. Каждый из них выглядит как в коде.

Однако другие классы представляют объект строкой, которая нарушает синтаксис Clojure. Примером служит класс \code{File}:

\begin{english}
  \begin{clojure}
(new java.io.File "test.txt")
;; #object[java.io.File "test.txt"]
  \end{clojure}
\end{english}

Очевидно, строку \code{\#object[java.io.File "test.txt"]} нельзя вычислить в REPL. Выражение с переменной \code{file}, как в примере ниже, даст форму, несовместимую с \code{eval}:

\begin{english}
  \begin{clojure}
(make-eval-form
 {'file (new java.io.File "test.txt")}
 '(.getAbsolutePath file))

(let [file #object[java.io.File 0x4e293fac "test.txt"]]
  (.getAbsolutePath file))
  \end{clojure}
\end{english}

Чтобы избежать ошибки, идут на интересный трюк. В правой части вектора \code{let} помещают не значение, а код, который получает его из некоего источника. С ним не нужно опасаться, что объект \code{File} вызовет ошибку синтаксиса:

\begin{english}
  \begin{clojure}
(eval '(let [file (get ... 'file)]
         (slurp file)))
  \end{clojure}
\end{english}

Осталось понять, чем является источник. Подойдет глобальная динамическая переменная \code{*locals*}, которую временно связывают с локальными переменными. Это ещё одна тонкость функции \code{eval}: она игнорирует локальные переменные, но учитывает динамические. Проверим это на практике: вычислим форму с переменной \code{*num*} в момент, когда она временно равна 3:

\begin{english}
  \begin{clojure}
(def ^:dynamic *num* 0)

(binding [*num* 3]
  (eval '(* *num* *num*)))
;; 9
  \end{clojure}
\end{english}

Объявим приватную динамическую переменную \code{*locals*}:

\begin{english}
  \begin{clojure}
(def ^:dynamic ^:private
  *locals* nil)
  \end{clojure}
\end{english}

С ней новая версия \code{eval+} выглядит так:

\begin{english}
  \begin{clojure/lines}
(defn eval+ [ns locals form]
  (binding [*locals* locals
            *ns* ns]
    (eval
     `(let ~(reduce
             (fn [result sym]
               (conj result sym `(get *locals* '~sym)))
             []
             (keys locals))
        ~form))))
  \end{clojure/lines}
\end{english}

Внутренняя форма \code{reduce} \coderef{5} производит вектор связывания для \code{let}. Обратите внимание, что значения переменных не участвуют в коде~--- нужны только их имена, чтобы составить пары \code{[x (get *locals* x)]}. Поэтому в \code{reduce} передаются только ключи локальных переменных, а не словарь целиком \coderef{4}. Вот что построит \code{reduce} для переменных \code{a} и \code{b}:

\begin{english}
  \begin{clojure}
[a (get *locals* 'a)
 b (get *locals* 'b)]
  \end{clojure}
\end{english}

Теперь, когда функция \code{eval+} готова, перейдём к последнему шагу~--- напишем свой отладчик для Clojure.

\subsection{Отладчик своими руками}

\index{отладчик}

\label{section-own-debugger}

\index{break}
\index{макросы!break}

Наш отладчик представлен макросом \code{break}, который работает как точка останова. Он принимает форму и запускает внутренний REPL. Выполнение формы откладывается, и пользователю доступны команды: справка, просмотр переменных, выполнение кода. При выходе из отладки управление переходит к форме. Вот как это выглядит в коде:

\begin{english}
  \begin{clojure}
(let [a 1
      b 2]
  (break (+ a b)))
;; 3
  \end{clojure}
\end{english}

Перед вычислением \code{(+ a b)} запустится REPL, в котором доступны переменные \code{a} и \code{b}. Когда отладка закончена, получим результат 3.

Подготовим черновик \code{break}. Он предваряет форму функцией \code{break-inner}, которая принимает пространство и локальные переменные. Результат макроса~--- исходная форма, которую он принял. Из-за этого макрос не скажется на вычислениях, что очень важно: отладка может замедлять работу программы, но отличие в результах недопустимо.

\begin{english}
  \begin{clojure}
(defmacro break
  [form]
  `(do
     (break-inner *ns* (get-locals))
     ~form))
  \end{clojure}
\end{english}

Функция \code{break-inner} работает как внутренний REPL с той особенностью, что некоторый ввод считается командой. Пока что реализуем четыре команды: печать локальных переменных, выполнение кода, справку и выход. Договоримся о синтаксисе: символ \code{!locals} означает вывести локальные переменные; по команде \code{!exit} отладка завершается. Символ \code{!help} служит для справки. Всё остальное отладчик воспринимает как код, который нужно выполнить. Вот как выглядит \code{break-inner}:

\pagebreaklarge

\begin{english}
  \begin{clojure}
(defn break-inner [ns locals]
  (loop []
    (let [input (read-line)
          form (read-string input)]
      (if (= form '!exit)
        (println "Bye")
        (let [result
              (case form
                !locals locals
                !help "Help message..."
                (eval+ ns locals form))]
          (println result)
          (recur))))))
  \end{clojure}
\end{english}

Добавьте макрос \code{break} в любое место кода и запустите его. Он сработает как точка останова в IDЕ: код прервется, и вы окажетесь в отладке. Приведём сеанс отладчика с простой формой \code{let}:

\begin{english}
  \begin{clojure}
(let [a 1 b 2]
  (break (+ a b)))

=> a
1

=> (+ a a b b)
6

=> !locals
{a 1, b 2}

=> !help
Help message...

=> !exit
3
  \end{clojure}
\end{english}

Команда \code{!exit} завершит отладку, и вы получите результат~--- число 3.

Примените к отладчику улучшения, что мы рассмотрели в начале главы: печать при помощи \code{pprint}, перехват исключений, переменные \code{*1}, \code{*2}, \code{*3}, \code{*e} и всё остальное.

Недостаток брейкпоинта в том, что он принимает команды только с клавиатуры. Продвинутая версия должна использовать сетевой протокол или графический интерфейс. В случае с сетью отладчик можно совместить с nREPL и Cider. Понадобится middleware, которое свяжет команды клиента с нашим отладчиком.

Для интерфейса можно использовать встроенный пакет Swing или браузер. В момент отладки запускается локальный HTTP-сервер. Функция \code{browse-url} из модуля \code{clojure.java.browse} открывает браузер на локальном хосте. Интерфейс строится на технологиях HTML, CSS и JavaScript. Браузер и сервер обмениваются данными через JSON API.

\subsection{Множественная отладка (теория)}

\index{debug}
\index{макросы!debug}

Выше мы покрыли отладкой только одну форму \code{(+ a b)}. На ней исполнение прервется, а затем продолжится. На практике сложный код исследуют по шагам: от одной формы переходят к другой, пока проблема не устранена. Так происходит потому, что порой трудно понять, где именно закралась ошибка. Точку останова ставят приблизительно и шагают по коду, сверяясь с состоянием программы.

Подумаем, как сделать пошаговую отладку. Это макрос \code{debug}, который принимает сложную форму и расставляет точки останова в её содержимом, в том числе вложенных формах. Например, форма \code{let} со сложением двух чисел после обработки макросом выглядит так:

\begin{english}
  \begin{clojure/lines}
(let [a (break 1)   ;; #1
      b (break 2)]  ;; #2
  (break (+ a b)))  ;; #3
  \end{clojure/lines}
\end{english}

Если её выполнить, процесс станет похож на настоящую отладку. Сперва вы окажетесь в первой точке \code{(break 1)}. В этот момент не доступна ни одна локальная переменная. В точке \code{(break 2)} появится доступ к переменной \code{a}. Выйдя из неё, вы окажетесь в третьей точке, где доступны \code{a} и \code{b}. Покинув третью точку, вы получите результат 3.

Обратите внимание, что в \code{let} нельзя оборачивать левую часть связывания. Если сделать это как в примере ниже, получим ошибку синтаксиса:

\begin{english}
  \begin{clojure}
(let [(break a) (break 1)
      (break b) (break 2)]
  (break (+ a b)))
  \end{clojure}
\end{english}

\code{Let}, точнее её внутренний вариант \code{let*}, относится к особым формам, синтаксис которых нельзя нарушать. Похоже устроены формы \code{def}, \code{defn}, \code{if} и другие. Некоторые их элементы опорные, потому что на них полагается парсер Clojure.

Мы не будем писать макрос \code{debug}, а только предположим, как он выглядит. Макрос принимает форму и обходит её сверху вниз. Для обхода и изменения дерева понадобятся модули \code{clojure.walk} или \code{clojure.zip}. Напишем наивную версию макроса:

\index{модули!clojure.walk}

\begin{english}
  \begin{clojure}
(require
  '[clojure.walk :as walk])

(defn wrapper
  [el]
  (list 'break el))

(defmacro debug
  [form]
  (walk/postwalk wrapper form))
  \end{clojure}
\end{english}

Проверим, что получится, если передать макросу форму \code{let}. Для развертки макроса служит функция \code{macroexpand}:

\begin{english}
  \begin{clojure}
(macroexpand
 '(debug
   (let [a 1 b 2]
     (+ a b))))
  \end{clojure}
\end{english}

Результат:

\begin{english}
  \begin{clojure}
(break
 ((break let)
  (break [(break a) (break 1)
          (break b) (break 2)])
  (break ((break +) (break a) (break b)))))
  \end{clojure}
\end{english}

На выходе форма \code{let}, где каждый элемент покрыт точкой останова. Очевидно, мы перестарались, потому что в таком виде результат нельзя скомпилировать. Функция \code{wrap} из \code{walk/postwalk} должна действовать более тонко: определять формы \code{let}, \code{def}, \code{if} и обрабатывать их особо.

Измените \code{wrap} таким образом, чтобы она опиралась на функции \code{needs-debug?} и \code{wrap-debug}. Первая проверяет, нужно ли оборачивать форму, а вторая делает это с учётом синтаксиса.

\begin{english}
  \begin{clojure}
(fn wrap [el]
  (if (needs-debug? el)
    (wrap-debug el))
  el)
  \end{clojure}
\end{english}

Это нетривиальное задание: с учётом всех тонкостей оно займёт несколько экранов. Сделайте так, чтобы код был расширяемым при помощи мультиметода. Начните с форм \code{let} и \code{def}, потому что они встречаются чаще других.

\def\urlcorespecalpha{https://github.com/clojure/core.specs.alpha}

Некоторые формы допускают разную запись. Например, у \code{defn} может быть несколько тел, строка документации, pre- и post-проверки и многое другое. Чтобы не плодить if/else, приведите их к единому виду. Один из способов это сделать~--- разобрать форму на части функцией \code{conform} из Clojure.spec. Для разбора понадобятся определения; взять их можно из пакета \footurl{clojure.core.specs.alpha}{\urlcorespecalpha}[Core Specs Alpha], где собраны спеки основных конструкций языка: \code{ns}, \code{let}, \code{def} и других.

\subsection{Отладочный тег}

Макросом \code{(break ...)} будет проще пользоваться, если назначать ему тег \code{\#my/break} или похожий. Вот как это выглядит в коде:

\begin{english}
  \begin{clojure}
(let [a 1 b 2]
  #my/break (+ a b))
  \end{clojure}
\end{english}

Мы добавили пространство \code{my}, потому что тег \code{\#break} уже занят пакетом Cider. Чтобы связать тег с функцией, создайте в директории \code{src} файл \code{data\_readers.clj} со словарем:

\begin{english}
  \begin{clojure}
{my/break my.namespace/break-reader}
  \end{clojure}
\end{english}

Ключ словаря~--- имя тега, а значение~--- полный путь к функции, которая его раскрывает. Функция принимает форму, стоящую после тега, и возвращает новую форму. В нашем случае \code{break-reader} обернет форму в \code{break}:

\begin{english}
  \begin{clojure}
(defn break-reader [form]
  `(break ~form))
  \end{clojure}
\end{english}

Проведите эксперименты с тегом \code{\#my/break}. Расставьте их в коде и убедитесь, что отладка запускается. Добавьте в редактор сочетание клавиш, которое вставляет тег на текущее место курсора.

\section{Отладка в Cider}

\index{Cider!отладка}
\index{теги!break}

Мы исследовали отладку так долго, чтобы читатель убедился: в~ней нет никакой магии. Отладчик~--- это код, который внедряется в исходный код и заставляет его работать с паузами. Во время паузы отладчик ждет команду пользователя и выполняет ее.

Теперь, когда вы знакомы с самодельным отладчиком, рассмотрим, что предлагает Cider. В нашем распоряжении два тега: \code{\#break} и \code{\#dbg}. Первый тег означает точку останова в том месте, где он расположен. Поставьте \code{\#break} в середину произвольного кода. Перед тем как запустить код, выполните его командой \code{cider-eval-...}, иначе эффект не вступит в силу.

Тег \code{\#break} ссылается на функцию \code{breakpoint-reader} из модуля \code{cider.nrepl.middleware.debug}. Она принимает форму и добавляет в её метаданные признак отладки. Далее сработает оснащение (или инструментирование)~--- алгоритм, который ищет отмеченные формы и оборачивает их кодом, который запускает отладку.

Когда оснащённый код запущен, в нужных местах он прерывается, и от клиента ожидают команду. Можно узнать локальные переменные, выполнить выражение или перейти к следующей точке. Так продолжается до тех пор, пока код не выполнен целиком.

В Emacs нет графических средств отладки. Информация выводится либо рядом с кодом, либо в отдельных буферах. В режиме отладки файл нельзя редактировать: клавиши не вводят текст, а вызывают команды. Ошибка новичков в том, что, попав в отладку, они нажимают все подряд, и процесс протекает с ошибками. Мы рассмотрим отладчик так, чтобы с вами этого не случилось.

\index{Joke API}

Вернёмся к функции \code{get-joke} для поиска шуток. Освежим в памяти её код:

\begin{english}
  \begin{clojure}
(defn get-joke [lang]
  (let [request
        {:url "https://v2.jokeapi.dev/joke/Programming"
         :method :get
         :query-params {:type "twopart" :contains lang}
         :as :json}

        response
        (client/request request)

        {:keys [body]}
        response

        {:keys [setup delivery]}
        body]
    #break
    (format "%s %s" setup delivery)))
  \end{clojure}
\end{english}

Добавьте тег \code{\#break} перед \code{(format ...)} и выполните \code{cider-eval-defun-at-point}. Обратите внимание, что тег может стоять как на одной строке с формой, так или на предыдущей:

\begin{english}
\begin{clojure}
#break (format ...)     #break
                        (format ...)
  \end{clojure}
\end{english}

Во втором случае тег легко удалить командами \code{kill-line} или \code{kill-whole-line}, не затрагивая код.

Запустите функцию с любым аргументом, например \code{(get-joke "python")}. Буфер перейдет в режим отладки. Над функцией появится меню действий. Это команды \code{continue}, \code{next}, \code{locals} и другие, которые мы рассмотрим позже. После стрелки \code{=>} показан результат формы, на которой вы остановились.

\begin{english}
  \begin{clojure}
Continue Next In Out Here Eval Inspect Locals Inject

(defn get-joke [lang]
  (let [request
        {:url "https://v2.jokeapi.dev/joke/Programming"
         :method :get
         :query-params {:type "twopart" :contains lang}
         :as :json}
        ...]
    #break
    (format "%s %s" setup delivery)))

 => "why do python programmers wear glasses? \
     Because they can't C#."
  \end{clojure}
\end{english}

В режиме отладки Emacs прослушивает клавиши \code{c}, \code{n}, \code{l} и другие, связанные с отладкой. Например, \code{l} (locals) покажет локальные переменные. Нажмите её, и появится буфер со словарем:

\begin{english}
  \begin{clojure*}{fontsize=\small}
Class: clojure.lang.PersistentArrayMap
Contents:
  body = { :category "Programming", :delivery "...",  ... }
  response = { :cached nil, :request-time 285,  ... }
  request = { :method :get, :query-params ... }
  delivery = "Because they can't C#."
  setup = "why do python programmers wear glasses?"
  \end{clojure*}
\end{english}

Клавиша \code{p} (inspect) исследует значение под курсором. Если её нажать, откроется буфер:

\begin{english}
  \begin{text}
Class: java.lang.String
Value: "why do python programmers wear glasses? ..."
  \end{text}
\end{english}

Клавиша \code{P} исследует произвольное значение. По нажатии Emacs запросит значение в мини-буфере ввода. Введите \code{response}, чтобы изучить ответ сервера:

\begin{english}
  \begin{clojure}
Class: clojure.lang.PersistentHashMap
Contents:
  :status = 200
  :headers = { "Server" "cloudflare", ... }
  :body = { :category "Programming",  ... }
  \end{clojure}
\end{english}

Поле \code{:body} не уместилось целиком. Подведите курсор к фигурным скобкам и нажмите \enter~--- оно откроется во вложенном буфере:

\begin{english}
  \begin{clojure}
Class: clojure.lang.PersistentHashMap
Contents:
  :category = "Programming"
  :delivery = "Because they can't C#."
  :type = "twopart"
  :setup = "why do python programmers wear glasses?"
  :lang = "en"
  :id = 294
  :error = false
  :safe = true
  :flags = { :nsfw false, :religious false, ... }
  \end{clojure}
\end{english}

Команда eval (клавиша \code{e}) выполнит произвольный код. Cider запросит его в мини-буфере:

\begin{english}
  \begin{text}
Expression to evaluate: (keys body)
=> (:category :delivery :type :setup ...)
  \end{text}
\end{english}

Недостаток мини-буфера в том, что он принимает одну строку и поэтому не подходит для сложных выражений.

Клавиша \code{s} (stacktrace) открывает стек вызовов. С его помощью мы узнаем, как пришли в текущее место. Вывод ниже, хоть и кажется шумным, верно показывает историю вызовов. Сократим его, оставив только значимую часть:

\begin{english}
  \begin{text/lines}
    debug.clj:  294  c.n.m.debug/debug-stacktrace
    debug.clj:  368  c.n.m.debug/read-debug-command
    debug.clj:  519  c.n.m.debug/break
         REPL:  128  sample/get-joke
         REPL:   78  sample/eval9967
Compiler.java: 7131  clojure.lang.Compiler/eval
     core.clj: 3210  clojure.core/eval
  \end{text/lines}
\end{english}

Мы вызвали в REPL форму \code{(get-joke "python")}, что соответствует \code{sample/eval9967} \coderef{5}. Далее шагнули в функцию \code{sample/get-joke} \coderef{4}. Функция оснащена отладкой, поэтому в ней был макрос, который вызывает \code{break}. В функции \code{break} случился вызов \code{read-debug-command}, которая отвечает за обработку команды отладчика \coderef{2}. По нажатии \code{s} и поступила команда на вывод стектрейса. Этим занимается функция \code{debug-stacktrace}, которая оказалась последней \coderef{1}.

Клавиша \code{q} (quit) завершит отладку, и код выполнится до конца без остановок.

Кроме перечисленных команд, доступны навигация, изменение локальных переменных, трассировщик и другие. Некоторые из них мы рассмотрим позже.

Когда код отлажен, удалите тег \code{\#break} и выполните форму ещё раз. При новом запуске отладки не будет.

Точка останова работает с любыми модулями. По аналогии с печатью (вставкой \code{println}), откройте пространство из jar-файла. Снимите режим <<только для чтения>>, добавьте тег \code{\#break} в нужную функцию и выполните её на сервере. Запустите код, который вызывает эту функцию, и вы окажетесь в отладке.

\index{теги!dbg}

Кроме \code{\#break}, Cider предлагает \code{\#dbg}~--- более мощный тег, который поддерживает навигацию по коду. Под навигацией имеют в виду команды next (переход к следующей форме), step in (шаг внутрь), step out (выход из текущей формы), продолжение до курсора и другие. С этим тегом отладка ведёт себя как в современных IDE.

Чтобы <<зарядить>> функцию отладкой, поставьте перед формой \code{(defn ...)} тег \code{\#dbg}. Чтобы не смещать код вправо, расположите его строкой выше:

\begin{english}
  \begin{clojure}
#dbg
(defn get-joke [lang]
  ...)
  \end{clojure}
\end{english}

Того же эффекта можно добиться командой \code{cider-debug-defun-at-point}, при этом курсор может быть в любом месте функции. Когда функция заряжена, выполните её:

\begin{english}
  \begin{clojure}
(get-joke "C#")
  \end{clojure}
\end{english}

Вы окажетесь в отладке, но не в конце функции, где раньше стоял \code{\#break}, а на этапе вычисления \code{request}. Обозначим комментарием вашу позицию \coderef{5}:

\begin{english}
  \begin{clojure/lines}
(defn get-joke [lang]
  (let [request
        {:url "https://v2.jokeapi.dev/joke/Programming"
         :method :get
         :query-params {:contains lang} => "C#" ;; <
         :as :json}
  \end{clojure/lines}
\end{english}

Из локальных переменных доступна только \code{lang}. Нажмите \code{n} (next), и управление перейдёт к отправке запроса:

\begin{english}
  \begin{clojure}
response
(client/request request) => {:method :get, ...}
  \end{clojure}
\end{english}

Справа от стрелки показано значение текущей формы, в нашем случае \code{request}. Продолжайте отладку нажатием \code{n}, и постепенно вы обойдёте всю функцию. В теле \code{(format ...)} будет две точки останова: на месте \code{setup} и \code{delivery}. Обозначим их вертикальной чертой:

\begin{english}
  \begin{clojure}
(format "%s %s" setup| delivery) =>
  "Why do programmers wear glasses?"

(format "%s %s" setup delivery|) =>
  "Because they need C#"
  \end{clojure}
\end{english}

\index{навигация}

Перечислим другие команды навигации. Наиболее важные из них~--- это in (ступить на уровень ниже) и out (подняться выше). Запустите отладку ещё раз и дождитесь, пока курсор не окажется на форме \code{(client/request ...)}. Нажмите \code{i}, чтобы управление перешло в функцию \code{request} из модуля \code{clj-http.client}. Вызвав \code{in} несколько раз, вы окажетесь на нижнем уровне HTTP-запроса~--- в функции \code{request} из \code{clj-http.core}. С помощью \code{l} (locals) исследуйте локальные переменные, доступные в этой области. Последующие команды o (out) постепенно вернут вас на уровень \code{get-joke}.

Отладку сложно передать словами, и мы советуем читателю закрепить её практикой.

Заглянем в технические недра отладки. Тег \code{\#dbg} устроен как множество точек остановки. Если предварить тегом форму, он расставит в ней столько точек, сколько это возможно. Каждая точка знает свой уровень вложенности, что важно для навигации. За счёт этого можно пропустить точки текущего уровня, ступить ниже или выше.

Отладка работает в несколько этапов. Теги \code{\#break} и \code{\#dbg} помечают метаданные формы особым ключом. Убедимся в этом функцией \code{meta}:

\begin{english}
  \begin{clojure}
(-> "#break (+ 1 2)"
    read-string
    meta)

#:cider.nrepl.middleware.util.instrument
{:breakfunction
 #'c.n.m.d/breakpoint-with-initial-debug-bindings}
  \end{clojure}
\end{english}

Далее форма попадает в функцию \code{instrument-tagged-code}, которая оснащает её~--- внедряет код для взаимодействия с пользователем.

\begin{english}
  \begin{clojure}
(instrument-tagged-code
 (read-string "#break (+ 1 2)"))
  \end{clojure}
\end{english}

Результат для помеченной формы:

\begin{english}
  \begin{clojure}
(#'c.n.m.d/breakpoint-with-initial-debug-bindings
 (+ 1 2) {:coor []} (+ 1 2))
  \end{clojure}
\end{english}

Вместо \code{(+ 1 2)} получили вызов макроса \code{breakpoint-with-initial-debug-bindings} с тремя аргументами. Это форма вычисления, состояние отладчика и первичная форма. В нашем случае первый и третий параметры одинаковы, но на практике бывает обратное: вычисление формы отличается от её представления в коде.

Состояние отладчика изначально пустое. В поле \code{:coor} хранятся координаты формы, необходимые для переходов.

Проверим, что станет с функцией \code{add} при оснащении ее отладкой. Для этого вызовем \code{instrument-tagged-code} с формой, покрытой тегом \code{\#dbg}:

\begin{english}
  \begin{clojure}
(require
 '[cider.nrepl.middleware.util.instrument
    :refer [instrument-tagged-code]])

(instrument-tagged-code
 (read-string "#dbg (defn add [a b] (+ a b))"))
  \end{clojure}
\end{english}

Результат получился объёмный. Чтобы сократить его, заменим пространство \code{cider.nrepl.middleware.debug} на сочетание \code{c.n.m.d}:

\begin{english}
  \begin{clojure}
(#'c.n.m.d/breakpoint-with-initial-debug-bindings
 (def
  add
  (fn*
   ([a b]
    (#'c.n.m.d/breakpoint-if-interesting
     (+
      (#'c.n.m.d/breakpoint-if-interesting
       a {:coor [3 1]} a)
      (#'c.n.m.d/breakpoint-if-interesting
       b {:coor [3 2]} b))
     {:coor [3]}
     (+ a b)))))
 {:coor []}
 (defn add [a b] (+ a b)))
  \end{clojure}
\end{english}

\index{отладка!оснащение}

Обратите внимание, что форма \code{(defn ...)} превратилась в комбинацию \code{(def ...)} и \code{(fn* ...)}. Это называют раскрытием макросов. Раскрытие необходимо, чтобы расставить как можно больше точек останова. Третьим аргументом указана исходная форма \code{(defn ...)}. Она нужна, чтобы сопоставить отладку с исходным кодом в редакторе.

Расстановка точек останова учитывает формы, синтаксис которых нельзя нарушать. Например, в объявлении функции нетронуты её название и вектор аргументов. В форме \code{let} предваряются только правые элементы (значения) и так далее.

Макрос \code{breakpoint-if-interesting} называется так потому, что не каждая форма нуждается в точке останова. В следующем разделе мы коротко рассмотрим, какие из них не подлежат отладке.

Cider запоминает, какие функции оснащены отладкой. По команде \code{M-x cider-browse-instrumented-defs} вы увидите их список. Протоколы и типизированные записи тоже работают с отладкой, но не видны в этом списке.

\subsection{Ограничения}

\index{отладка!ограничения}

Чтобы пользоваться отладчиком эффективно, нужно знать его особенности. Прежде всего Cider не ставит точки останова перед литералами, которые выражаются в сами себя: числами, строками, кейвордами. Сравним отладку вектора с символами:

\begin{english}
  \begin{clojure}
(instrument-tagged-code
 (read-string "#dbg [a b c]"))

[(#'.../breakpoint-if-interesting a {:coor [0]} a)
 (#'.../breakpoint-if-interesting b {:coor [1]} b)
 (#'.../breakpoint-if-interesting c {:coor [2]} c)]
  \end{clojure}
\end{english}

\noindent
и литералами:

\begin{english}
  \begin{clojure}
(instrument-tagged-code
 (read-string "#dbg [1 \"hello\" :foobar]"))

[1 "hello" :foobar]
  \end{clojure}
\end{english}

В первом случае точки оказались перед каждым элементом, а во втором вектор остался нетронут.

Множества не поддаются отладке:

\begin{english}
  \begin{clojure}
(instrument-tagged-code
 (read-string "#dbg #{a b c d e}"))

;; #{a e c b d}
  \end{clojure}
\end{english}

Аналогично ведут себя словари длиннее восьми элементов.

\index{loop}

Трудности могут возникнуть с рекурсией (формы \code{loop} и \code{recur}). Компилятор требует, чтобы \code{recur} была строго в конце \code{loop}. При установке точек это правило нарушается и вместо \code{(recur (inc x))} образуется код:

\begin{english}
  \begin{clojure}
(#'c.n.m.d/breakpoint-if-interesting
 (recur (inc x))
 {:coor [3 2]}
 (recur (inc x)))
  \end{clojure}
\end{english}

Чтобы не вызвать ошибку компиляции, отладчик не ставит точку перед \code{recur}. В примере ниже тег \code{\#break} игнорируется:

\begin{english}
  \begin{clojure}
(loop [x 0]
  #break
  (when (< x 10)
    (println x)
    (recur (inc x))))
  \end{clojure}
\end{english}

\subsection{Итог}

Отладчик, не важно~--- встроенный или самодельный, крайне полезен в работе. С ним проще понять запутанный код и решить задачу вовремя. Однако отладчик требует времени и сил. Отложите его до лучших времен, если не полностью уверены в REPL и Emacs. На первых порах достаточно печати и инспекции.

\def\urlbogus{https://github.com/igrishaev/bogus}

Иные библиотеки предлагают свои отладчики. Коротко рассмотрим некоторые из них.

\index{библиотеки!Bogus}
\index{Bogus}

Библиотека \footurl{Bogus}{\urlbogus}[Bogus][-33mm] появилась по мере написания этой книги. Её тег \code{\#bogus} открывает окно Swing, где доступны локальные переменные и выполнение кода. Bogus можно считать минимальным графическим отладчиком для Clojure.

\def\urlscopecap{https://github.com/vvvvalvalval/scope-capture}
\def\urlscopecapnrepl{https://github.com/vvvvalvalval/scope-capture-nrepl}

\index{Scope capture}
\index{библиотеки!Scope capture}

Библиотека \footurl{Scope capture}{\urlscopecap}[Scope capture][-30mm] служит для работы с локальными переменными. Кроме просмотра и отладки, можно сохранить их в файл и позже воссоздать для участка кода. Scope capture сочетается с nREPL при помощи отдельной библиотеки \footurl{scope-capture-nrepl}{\urlscopecapnrepl}[Scope capture nREPL][-25mm].

\def\urlcljdebugger{https://github.com/razum2um/clj-debugger}

\index{библиотеки!Clj-debugger}
\index{Clj-debugger}

Ещё один отладчик для Clojure называется \footurl{clj-debugger}{\urlcljdebugger}[clj-de\-bug\-ger][-5mm]. Он предлагает REPL, в котором доступны локальные переменные, выполнение кода и другие возможности.

Даже уделив отладчику Cider столько времени, мы не покрыли его целиком. В числе прочего мы не коснулись профилировщика (profiler) и трассировщика (tracing). Первый служит для поиска медленного кода. Профилировщик оборачивает функции в макрос, который собирает метрики и выводит сводную таблицу. В~ней указано, сколько времени занял код в целом и функции по отдельности.

Трассировщик анализирует стек вызовов: порядок функций, их аргументы и промежуточные результаты. Трассировщик особенно полезен в цикле и рекурсии. Предоставим читателю самому разобраться с этими инструментами.

На этом мы закончим тему отладки и двинемся дальше: рассмотрим особые способы подключения к nREPL.

\section{nREPL в Docker}

\label{section-repl-docker}

\index{nREPL!Docker}
\index{Docker!nREPL}

Чтобы запустить проект на Clojure, устанавливают Java SDK, утилиты lein, Clojure CLI, maven и другие. Они написаны на Java и работают на всех платформах, поэтому с окружением редко бывают проблемы. Если всё-таки вы не можете что-то установить, остаётся запасной вариант~--- запустить проект в Docker.

Программу Docker мы упоминали в первой книге, поэтому не будем разбирать всё сначала. Запуск проекта в контейнере сводится к шагам:

\def\urldockerhubclj{https://hub.docker.com/\_/clojure}

\begin{itemize}

\item
  скачать образ с нужной версией Java, lein и прочими утилитами. \footurl{Репозиторий Clojure}{\urldockerhubclj}[Docker Hub Clojure] на Docker Hub предлагает более сотни образов с различными SDK, утилитами и их версиями;

\item
  при запуске образа смонтировать в него папку проекта и указать её как рабочую (work dir);

\item
  сопоставить локальный порт nREPL с портом в Docker. В~настройках nREPL явно указать порт;

\item
  запустить образ и подключиться к порту из Emacs.

\end{itemize}

Пройдя эти шаги, вы получите работающий nREPL, при этом в системе не останется следов установки Java и утилит.

Подготовим проект к запуску в Docker. Откройте файл \code{pro\-ject.clj} и добавьте профиль \code{:docker} с настройками ниже. Профиль нужен, чтобы не нарушить запуск проекта в обычном режиме.

\begin{english}
  \begin{clojure/lines}
:profiles
{:docker
 {:repl-options
   {:port 9911
    :host "0.0.0.0"}
  :plugins [[cider/cider-nrepl "0.28.3"]]}}
  \end{clojure/lines}
\end{english}

Обратите внимание, что мы указали хост и порт явным образом \coderefs{4 и 5}. Если бы хост был \code{127.0.0.1} или \code{localhost}, к нему нельзя было бы подключиться извне сети Docker. То же самое относится к порту: он должен быть известен заранее, чтобы объявить его в списке открытых (exposed) портов.

Запустите образ командой:

\begin{english}
  \begin{bash/lines}
docker run -it --rm \
  -p 9911:9911 \
  -v `pwd`:/project \
  -w /project \
  clojure \
  lein with-profile +docker repl
  \end{bash/lines}
\end{english}

Прокомментируем основные моменты. Аргумент \code{-p} \coderef{2} сопоставляет внутренний порт контейнера с портом операционной системы. Чтобы избежать путаницы, мы указали одинаковые порты. Позже мы рассмотрим случай, когда они отличаются.

Опция \code{-v} \coderef{3} сопоставляет пути локальной машины и контейнера. Мы смонтировали папку с проектом на путь \code{/project} в контейнере. Docker требует абсолютного пути, поэтому нельзя указать его точкой (например, \code{-v .:/project}). Выражение \code{pwd} в обратных кавычках выполняет команду в отдельном шелле и поставляет результат. В случае автора \code{pwd} вернул следующий путь:

\begin{english}
  \begin{text*}{breaklines, breakafter=/}
/Users/ivan/work/book-sessions/repl-chapter
  \end{text*}
\end{english}

Проект окажется в папке контейнера \code{/project}. Важно понимать разницу в монтировании и копировании файлов. В первом случае файлы остаются на локальной машине, а контейнер получает к ним доступ. Изменения с файлами, проделанные в контейнере, видны локальной системе, и наоборот. Если код в контейнере создает файлы, вы увидите их локально. Если скопировать файлы в контейнер, они будут жить отдельно от оригиналов, что помешает разработке.

Параметр \code{clojure} \coderef{5} означает имя образа. Он указан без тега, и по умолчанию будет использован тег \code{latest}. На момент написания книги \code{latest} включает в себя OpenJDK 17, Clojure 1.11.1 и lein 2.9.8. С выходом новых версий тег \code{latest} будет перезаписан. Изменится его контрольная сумма, что приведёт к повторному скачиванию. Чтобы этого избежать, задайте тег явно, например \code{clojure:openjdk-17-lein-2.9.6}.

Выражение \code{lein with-profile +docker repl} \coderef{6} означает команду, которую выполнит контейнер после запуска. По умолчанию она равна \code{lein run}, но мы указали \code{lein repl} с профилем \code{docker}, в котором особые настройки nREPL.

Очевидно, набрать команду \code{docker run} с учётом всех аргументов трудно. Запишите её в шелл-скрипт или добавьте цель в \code{Makefile}.

В папке проекта появится файл \code{.nrepl-port}. Он создан внутри контейнера, но из-за монтирования путей доступен снаружи. Перейдите в Emacs и откройте любой файл проекта. Подключитесь к nREPL командой

\begin{english}
  \begin{text}
M-x cider-connect RET 127.0.0.1 RET 9911 RET
  \end{text}
\end{english}

При подключении к nREPL сработает автодополнение: когда редактор запросит порт, нажмите TAB, и появится вариант с портом 9911.

Дальнейшие шаги аналогичны тем, что мы уже рассмотрели. Загрузите пространства командой \code{cider-ns-refresh}, выполните несколько функций или тестов.

\def\urlcgroups{https://en.wikipedia.org/wiki/Cgroups}

\index{Cgroups}

Когда проект запущен в Docker, вы заметите задержку на каждое действие. В системах, отличных от Linux, она будет ощутимой. Причина в том, что контейнер запускается не в \footurl{cgroups}{\urlcgroups}[cgroups][-10mm] (встроенной возможности Linux), а в виртуальной машине, накладные расходы на которую выше. За абстракцию приходится платить ресурсами.

Запуск проекта в Docker~--- тот случай, когда команда \code{cider-connect} необходима. Вы словно подключаетесь к удаленной машине, хоть она и запущена на том же компьютере.

Docker полезен для систем непрерывной интеграции (Continuous Integration, CI). С его помощью прогоняют тесты и собирают проект. Если ставить на каждой машине Java и утилиты, это займет время. Иногда нужны разные версии SDK, но их совместная установка ведёт к ошибкам, которые трудно расследовать. Docker сводит эти факторы на нет: достаточно сменить тег образа.

Итак, проект работает в Docker. В следующих шагах мы улучшим его настройки.

\index{Clojars}
\index{Maven}

\textbf{Зависимости.} Перед тем как включить nREPL, образ скачает зависимости из Clojars и Maven Central:

\begin{english}
  \begin{text}
Retrieving .../cider-nrepl-0.28.3.pom from clojars
Retrieving .../nrepl-0.9.0.pom from clojars
Retrieving .../clj-http-3.9.1.pom from clojars
...
  \end{text}
\end{english}

Однако при следующем запуске их загрузка начнется опять, что отнимает время и трафик. Хотелось бы, чтобы это был разовый шаг.

Причина в том, что по умолчанию зависимости оседают в папке контейнера \code{/root/.m2}. Поскольку контейнер не имеет состояния, при новом запуске папка окажется пуста, что вынудит lein скачать зависимости. Проблему решают двумя способами.

\textbf{Первый:} сопоставьте локальный путь Maven с папкой контейнера. Для этого передайте в команду \code{docker run} еще один параметр \code{-v}:

\begin{english}
  \begin{bash}
> docker run ... -v ~/.m2:/root/.m2
  \end{bash}
\end{english}

С ним зависимости, загруженные локально, будут видны в Docker, и наоборот.

\textbf{Второй:} добавьте в профиль \code{:docker} опцию \code{:local-repo}, которая меняет стандартный путь Maven.

\begin{english}
  \begin{clojure}
{:profiles
 {:docker {:local-repo ".docker/m2"}}}
  \end{clojure}
\end{english}

При запуске \code{docker run} будет создана папка \code{.docker/m2}, куда Maven скачает jar-файлы. Во второй раз проект подхватит их без новой загрузки. Добавьте путь \code{.docker/m2} в \code{.gitignore}, чтобы случайно не сделать их частью репозитория.

\textbf{Локальный профиль.} Выше мы упоминали файл \code{\tilde{}/.lein/\-pro\-fi\-les.clj}, который хранит локальные профили. Мы поместили в него зависимость \code{cider/cider-nrepl} и другие служебные библиотеки. Хотелось бы, чтобы Docker подхватил этот файл. Добавьте сопоставление путей:

\begin{english}
  \begin{bash}
docker run ... \
  -p ~/.lein/profiles.clj:/etc/leiningen/profiles.clj
  \end{bash}
\end{english}

Теперь плагин \code{cider/cider-nrepl} можно удалить из профиля \code{:docker}~--- он будет прочитан из файла \code{profiles.clj}. Утилита \code{lein} проверяет путь \code{/etc/leiningen/} на наличие профилей и загружает их.

\index{переменные среды}
\index{nREPL!порт}

Следующее улучшение~--- сделать так, чтобы \textbf{порт nREPL} можно было задать произвольно. На текущий момент порт <<захардкожен>>, что не совсем удобно. Исправим это в несколько этапов. Во-первых, укажем, что порт находится в переменной среды \code{NREPL\_PORT}:

\begin{english}
  \begin{clojure}
{...
 :repl-options
   {:port ~(some-> "NREPL_PORT"
                   System/getenv
                   Integer/parseInt)
    :host "0.0.0.0"}}
  \end{clojure}
\end{english}

Если она не задана, форма \code{(some-> ...)} вернет \code{nil}, и будет выбран случайный порт. Объявим в терминале переменную с желаемым портом:

\begin{english}
  \begin{bash}
export NREPL_PORT=9955
  \end{bash}
\end{english}

Доработаем команду \code{docker run}: в сопоставлении портов заменим значение на переменную. Доллар перед переменной вернет её значение. Кроме портов (параметр \code{-p}), переменную нужно передать параметром \code{-e}, чтобы сделать её доступной контейнеру.

\begin{english}
  \begin{bash}
docker run ... \
  -p $NREPL_PORT:$NREPL_PORT \
  -e NREPL_PORT=$NREPL_PORT ...
  \end{bash}
\end{english}

Подключитесь из редактора к порту 9955, и вы окажетесь в рабочем сеансе. По аналогии запустите контейнер с другим портом.

\def\urldockercompose{https://docs.docker.com/compose/}

Ещё один способ упростить работу с Docker~--- \textbf{составить конфигурацию} для \footurl{Docker Compose}{\urldockercompose}[Docker Compose]. Это программа, которая запускает контейнеры из файла на языке YAML. С ней не нужно запоминать все параметры \code{docker run}. Создайте файл \code{docker-compose.yaml}:

\index{Docker compose}
\index{YAML}

\begin{english}
  \begin{yaml}
version: '3.8'
services:
  nrepl:
    container_name: my_project
    image: clojure
    volumes:
      - .:/project
    ports:
      - $NREPL_PORT:$NREPL_PORT
    environment:
      NREPL_PORT: $NREPL_PORT
    working_dir: /project
    command:
      - "lein"
      - "with-profile"
      - "+docker"
      - "repl"
      - ":headless"
  \end{yaml}
\end{english}

Выполните команду ниже, после чего подключитесь к сеансу nREPL в Docker.

\begin{english}
  \begin{bash}
> docker-compose up
  \end{bash}
\end{english}

Обратите внимание на аргумент \code{:headless} в последней строке YAML. С ним nREPL запустится в <<безголовом>> режиме, когда ввода с клавиатуры нет, а обмен происходит только по сети. Если не добавить \code{:headless}, nREPL завершится с ошибкой, потому что стандартный канал ввода (stdin) будет недоступен.

Docker крайне полезен для разработки, прогона тестов и сборки проекта. С ним окружение запускают одной командой, а не настраивают вручную. С Docker легко добиться, чтобы окружение вело себя одинаково на разных машинах, в том числе в CI. И хотя у него есть недостатки~--- медленная работа на отличных от Linux системах, значимое потребление ресурсов,~--- на долгой дистанции Docker оправдывает себя.

\section{nREPL в боевом режиме}

\index{nREPL!на сервере}

Необычный вопрос: как подключиться к проекту, который уже развернут и обслуживает пользователей? Тема противоречива: с~одной стороны, приём опасен и считается плохой практикой. С~другой стороны, об этом часто спрашивают новички, и уходить от вопроса неправильно.

Когда проект запущен локально при помощи \code{lein repl}, к нему подключаются как обычно. Но если собрать и запустить jar-файл, в нем не будет сервера nREPL. Так происходит потому, что nREPL запускается силами \code{lein}. Считайте его вспомогательным средством, доступным только в разработке.

Когда на сервере случается ошибка, у новичка возникает шальная мысль: если бы можно было подключиться, я бы все исправил. И хотя это возможно технически, предостережем читателя от такого подхода.

Ошибки на сервере говорят о том, что код недостаточно покрыт тестами. Если код завершился аварийно, повторите ситуацию локально и добавьте тест. Чаще всего причина в том, что из источников приходят не те данные, что вы ожидали. Например, сервис \code{jokeapi.dev} изменил структуру ответа, и вместо полей мы получаем nil. Добавьте отладочный лог, как в примере ниже \coderef{11}:

\begin{english}
  \begin{clojure/lines}
(defn get-joke [lang]
  (let [request
        {...}

        response
        (client/request request)

        {:keys [body]}
        response

        _ (log/debugf "Data from jokeapi.dev: %s" body)

        {:keys [setup delivery]}
        body]
    (format "%s %s" setup delivery)))
  \end{clojure/lines}
\end{english}

Измените настройки так, чтобы логи с уровнем \code{debug} записывались в файл. Перезапустите проект. Спровоцируйте вызов \code{get-joke}, и вы узнаете, что пришло от сервиса. Исправьте код под новые данные, соберите uberjar и загрузите на сервер.

Если возникло исключение, соберите как можно больше данных о нем. Исключение~--- сложный объект: иногда одно исключение ссылается на второе, то~--- на третье и так далее. В~каждом звене нас интересует его класс, сообщение и данные \code{ex-info}. В~первой книге мы рассмотрели, как всё это собрать и куда передать.

Логирование, тесты и сбор ошибок полезнее отладки на боевом сервере. За годы работы автор ни разу не подключался к нему по nREPL~--- в этом не было нужды. Современные практики~--- тесты, логи, CI, деплой~--- в корне противоречат горячей перезагрузке кода~--- приёму, когда приложение обновляют в обход стандартных процедур.

\def\urlnasalisp{https://thenewstack.io/nasa-programmer-remembers-debugging-lisp-in-deep-space}

\index{NASA}

Возможно, читатель слышал романтические истории о программистах, которые устраняли катастрофы в другой части света. Например, инженеры NASA исправили код космического корабля, когда он летел рядом с \footurl{Юпитером}{\urlnasalisp}[Lisp in Deep Space][-10mm]. Это исключительный случай, и вряд ли в NASA были рады инциденту. Гораздо лучше найти ошибку до запуска проекта.

Автор работал в проекте, где на боевом сервере был запущен даже не один, а два nREPL~--- по одному на каждую команду разработчиков. Стоит ли говорить, что эта практика вышла из негативных предпосылок. В проекте почти не было тестов, никто не настраивал локальное окружение. Логи писались в файлы, разбросанные по разным машинам, которые никто не читал. При этом в фирме верили, что подключение к nREPL решит все эти проблемы.

С внедрением практик, что мы рассмотрели,~--- тесты, окружение, централизованный сбор ошибок~--- стало возможным повторить любую ситуацию на машине разработчика. Это быстрее и безопаснее, чем подключение к удалённой машине. В команде автора больше не использовали сервер как площадку для экспериментов, и хочется верить, что это до сих пор так.

Для своего времени подключение к программе, запущенной где-то далеко, было революционным. Это выгодно отличало Лисп от компилируемых языков, которые без особых ухищрений не позволяли внедряться в их программы. Но по сегодняшним меркам у этого подхода важные недостатки: он ситуативен и не поддается автоматизации. Ошибки нужно фиксировать тестами, чтобы в будущем кто-то другой не расследовал их опять. Если у вас больше одной машины, горячую перезагрузку кода нужно выполнить на каждой из них. Тесты и окружение хорошо поддаются автоматизации, и поэтому их преимущество бесспорно.

Текст ниже предполагает, что автор либо не убедил читателя, либо не знает о случаях, когда nREPL на сервере необходим. Если это так, напишите автору письмо с описанием ситуации, и она займёт место в книге.

Итак, чтобы nREPL стал частью приложения, выполните следующее. Поместите библиотеку \code{nrepl/nrepl} в основные зависимости:

\begin{english}
  \begin{clojure}
  :dependencies
  [[org.clojure/clojure "1.10.1"]
   [nrepl/nrepl "0.9.0"]]
  \end{clojure}
\end{english}

Добавьте модуль, который запускает сервер nREPL. В примере ниже функция \code{nrepl-start!} включает сервер, а \code{nrepl-stop!} выключает его. Глобальная переменная \code{server} хранит объект сервера.

\begin{english}
  \begin{clojure}
(ns nrepl-prod.core
  (:gen-class)
  (:require
   [nrepl.server :refer [start-server stop-server]]))

(defonce server nil)

(defn nrepl-start! []
  (alter-var-root
   #'server
   (constantly
    (start-server :bind "0.0.0.0" :port 9911))))

(defn nrepl-stop! []
  (alter-var-root #'server stop-server))

(defn -main
  [& _]
  (nrepl-start!)
  (println "The nREPL server has been started"))
  \end{clojure}
\end{english}

Соберите проект командой \code{lein uberjar}. Готовый jar-файл находится в папке \code{target/uberjar}, если не задано иное опцией \code{:target-path}.

Далее понадобится удаленная машина с доступом по SSH. Для краткости опустим её первичную настройку: создание пользователя, sudo, SSH-ключи и прочее. Считаем, что вы достигли этапа, когда команда \code{ssh <IP>} открывает сеанс bash на удалённой машине. При этом учетная запись отличается от \code{root}, \code{ubuntu} и прочих системных.

\index{lein!uberjar}
\index{uberjar}

Установите виртуальную машину Java командами:

\begin{english}
  \begin{bash}
> sudo apt update
> sudo apt install default-jre
  \end{bash}
\end{english}

Проверьте установку:

\begin{english}
  \begin{bash}
> java -version
openjdk version "11.0.15" 2022-04-19
  \end{bash}
\end{english}

Загрузите файл на сервер с локального компьютера:

\begin{english}
  \begin{bash}
> scp target/uberjar/nrepl_prod-0.1.0-standalone.jar \
      <IP>:/home/<user>/
  \end{bash}
\end{english}

Подключитесь к машине по SSH. Перейдите в домашнюю папку пользователя и выполните:

\begin{english}
  \begin{bash}
> java -jar nrepl_prod-0.1.0-standalone.jar
  \end{bash}
\end{english}

Появится сообщение, что сервер nREPL запущен. Откройте Emacs и подключитесь к nREPL:

\begin{english}
  \begin{text}
M-x cider-connect <RET> <IP> <RET> 9911 <RET>
  \end{text}
\end{english}

Откроется сеанс nREPL на удалённой машине. Выполните выражение:

\begin{english}
  \begin{clojure}
(.println System/out "hello")
  \end{clojure}
\end{english}

В терминале с SSH, где запущен jar-файл, появится <<hello>>. Проверьте переменную \code{server}:

\begin{english}
  \begin{clojure}
(in-ns 'nrepl-prod.core)
server

;; #n.s.Server{addr=0.0.0.0, localport=9911, ...}
  \end{clojure}
\end{english}


С помощью функции \code{sh} выполните системную команду. Это может быть чтение каталога, удаление файлов, сбор информации о системе и многое другое.

\begin{english}
  \begin{clojure}
(require '[clojure.java.shell :refer [sh]])
  \end{clojure}
\end{english}

\index{утилиты!uname}
\index{uname}

Для начала запустите \code{uname}~--- утилиту, которая выводит данные об операционной системе:

\begin{english}
  \begin{clojure}
=> (:out (sh "uname" "-a"))

;; Linux 5-63-153-107 x86_64 GNU/Linux
  \end{clojure}
\end{english}

Прочитаем корневой каталог командой \code{ls}:

\begin{english}
  \begin{text}
(println (:out (sh "ls" "-l" "/")))

;; lrwxrwxrwx   1    7 Jun  9 01:11 bin -> usr/bin
;; drwxr-xr-x   3 4096 Jun  9 01:18 boot
;; ...
  \end{text}
\end{english}

После экспериментов с \code{sh} завершите процесс Java. Выполните \code{(nrepl-stop!)} в пространстве \code{nrepl-prod.core}, и сервер остановится. При этом не останется потоков, которых ожидает главный поток JVM, и процесс завершится.

Итак, подключение к удаленному nREPL прошло успешно. Отложив вопросы безопасности до следующего раздела, рассмотрим, что можно улучшить с технической точки зрения.

На текущий момент наш nREPL <<голый>>, то есть не оснащенный возможностями Cider. Должно быть, при подключении в Emacs вы видели строку:

\begin{english}
  \begin{text}
CIDER requires cider-nrepl to be fully functional.
Some features will not be available without it!
  \end{text}
\end{english}

Из-за этого в Cider доступны только базовые команды вроде \code{eval} и \code{lookup}. Чтобы это исправить, добавьте в зависимости библиотеку \code{cider/cider-nrepl}:

\begin{english}
  \begin{clojure}
  :dependencies
  [[org.clojure/clojure "1.10.1"]
   [nrepl/nrepl "0.9.0"]
   [cider/cider-nrepl "0.28.3"]]
  \end{clojure}
\end{english}

Импортируйте её в главный модуль:

\begin{english}
  \begin{clojure}
(ns nrepl-prod.core
  (:gen-class)
  (:require [cider.nrepl] ...))
  \end{clojure}
\end{english}

В функцию \code{start-server} передайте обработчик \code{cider-nrepl-handler}. Это обычный обработчик nREPL, <<заряженный>> встроенными в Cider middleware:

\begin{english}
  \begin{clojure}
(start-server :bind "0.0.0.0" :port 9911
              :handler cider.nrepl/cider-nrepl-handler)
  \end{clojure}
\end{english}

Соберите проект и загрузите на сервер. При запуске вы получите доступ ко всем возможностям Cider, что мы рассмотрели: переходу к определениям, тестам, отладке и остальному. Однако все эти средства пригодны только для разработки; на боевом сервере они принесут больше вреда, чем пользы.

\subsection{nREPL в системе}

Способ с \code{alter-var-root}, которым мы запускаем nREPL в примере выше, оставляет желать лучшего. Это неуклюжее решение, пригодное только для демонстрации. В реальных проектах избегают глобального состояния. Объекты с семантикой <<включить и выключить>> оборачивают в компоненты, а управляет ими система.

\def\urlcomponent{https://github.com/stuartsierra/component}

\index{Component}
\index{библиотеки!Component}

Мы подробно рассмотрели системы в первой книге, поэтому не будем начинать все с нуля. Покажем, как выразить сервер nREPL компонентом. В качестве системы выберем библиотеку \footurl{Component}{\urlcomponent}[Com\-po\-nent][-5mm]. Добавьте её в зависимости:

\begin{english}
  \begin{clojure}
[com.stuartsierra/component "0.4.0"]
  \end{clojure}
\end{english}

Код компонента:

\iflarge

\begin{english}
  \begin{clojure/lines}
(defrecord nREPLServer
    [options
     server]

  component/Lifecycle

  (start [this]
    (let [options
          (update options :handler #(-> % resolve deref))

          arg-list
          (mapcat identity options)

          server
          (apply start-server arg-list)]

      (assoc this :server server)))
  \end{clojure/lines}
\end{english}

\pagebreak[4]

\begin{english}
  \begin{clojure/lines*}{firstnumber=18}
  (stop [this]
    (when server
      (stop-server server))
    (assoc this :server nil)))
  \end{clojure/lines*}
\end{english}

\else

\begin{english}
  \begin{clojure/lines}
(defrecord nREPLServer
    [options
     server]

  component/Lifecycle

  (start [this]
    (let [options
          (update options :handler #(-> % resolve deref))

          arg-list
          (mapcat identity options)

          server
          (apply start-server arg-list)]

      (assoc this :server server)))

  (stop [this]
    (when server
      (stop-server server))
    (assoc this :server nil)))
  \end{clojure/lines}
\end{english}

\fi

Его конструктор и пример вызова:

\begin{english}
  \begin{clojure}
(defn make-nrepl-server [options]
  (map->nREPLServer {:options options}))

(make-nrepl-server
 {:port 9911
  :handler 'cider.nrepl/cider-nrepl-handler})
  \end{clojure}
\end{english}

Комментария заслуживает вызов \code{(update ...)}, где поле \code{:han\-d\-ler} превращается в функцию комбинацией \code{resolve} и \code{deref} \coderef{9}. Это нужно затем, что \code{:handler} содержит символ, и нужно привести его к функции. Почему бы не передать сразу функцию? Дело в том, что конфигурацию часто хранят в *.edn-файле, в котором нельзя сослаться на объект Clojure. При чтении файла в словаре окажется символ; далее компонент приведёт его к функции.

\begin{english}
  \begin{clojure}
;; config.edn
{:handler cider.nrepl/cider-nrepl-handler}
  \end{clojure}
\end{english}

В идеале компонент должен поддерживать как символ, так и функцию в поле \code{:handler}. Доработайте код, чтобы это требование выполнялось.

Выражение \code{mapcat} \coderef{12} превращает словарь в плоский список, где чередуются ключи и значения:

\begin{english}
  \begin{clojure}
(mapcat identity {:foo 1 :bar 2 :baz 3})

(:foo 1 :bar 2 :baz 3)
  \end{clojure}
\end{english}

Далее его передают в \code{start-server} при помощи \code{apply}. Это вызвано тем, что \code{start-server} принимает остаточные аргументы, а не словарь:

\begin{english}
  \begin{clojure}
(start-server :port 9911 :host "..." :handler ...)
  \end{clojure}
\end{english}

\def\urlmaparg{https://clojure.org/news/2021/03/18/apis-serving-people-and-programs}

\index{аргументы}

Опытный читатель заметит, что в последних версиях Clojure проблема решена: можно передать словарь в функцию \code{start-server}, и он \footurl{преобразуется в список}{\urlmaparg}[Map ar\-gu\-ments]. Но чтобы код не зависел от версии Clojure, проделаем то же самое явно.

Приведем код запуска минимальной системы. Вынесем конфигурацию в файл \code{resources/config.edn}:

\begin{english}
  \begin{clojure}
;; config.edn
{:nrepl {:bind "0.0.0.0"
         :port 9911
         :handler cider.nrepl/cider-nrepl-handler}}
  \end{clojure}
\end{english}

Прочитаем его и построим систему:

\begin{english}
  \begin{clojure}
(def system-config
  (-> "config.edn"
      io/resource
      slurp
      edn/read-string))

(def system-init
  (component/system-map
   :nrepl (make-nrepl-server (:nrepl system-config))))
  \end{clojure}
\end{english}

Запустите систему в функции \code{-main}, и nREPL готов к подключению:

\begin{english}
  \begin{clojure}
(defn -main
  [& _]
  (let [system-started
        (component/start system-init)]
    (println "The nREPL server has been started")))
  \end{clojure}
\end{english}

В работе с компонентами проступает важное свойство: все задано конфигурацией. Если понадобится другой порт, измените настройки и перезагрузите сервис, не меняя кода. Тонкости конфигурации мы рассмотрели в первой книге.

Выше мы использовали библиотеку Component, однако это не ограничивает ваш выбор. Компонент nREPL легко перенести в Mount или Integrant. В качестве упражнения перепишите код из этого раздела под ту систему, что удобна вам.

\subsection{Безопасность}

\index{сканер портов}
\index{безопасность}

До сих пор мы откладывали вопрос безопасности, и пора это исправить. Выше мы подключаемся к серверу так, словно он доступен всем желающим. Это настоящая катастрофа, поскольку злоумышленник может выполнить произвольный код: обратиться к базе, файлам и системным утилитам.

\def\urlportscan{https://en.wikipedia.org/wiki/Port\_scanner}

Даже если вы никому не сказали, что на сервере запущен nREPL, это легко обнаружить. Существуют \footurl{сканеры портов}{\urlportscan}[Port scanners]~--- программы, которые перебирают порты веб-сервисов (8080, 8888), баз данных (5432, 3306) и других служб на удалённой машине. Продвинутые сканеры определяют программу за тем или иным портом, посылая различные сообщения.

nREPL не предлагает проверки доступа по логину и паролю. И хотя её легко написать (это будет лишнее middleware в стеке), будет правильно защитить nREPL другим способом~--- сетевыми настройками. Ниже мы рассмотрим два подхода: iptables и SSH-туннель.

\def\urliptables{https://en.wikipedia.org/wiki/Iptables}

\index{iptables}
\index{утилиты!iptables}

Программа \footurl{iptables}{\urliptables}[iptables] задает правила обмена трафиком в Unix-подобных системах. Сделаем так, чтобы к nREPL можно было подключиться только с определенного IP (или диапазона), например из офиса. Пусть порт nREPL задан 9911, а внешний IP офиса~--- 179.211.55.130. Первое правило запрещает доступ к порту 9911 с любого IP:

\begin{english}
  \begin{bash}
> sudo iptables -A INPUT \
       -p tcp \
       -s 0.0.0.0/0 \
       --dport 9911 -j DROP
  \end{bash}
\end{english}

Второе правило в порядке исключения открывает доступ с адреса 179.211.55.130:

\begin{english}
  \begin{bash}
> sudo iptables -A INPUT \
       -p tcp \
       -s 179.211.55.130 \
       --dport 9911 -j ACCEPT
  \end{bash}
\end{english}

Введите их на удалённой машине. После этого запустите проект на сервере командой \code{java -jar ...} и подключитесь к nREPL. Если ваш IP совпадает с тем, что задан в правилах, подключение пройдёт без ошибок. Сделайте то же самое с другим IP: включите VPN или раздайте интернет с телефона. В этом случае подключение не состоится.

Правила iptables действуют до перезагрузки операционной системы, и в следующий раз придётся ввести их снова. Чтобы этого избежать, воспользуйтесь утилитой \code{iptables-persistent}. Установите её командой

\begin{english}
  \begin{bash}
> sudo apt install iptables-persistent
  \end{bash}
\end{english}

\index{утилиты!iptables-persistent}
\index{iptables-persistent}

Введите правила и выполните команду ниже. После перезагрузки утилита восстановит их.

\begin{english}
  \begin{bash}
> sudo netfilter-persistent save
  \end{bash}
\end{english}

Подключение по SSH-туннелю безопаснее и поэтому предпочтительнее. На удалённой машине порт nREPL доступен только для локального подключения (с 127.0.0.1 или localhost). Утилита ssh устанавливает шифрованный туннель между локальной и удалённой машинами. Каждому концу туннеля назначается порт. Покажем это на схеме:

\begin{figure}[H]
  \centering
  \includesvg{charts/repl02.svg}
  \label{fig:chart-repl-02}
\end{figure}

Трафик локальной машины, переданный в порт 19911, поступит на порт удалённой машины 9911 и обратно. С точки зрения обеих машин это будут локальные подключения. Все вопросы безопасности~--- доступ, шифрование, отзыв ключа~--- берёт на себя SSH.

Откройте конфигурацию проекта. Укажите, что сервер nREPL прослушивает только локальный хост:

\begin{english}
  \begin{clojure}
;; config.edn
{:nrepl
  {:bind "127.0.0.1"
   :port 9911
   :handler cider.nrepl/cider-nrepl-handler}}
  \end{clojure}
\end{english}

Скомпилируйте jar-файл, перенесите на сервер и запустите командой \code{java -jar ...}. Откройте новую вкладку терминала и выполните:

\begin{english}
  \begin{bash}
> ssh -N -L 19911:127.0.0.1:9911 <IP>
  \end{bash}
\end{english}

Разберём параметры этой команды:

\begin{itemize}

\item
  \code{-L}~--- установить туннель (link);

\item
  19911~--- порт локальной машины;

\item
  9911~--- порт удалённой машины;

\item
  \code{-N}~--- не принимать ввода с клавиатуры, а просто ждать;

\item
  \code{<IP>}~--- адрес удалённой машины.

\end{itemize}

\index{SSH!туннель}
\index{туннель!SSH}

Команда запустит процесс \code{ssh} без ввода с клавиатуры. Туннель работает до тех пор, пока вы не нажмёте \code{Ctrl+C}. Перейдите в Emacs и подключитесь к порту 19911:

\begin{english}
  \begin{clojure}
M-x cider-connect <RET> 127.0.0.1 <RET> 19911 <RET>
  \end{clojure}
\end{english}

\index{nREPL!порт}

Поскольку система не знает о туннеле, в проекте не будет файла \code{.nrep-port}. Автодополнение не сработает, и порт придется ввести вручную. Это легко исправить, создав файл командой

\begin{english}
  \begin{bash}
> echo 19911 > .nrepl-port
  \end{bash}
\end{english}

После лёгкой задержки вы подключитесь к удалённому nREPL, словно он запущен локально. Этот способ лучше прямого подключения, потому что SSH берёт на себя вопросы безопасности. Например, если сотрудник покинул фирму, удалите его ключ на сервере, и он не сможет подключиться.

Приемы, что мы рассмотрели,~--- iptables и SSH-туннель~--- справедливы для запуска на чистом Linux, то есть без виртуализации и контейнеров. Если вы пользуетесь решениями вроде Kubernetes или Elastic Container Service, настройки будут другими. Конфигурация этих программ выходит за рамки главы.

Перед тем как закончить раздел, напомним читателю о спорной природе nREPL на сервере. Идите на этот шаг, только если полностью понимаете риск.

\section{REPL в других средах}

До сих пор мы изучали REPL, который выполняет код в виртуальной машине Java (JVM). Это популярный, но не единственный вариант. Теперь мы рассмотрим схему, где вычисления берет на себя платформа JavaScript. Этот способ сложнее, он задействует больше технологий и протоколов; в то же время он открывает новые возможности.

Язык JavaScript недаром называют платформой. В широком смысле это набор стандартов и реализаций под разные устройства. JavaScript работает в браузерах, телефонах, одноплатных компьютерах. Были попытки сделать на нем операционную систему для мобильных устройств. Популярность JavaScript не нужно доказывать~--- он повсюду.

\index{CoffeeScript}
\index{языки!CoffeeScript}

\def\urlcoffeescript{https://coffeescript.org}

JavaScript стал платформой для программ, написанных на других языках. Пионером в этой области был \footurl{CoffeeScript}{\urlcoffeescript}[Coffee\-Script][-10mm]~--- легковесный язык, который сглаживал острые углы прародителя: сравнение, проверку на null и undefined, классы и наследование. Позже появились Elm и PureScript, вдохновлённые строгостью Haskell. В последние годы популярен TypeScript~--- статически типизированный язык Microsoft.

\def\urlclojurescript{https://clojurescript.org}

\def\urlgooglecc{https://developers.google.com/closure/compiler}

\index{ClojureScript}
\index{языки!ClojureScript}

Экосистема Clojure предлагает \footurl{ClojureScript}{\urlclojurescript}[Clojure\-Script][-15mm]~--- компилятор языка, близкого к Clojure, в JavaScript. Мы не случайно написали <<близкого>>: хотя Clojure и ClojureScript похожи, между ними есть отличия, которые выступают в работе. ClojureScript опирается на компилятор Google, названный \footurl{Closure Compiler}{\urlgooglecc}[Closure Compiler][-10mm]. Обратите внимание на разницу в написании: Clo\textbf{s}ure не имеет отношения к Clo\textbf{j}ure.

\index{Closure Compiler}
\index{Google}

Особенность компилятора~--- в том, что при должных настройках он эффективно сжимает код и удаляет его неиспользуемые части. В Closure Compiler приняты пространства имен, близкие к Clojure, что облегчает интеграцию проектов.

Важно помнить, что ClojureScript~--- только компилятор, но не интерпретатор. Это вводит новичков в заблуждение: загрузив ClojureScript, они ищут исполняемый файл, который бы выполнил код на Clojure. На самом деле ClojureScript только переводит код на Clojure в JavaScript. Рассмотрим простой модуль \code{foo} с функцией \code{add}:

\begin{english}
  \begin{clojure}
(ns foo)

(defn add [a b]
  (+ a b))
  \end{clojure}
\end{english}

После обработки его библиотекой ClojureScript получим код на JavaScript:

\begin{english}
  \begin{javascript}
// Compiled by ClojureScript 1.10.758 {}
goog.provide('foo');
goog.require('cljs.core');
foo.add = (function foo$add(a,b){
return (a + b);
});
  \end{javascript}
\end{english}

Как именно его выполнить~--- в какой среде и настройками,~--- остаётся на ваше усмотрение.

Мы подводим читателя к понятию рантайма~--- среды исполнения JavaScript. Недостаточно написать код на этом языке: необходима среда, в которой он запустится. Ниже мы рассмотрим две популярные среды: браузер и движок Node.js. Мы увидим разницу между ними: проделаем в каждой то, что невозможно в другой.

\index{JavaScript}
\index{языки!JavaScript}

Покажем отличие в вычислениях между Clojure и ClojureScript. В первом случае код выполняется в JVM:

\begin{figure}[H]
  \centering
  \includesvg{charts/repl03.svg}
  \label{fig:chart-repl-03}
\end{figure}

Для ClojureScript платформа JVM выступает в роли посредника. Её задача~--- обеспечить связь между REPL и JavaScript. Связь работает через какой-то транспорт. Это может быть REST API, веб-сокет, TCP-соединение или что-то другое.

\begin{figure}[H]
  \centering
  \includesvg{charts/repl04.svg}
  \label{fig:chart-repl-04}
\end{figure}

Рассмотрим, как собрать эту схему на практике. Для начала понадобится ClojureScript. Это обычная библиотека на Java, поэтому добавим её в зависимости. Также сделаем уточнение: до сих пор мы в основном пользовались lein. Чтобы не формировать у читателя предвзятость, проведем эксперименты в Clojure CLI.

Создайте файл \code{deps.edn} следующего содержания:

\begin{english}
  \begin{clojure}
{:deps
 {org.clojure/clojurescript {:mvn/version "1.10.758"}}}
  \end{clojure}
\end{english}

Запустите REPL командой \code{clj}. Когда появится приглашение, введите код:

\begin{english}
  \begin{clojure}
(require '[cljs.repl :as repl])
(require '[cljs.repl.node :as node])

(def env (node/repl-env))
(repl/repl env)
  \end{clojure}
\end{english}

Пока мы не ушли дальше, разберемся, что происходит. Пространства имен ClojureScript начинаются с \code{cljs}, чтобы не было путаницы с Clojure. Функция \code{repl} из последней строки запускает внутренний REPL для ClojureScript. Она принимает обязательный аргумент~--- окружение, которое мы создали вызовом \code{(node/repl-env)}.

Окружение отвечает за взаимодействие с платформой JavaScript. В техническом плане это объект, реализующий протокол \code{IJavaScriptEnv} с методами \code{-setup}, \code{-evaluate} и другими. Выше мы создали окружение для движка Node.js. Оно ищет программу \code{node}, установленную локально, и запускает её. Процесс Node.js легко обнаружить командой

\begin{english}
  \begin{bash}
> ps aux | grep node
  \end{bash}
\end{english}

Как только процесс запущен, REPL готов принять команду. В~терминале появятся версия ClojureScript и приглашение. Введите \code{(+ 1 2)}, чтобы убедиться, что схема работает.

\begin{english}
  \begin{text}
ClojureScript 1.10.758

cljs.user=>(+ 1 2)
3
  \end{text}
\end{english}

\index{Node.js}

Исследуйте \code{js/process}~--- центральный объект Node.js с информацией о системе, среде и многим другим. Если обратиться к полю \code{js/process.env}, получим загадочный вывод:

\begin{english}
  \begin{clojure}
=> js/process.env
#object[Object [object Object]]
  \end{clojure}
\end{english}

Это обычное представление объекта JavaScript при печати. Чтобы сделать вывод понятнее, приведем объект к словарю Clojure. Напишем функцию \code{environment}, которая вернёт словарь, где ключи~--- кейворды, а значения~--- переменные среды. Вот что получилось у автора:

\begin{english}
  \begin{clojure/lines}
(defn environment []
  (persistent!
   (reduce
    (fn [result var-name]
      (assoc! result
              (keyword var-name)
              (aget js/process.env var-name)))
    (transient {})
    (js-keys js/process.env))))
  \end{clojure/lines}
\end{english}

Наберите эту функцию в редакторе и скопируйте в REPL, после чего обратитесь к ней:

\begin{english}
  \begin{clojure}
(environment)

;; {:HOME "/Users/ivan", :USER "ivan", ...}
  \end{clojure}
\end{english}

Обратите внимание на функцию \code{js-keys}, доступную только в ClojureScript \coderef{9}. Функция возвращает ключи JavaScript-объекта; префикс \code{js-} означает, что аргумент~--- значение JavaScript, например объект или массив.

\index{браузер}

Когда эксперименты с Node.js закончены, перейдем к браузеру. Запуск браузера отличается только окружением. Подключите модуль \code{cljs.repl.browser} и выполните код:

\begin{english}
  \begin{clojure}
(require '[cljs.repl.browser :as browser])

(def env (browser/repl-env))

(repl/repl env)
  \end{clojure}
\end{english}

Откроется браузер, назначенный в системе по умолчанию. Адрес страницы будет \code{http://localhost:9000} (порт и другие параметры задают в окружении). Вы увидите логотип ClojureScript и краткую справку. Вернитесь в терминал, где запущен REPL, и введите:

\begin{english}
  \begin{clojure}
(js/alert "Hello REPL!")
  \end{clojure}
\end{english}

Перейдите в браузер и проверьте, что появилось модальное окно с приветствием. Исследуйте объект \code{localStorage}: установите значение по ключу и прочитайте его:

\begin{english}
  \begin{clojure}
(.setItem js/window.localStorage "key-1" "val-1")
(.getItem js/window.localStorage "key-1")

;; "val-1"
  \end{clojure}
\end{english}

Измените заголовок страницы. Для этого присвойте свойству \code{document.title} произвольную строку. Убедитесь, что название вкладки в браузере поменялось:

\begin{english}
  \begin{clojure}
(set! js/window.document.title "New Title")
  \end{clojure}
\end{english}

Разберёмся, как связаны между собой браузер и REPL. Откройте консоль разработчика и перейдите на закладку Network. Видно, что браузер посылает POST-запрос на порт 9000. Эту технику называют long polling (долгий опрос), потому что запрос длится до тех пор, пока в REPL не введут выражение.

Как только REPL получил его, он компилирует код на ClojureScript и отправляет браузеру. Для ввода \code{(+ 1 2)} браузер получит код ниже. На первый взгляд он смотрится жутко: в нем лишние скобки и машинные имена, потому что его произвела программа. Разберемся, что здесь происходит.

\begin{english}
  \begin{json/lines}
{"repl": "main",
 "form": "(function () {
  try {
    return cljs.core.pr_str.call(
      null,
      (function () {
        var ret__6698__auto__ = ((1) + (2));
        (cljs.core._STAR_3 = cljs.core._STAR_2);
        (cljs.core._STAR_2 = cljs.core._STAR_1);
        (cljs.core._STAR_1 = ret__6698__auto__);
        return ret__6698__auto__;
    })());
  } catch (e617) {
    var e__6699__auto__ = e617;
    (cljs.core._STAR_e = e__6699__auto__);
    throw e__6699__auto__;
  }
})()"}
  \end{json/lines}
\end{english}

Выражение \code{(+ 1 2)} стало \code{((1) + (2))} \coderef{7}; оно обернуто в анонимную функцию без параметров. Длинное выражение \code{cljs.core.pr\_str.call} означает функцию \code{pr-str} в Clojure, которая приводит объект к строке.

Переменные \code{\_STAR\_} с номером на конце~--- это \code{*1}, \code{*2} и \code{*3} для хранения последних результатов. Код обернут в \code{try/catch}, чтобы в случае ошибки назначить переменной \code{*e} (после компиляции~--- \code{\_STAR\_e}) последнее исключение \coderef{15}.

Полученный код выполняется в браузере обычным \code{eval}. На сервер уходит отчет о вычислении:

\pagebreaklarge

\begin{english}
  \begin{clojure}
{:repl "main",
 :type :result,
 :content "{:status :success, :value \"3\"}",
 :order 3}
  \end{clojure}
\end{english}

В зависимости от статуса REPL покажет результат или сведения об ошибке.

Запустить REPL в Node.js или браузере можно и без кода. Для этого служат параметры командной строки:

\begin{english}
  \begin{bash}
clj -M -m cljs.main --repl-env node
clj -M -m cljs.main --repl-env browser
  \end{bash}
\end{english}

В обоих случаях вы получите REPL в нужном окружении.

\subsection{Поддержка Cider}

\index{Cider!ClojureScript}
\index{ClojureScript!Cider}
\index{Piggieback}
\index{библиотеки!Piggieback}

\def\urlpiggieback{https://github.com/nrepl/piggieback}

До сих пор мы вводили команды в терминале, что непривычно после Emacs и Cider. Наверняка вам захочется связать ClojureScript с редактором. Для этого служит библиотека с забавным названием \footurl{Piggieback}{\urlpiggieback}[Piggieback], что означает <<нести на закорках>>.

Piggieback служит мостом между nREPL и JavaScript. В техническом плане это middleware, которое передаёт сообщения от Cider к JavaScript и обратно. С Piggieback схема усложняется ещё больше и выглядит так:

\begin{figure}[H]
  \centering
  \resizebox{\columnwidth}{!}{\includesvg{charts/repl05.svg}}
  \label{fig:chart-repl-05}
\end{figure}

Чтобы подключиться к ClojureScript из редактора, задайте профиль \code{:nrepl/piggieback} в файле \code{deps.edn}. Он добавляет \code{wrap-cljs-repl} в стек middleware:

\begin{english}
  \begin{clojure*}{fontsize=\small}
{:aliases
 {:nrepl/piggieback
  {:extra-deps
   {nrepl/nrepl {:mvn/version "0.8.3"}
    cider/piggieback {:mvn/version "0.5.3"}
    cider/cider-nrepl {:mvn/version "0.28.3"}}
   :main-opts
   ["-m" "nrepl.cmdline"
    "--bind" "localhost"
    "--middleware"
      "[cider.piggieback/wrap-cljs-repl,
        cider.nrepl/cider-middleware]"]}}
 :deps
 {org.clojure/clojurescript {:mvn/version "1.10.758"}}}
  \end{clojure*}
\end{english}

Запустите проект командой

\begin{english}
  \begin{bash}
> clj -M:nrepl/piggieback
  \end{bash}
\end{english}

На первый взгляд не произойдет ничего особенного: запустится сеанс nREPL. Перейдите в Emacs и подключитесь командой \code{cider-connect-cljs} (клавиши \code{C-c M-c}). Это новая команда, которой мы ещё не пользовались. Если всё настроено без ошибок, Emacs запросит тип REPL: браузер, Node.js, shadow-clj и другие, которых мы не касались в этой главе.

Выберите пункт <<node>> или <<browser>>, чтобы запустить REPL в нужном окружении. Теперь любой код, выполненный в Cider, будет вычислен силами JavaScript, а не JVM. Перейдите в буфер \code{*cider-repl*} и проверьте объекты \code{js/window} или \code{js/process}. Откройте любой cljs-файл и выполните код командами \code{cider-eval-...}.

Кроме выполнения кода, вам доступен переход к определению (\code{cider-find-var}), загрузка буфера (\code{cider-load-buffer}), запуск тестов и многое то, что мы уже рассмотрели. Исключением станут отладка и профилирование: эти техники не работают в ClojureScript, поскольку полагаются на JVM.

\subsection{Прочие сведения}

ClojureScript слишком заманчив, чтобы так быстро расстаться с ним. Перечислим библиотеки из этой области, не погружаясь в них слишком глубоко. Раздел носит обзорный характер и направлен на то, чтобы разжечь в читателе интерес.

\def\urlrenatal{https://github.com/drapanjanas/re-natal}

\index{библиотеки!Re-Natal}
\index{Re-Natal}

Проект \footurl{Re-Natal}{\urlrenatal}[Re-\-Natal] служит для разработки приложений под iOS и Android на ClojureScript. Это обёртка над фреймворком React Native. Сильно упрощая, его можно описать как процесс на Node.js, который управляет деревом компонентов. Каждый компонент представляет нативный виджет, поэтому приложение на React Native выглядит естественно на всех платформах.

Re-Natal устанавливает сеанс REPL с процессом Node в устройстве или эмуляторе. Подключившись из редактора, вы получите полный контроль над системой. Представьте, насколько удобно выполнять код на устройстве, не дожидаясь сборки приложения. Например, послать HTTP-запрос или получить снимок с камеры. Это упрощает разработку, позволяет проверить разные сценарии, в том числе негативные (проблемы связи, нет доступа к камере и другие).

\index{библиотеки!React Native}
\index{React Native}
\index{Яндекс}
\index{компании!Яндекс}

\def\urlrnyayt{https://www.youtube.com/watch?v=WOMnm8mrWFE}

Разработка на React Native противоречива: в ней есть как преимущества, так и недостатки. Предлагаем читателю доклад Андрея Мелихова \footurl{<<Как я полюбил и возненавидел React Native>>}{\urlrnyayt}[React Native]. Автор взвешенно объясняет, почему Яндекс инвестировал усилия в React Native, но в итоге отказался от него. Впрочем, вы не обязаны следовать IT-гигантам. Если нужно простое приложение под обе платформы и вы знаете Clojure, рассмотрите Re-Natal: возможно, он сэкономит время.

\def\urlkrell{https://github.com/vouch-opensource/krell}

\index{библиотеки!Krell}
\index{Krell}

Ещё одна адаптация React Native называется \footurl{Krell}{\urlkrell}[Krell]. Она компилирует код на ClojureScript и загружает в устройство. Как и Re-Natal, Krell запускает REPL с полным доступом к среде исполнения.

\def\urlespruino{https://www.espruino.com/Features}

\index{устройства!Espruino}
\index{Espruino}

Говоря об устройствах, нельзя обойти стороной Espruino~--- микроконтроллер стоимостью 20 долларов. От Arduino и аналогов он отличается тем, что код под него пишут не на C/C\Plus\Plus, а на JavaScript. Из-за ограничений Espruino покрывает не все возможности JavaScript, но основную их часть. Технические детали вы найдете на \footurl{странице проекта}{\urlespruino}[Espruino].

\def\urlyodo{https://amperka.ru/product/yodo}

\index{устройства!Йодо}
\index{Йодо}
\index{Амперка}
\index{компании!Амперка}

Российская фирма Амперка выпускает \footurl{набор Йодо}{\urlyodo}[Йодо] для обучения детей программированию. В набор входят плата Espruino, датчики, провода и брошюра с проектами. В Амперке отлично адаптировали Espruino для детей. К каждому датчику прилагается модуль с документацией на русском языке. Модули загружаются с npm-сервера Амперки и оперативно обновляются. Работают форум и команда поддержки.

Запуск ClojureScript на устройстве со множеством датчиков~--- крайне интересное занятие. Можно сделать игрушку, будильник, телеграф или подобие умного дома. Более амбициозный проект~--- инкубатор яиц, где влажность и температура выводятся на дисплей, а при отклонении от нормы включается сигнал.

Сказанное выше относится к устройствам, в сердце которых Node.js. Ещё больше разнообразия ждёт вас в разработке под браузер.

\def\urlshadowclj{https://github.com/thheller/shadow-cljs}

\index{библиотеки!Shadow-cljs}
\index{shadow-cljs}

Проект \footurl{Shadow CLJS}{\urlshadowclj}[Shadow CLJS] можно описать как улучшенный сборщик ClojureScript. Он предлагает ускоренную компиляцию, когда собирается не весь проект, а только изменённые файлы. Shadow CLJS поддерживает кеш компиляции, живую перезагрузку кода, удобный REPL, сборку под разные платформы (браузер, node, расширение Chrome) и многое другое.

\index{библиотеки!Figwheel}
\index{Figwheel}

\def\urlfigwheel{https://figwheel.org}

Библиотека \footurl{Figwheel}{\urlfigwheel}[Figwheel] сокращает время между написанием кода и результатом в браузере. Figwheel запускает сервер, который следит за изменениями в файлах ClojureScript. Как только файл изменился, Figwheel отправляет браузеру его скомпилированную версию. Браузер обновляет страницу без перезагрузки, что удобно при разработке.

\def\urlweasel{https://github.com/nrepl/weasel}

\footurl{Weasel}{\urlweasel}[Weasel]~--- ещё один REPL для браузера. Для транспорта он использует веб-сокет, а не долгий POST-запрос, как это делает обычный REPL. Weasel работает в паре с Piggieback, и его настройка аналогична шагам, что мы рассмотрели.

\def\urllumo{https://github.com/anmonteiro/lumo}

\footurl{Lumo}{\urllumo}[Lumo][3mm]~--- мощный инструмент для разработки на ClojureScript. Одно из главных его преимуществ~--- REPL, скомпилированный для Node.js. Чтобы запустить его, не нужно устанавливать JVM и компилятор ClojureScript. Достаточно скачать пакет lumo-cljs из npm и вызвать одну функцию.

\def\urlcljcompilerjs{https://github.com/google/closure-compiler-js}

Другая особенность Lumo~--- компиляция силами JavaScript без участия JVM. Это возможно при помощи клона Closure Compiler, написанного \footurl{на JavaScript}{\urlcljcompilerjs}[Closure Compiler JS]. К сожалению, оба проекта сданы в архив и, скорее всего, не будут развиваться.

\subsection{REBL}

\def\urlrepl{https://docs.datomic.com/cloud/other-tools/REBL.html}

\index{REBL}

Наш обзор замыкает \footurl{проект RE\textbf{B}L}{\urlrepl}[REBL]~--- графический REPL фирмы Cognitect. Буква B в названии означает browse~--- обозревать, что указывает на богатые возможности REBL в работе с данными.

\def\urldatomicconsole{https://docs.datomic.com/on-prem/other-tools/console.html}

\index{Datomic}

REBL берет начало от базы данных Datomic, точнее её \footurl{графической консоли}{\urldatomicconsole}[Datomic console]. Со временем консоль вынесли в отдельную библиотеку, а веб-интерфейс заменили на JavaFx~--- так и получился REBL. Он нацелен на общую работу с данными, а не только Datomic.

Особенность REBL в том, что его код закрыт. Это непривычно в мире Clojure, где преобладает открытый код. По той же причине усложнена установка: чтобы получить REBL, нужно заполнить форму на сайте Cognitect, после чего на почту придет ссылка на архив. Распакуйте его и запустите скрипт \code{install}. Он скопирует jar-файлы в локальную папку Maven (по умолчанию \code{\tilde{}/.m2}).

Приведём минимальный \code{deps.edn} для запуска REBL на странице справа. Основную его часть занимают зависимости JavaFx \lis{fig:rebl-deps}.

\index{JavaFx}

\begin{figure}[ht!]

\begin{english}
  \begin{clojure*}{fontsize=\small}
{:aliases
 {:rebl
  {:extra-deps
    {com.cognitect/rebl          {:mvn/version "0.9.245"}
     org.openjfx/javafx-fxml     {:mvn/version "15-ea+6"}
     org.openjfx/javafx-controls {:mvn/version "15-ea+6"}
     org.openjfx/javafx-swing    {:mvn/version "15-ea+6"}
     org.openjfx/javafx-base     {:mvn/version "15-ea+6"}
     org.openjfx/javafx-web      {:mvn/version "15-ea+6"}}
   :main-opts ["-m" "cognitect.rebl"]}}}
  \end{clojure*}
\end{english}

\captionsetup{labelformat=lis}
\caption{Зависимости REBL}

\label{fig:rebl-deps}

\end{figure}

Выполните \code{clj -M:rebl} и дождитесь загрузки библиотек. Появится окно, разбитое на несколько частей: ввод кода, результат вычислений и другие. Область ввода напоминает простой редактор: в нём работают подсветка синтаксиса и балансировка скобок. Можно узнать метаданные вычислений и вывести несложные диаграммы.

\pagebreaklarge

\def\urlreblvideo{https://www.youtube.com/watch?v=c52QhiXsmyI}

\index{Хэллоуэй, Стюарт}
\index{Stuart Halloway}

REBL поддерживает интеграцию с Cider и nREPL, горячие клавиши, просмотр метаданных. Эти и другие возможности описаны на странице Cognitect. \footurl{В одноименном видео}{\urlreblvideo}[REBL~--- Stuart Halloway] Стюарт Хэллоуэй (Stuart Halloway), основатель фирмы, вживую показывает, как пользоваться REBL.

На этом мы закончим обзор библиотек и подведём итоги главы.

\section{Заключение}

\index{REPL}

Аббревиатура REPL означает Read, Eval, Print, Loop~--- прочитать, выполнить, напечатать, повторить. Это режим программы, когда введённый код сразу выполняется. С помощью REPL программист проверяет код по мере написания. Так он раньше поймет, какие данные приводят к ошибкам и куда двигаться дальше.

REPL~--- один из столпов в языках семейства Лисп. Первые Лисп-машины работали в режиме приглашения, и это правило сохранилось до наших дней. От любой Лисп-системы ожидают интерактивного режима. Современные языки тоже предлагают REPL (\code{python}, \code{irb}), но их возможности крайне скудны.

Первые машины читали ввод с клавиатуры, но со временем стало ясно: REPL может работать по сети. Появились сетевые версии REPL со своими протоколами. В мире Common Lisp это проекты Slime и Swank, в Clojure~--- nREPL. В отличие от простого REPL, сетевая версия поддерживает несколько клиентов одновременно. Сервер отвечает на сообщения асинхронно: на один запрос клиент может получить несколько ответов.

Чтобы подключиться к REPL из редактора, нужен специальный модуль (плагин, расширение). Наиболее продвинутый клиент для nREPL называется Cider~--- модуль для Emacs. В свою очередь, Emacs~--- один из самых старых редакторов. Порог входа в него выше, чем у современных VS Code или Sublime Text, но и возможностей, накопленных за сорок лет развития, гораздо больше.

Сетевой REPL полезен в виртуальном окружении, например Docker. В редких случаях его оставляют в промышленном запуске, однако это небезопасно и требует настроек.

nREPL расширяют при помощи middleware~--- промежуточных слоев, устроенных по принципу Ring. Библиотека \code{cider-nrepl} добавляет поддержку тестов, навигацию по коду, отладку, трассировку, словом, все то, что предлагают современные IDE.

ClojureScript~--- это язык, похожий на Clojure, который компилируется в JavaScript. Он поддерживает REPL в той же мере, что и Clojure. Код выполняется в среде JavaScript, роль которой играет браузер, движок Node.js, устройство или эмулятор.

Из-за разнообразия JavaScript создано множество REPL'ов к нему. Все они предлагают те или иные преимущества в работе с проектом. С помощью библиотеки Piggieback легко связать ClojureScript с редактором. С ней код на устройстве можно выполнить прямо из Emacs, что упрощает работу.

\begin{framed}

Опытные программисты знают, что REPL~--- не просто приятная возможность языка. Это иной процесс разработки: он кардинально отличается от привычной модели, когда код сперва пишут, а потом запускают. В REPL написание кода и запуск чередуются мелкими итерациями. Так мы исследуем код задолго до запуска в бою.

Если вы интересуетесь языками семейства Лисп, потратьте время на настройку редактора и окружения. Эти затраты окупаются: с REPL ваша производительность возрастет многократно. Когда вы овладели REPL, разработка на других языках без него покажется немыслимой.

\end{framed}
